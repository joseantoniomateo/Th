\subsection{Extended Timed-Arc Petri Nets} \label{sec:def}
%We shall now give some preliminaries in order to define the model
%of extended timed-arc Petri nets. 
%We consider only discrete time
%as for closed intervals the reachability/soundness problems
%are equivalent with the continuous time variant as discussed in 
%Section~\ref{sec:cont}.
A \emph{discrete timed transition system} (DTTS) 
is a triple $\left(\proc, \act,\rightarrow\right)$
where $\proc$ is the set of states, $\act$ is the set of actions
and $\rightarrow\: \subseteq \proc \times (\act \cup \nnul)  \times \proc$ is the 
transition relation written as $s \trans{a} s'$ whenever $(s,a,s') \in \rightarrow$. In continuous time,
the transition relation is defined as $\rightarrow\: \subseteq \proc \times (\act \cup \rnul)  \times \proc$.
If $a \in \act$ then we call it a \emph{switch transition}, if
$a \in \nnul$ we call it a \emph{delay transition}.
%By $\rightarrow^{*}$ we denote the reflexive and transitive closure of 
%the relation
%$\rightarrow  \eqdef \bigcup_{a \in \act} \trans{a} \; \cup \; \bigcup_{d \in \nnul} \trans{d}$. 
We also define the set of \emph{well-formed time intervals} 
by: $$\int =\{ [a,a] \;|\; [a,b] \;|\; [a,b) \;|\; (a,b] \;|\; (a,b) \;|\; [a,\infty) \;|\; (a,\infty) \mid a,b \in \nnul,a < b\}$$
Moreover, its subset: $$\intinv = \{ [0,0] \;|\; [0,b] \;|\; [0,b) \;|\; [0,\infty) \mid b \in \nnul\}$$
is used in age invariants. Let us note that we use a discrete time semantics in our definitions 
to avoid the duplication for the case of continuous time. Nevertheless, it is enough to include $\rnul$ instead of $\nnul$ to provide a continuous time semantics.


\begin{definition}[(Extended timed-Arc Petri Net)] \label{defetapn}  
An \emph{extended timed-arc Petri net} 
(ETAPN) is a 9-tuple $N = \tapntuple$ where 
\begin{itemize}
\item $P$ is a finite set of \emph{places},
\item $T$ is a finite set of \emph{transitions} 
such that $P \cap T = \emptyset$, 
\item $\Turg \subseteq T$ is the set of \emph{urgent transitions},
\item $\ia \subseteq P \times T$ is a finite set of \emph{input arcs},
\item $\oa \subseteq T \times P$ is a finite set of \emph{output arcs},
\item $\cfunction : \ia \rightarrow \int$ is a \emph{time constraint function} assigning 
guards %(time intervals) 
to input arcs,
\item $\wfunction : \ia\cup \oa \rightarrow \mathbb{N}$ is a function assigning \emph{weights} to input and output arcs,
\item $\type : \ia \cup \oa \rightarrow \types$ is a \emph{type function} assigning a type to all arcs where $\types = \{\normal, \inhib\} \cup \{\transporti \mid j \in \mathbb{N} \}$ such that  
\begin{itemize}
\item if $\type(a) = \inhib$ then $a \in \ia$ and $\cfunction(a)=[0,\infty]$, 
\item if $(p,t) \in \ia$ and $t \in \Turg$ then $\cfunction((p,t))=[0,\infty]$,
\item if $\type((p,t)) = \transporti$ for some $(p,t) \in \ia$ then there is exactly one $(t,p^{\prime}) \in \oa$ such that $\type((t,p^{\prime})) = 
\transporti$, 
%and moreover $\wfunction((p,t))=\wfunction((t,p^{\prime}))$,
\item if $\type((t,p^{\prime})) = \transporti$ for some $(t,p^{\prime}) \in \oa$ then there is exactly one $(p,t) \in \ia$ such that $\type((p,t)) = 
\transporti$, 
%and moreover $\wfunction((p,t))=\wfunction((t,p^{\prime}))$,
\item if $\type((p,t)) = \transporti = \type((t,p^{\prime}))$ 
then $\wfunction((p,t))=\wfunction((t,p^{\prime}))$,
\end{itemize}
\item $\inv : P \rightarrow \int^{inv}$ is a function assigning \emph{age invariants} to places.
\end{itemize}
\end{definition}

\begin{remark}
Note that for transport arcs we assume that they come in pairs (for
each type $\transporti$) so that their weights match.
Also for inhibitor arcs and for input arcs to urgent transitions, we
require that the guards are $[0,\infty]$. This restriction is important
for some of the results presented in this Thesis and it also guarantees that 
we can use DBM-based algorithms in the tool TAPAAL~\cite{DJJJMS:TACAS:12}.
\end{remark}

The ETAPN model is not monotonic, meaning
that adding more tokens to markings can disable time delays or
transition firing.
Therefore we define a subclass of 
ETAPN where the monotonicity breaking features are not allowed.
In the literature such nets are often considered as the standard
timed-arc Petri net model~\cite{BLT:90,Hanisch:93} but we add the 
prefix monotonic for clarity reasons. 

\begin{definition}[(Monotonic timed-arc Petri net)] \label{deftapn}
A \emph{monotonic timed-arc Petri net} 
(MTAPN) is an extended timed arc Petri net 
with no urgent transitions ($\Turg=\emptyset)$, no age invariants
($\inv(p) = [0,\infty]$ for all $p \in P$) and no 
inhibitor arcs ($\type(a) \not= \inhib$ for all $a \in \ia$).
\end{definition}


%Let $N = \tapntuple$ be a ETAPN and $P^\prime \subseteq P$, the projection $N|_{P^\prime}$ 
%is the net \ensuremath{(P^\prime, T, \ia^\prime,\allowbreak \oa^\prime, \cfunction^\prime, 
%\wfunction^\prime, \type^\prime, \inv^\prime)}, 
%where $\ia^\prime=\ia \cap (P^\prime \times T)$, $\oa^\prime=\oa \cap (T \times P^\prime)$,
%$\cfunction^\prime : \ia^\prime \rightarrow \int$, $\wfunction^\prime : \ia^\prime \cup \oa^\prime \rightarrow \mathbb{N}$,
%$\type^\prime : \ia^\prime \cup \oa^\prime \rightarrow \types$, and $\inv^\prime : P^\prime \rightarrow \int^{inv}$. From now on, we will denote by 
%$P_s$ the set of places shared by various nets. Then, let $N$, $N^\prime$ be two ETAPNs such that $P \cap P^\prime \subseteq P_s$, the disjoint union of $N$ and $N^\prime$ is a ETAPN \ensuremath{(P^{\prime\prime},T^{\prime\prime},\ia^{\prime\prime}, \oa^{\prime\prime},\cfunction^{\prime\prime},\wfunction^{\prime\prime},\type^{\prime\prime}, \inv^{\prime\prime})}, where $P^{\prime\prime}= P\cupdot P^\prime, T^{\prime\prime}=T\cupdot T^\prime, \ia^{\prime\prime}=\ia\cupdot \ia^\prime,\oa^{\prime\prime}=\oa \cupdot \oa^\prime,\cfunction^{\prime\prime}:\ia^{\prime\prime}\cup \oa^{\prime\prime}\rightarrow \int, \wfunction^{\prime\prime}: \ia^{\prime\prime}\cup \oa^{\prime\prime}\rightarrow \mathbb{N}, 
%\type^{\prime\prime} : \ia^{\prime\prime} \cup \oa^{\prime\prime} \rightarrow \types, \text{ and } \inv^{\prime\prime} : P^{\prime\prime} \rightarrow \int^{inv}$.

Before giving the formal semantics of the model, let us fix some notation.
We will recall some definitions presented previously, although applying them
to this kind of nets.
Let $N = \tapntuple$ be an ETAPN. 
%Let $F\eqdef \ia \cup \oa$. 
We denote by ${}^\bullet x \eqdef 
\{y \in P \cup T \mid (y,x) \in (\ia \cup \oa),\ \type((y,x)) \neq \inhib \}$ 
the preset of a transition or a place $x$.
Similarly, the postset $x^\bullet$ is defined as 
$x^\bullet \eqdef \{y \in P \cup T \mid (x,y) \in (\ia \cup \oa) \}$.
Let $\mathcal{B}(\nnul)$ be the set 
of all finite multisets over $\nnul$. A \emph{marking} $M$ on $N$ 
is a function $M : P \longrightarrow \mathcal{B}(\nnul)$ 
where for every place $p \in P$ and 
every token $x \in M(p)$ we have $x \in \inv(p)$. In other words
all tokens have to satisfy the age invariants. 
%The projection of $P^\prime \subseteq P$ in $M$ is a function 
%$M|_{P^\prime} : P^\prime \longrightarrow \mathcal{B}(\nnul)$.
The set of all markings in the net $N$ 
is denoted by $\mathcal{M}(N)$.

We write $(p,x)$ to denote a token at a place $p$ with the 
age $x\in \nnul$. Then $M=\multiset{(p_1,x_1),(p_2,x_2),\dots ,(p_n,x_n)}$ 
is a multiset representing a marking $M$ with $n$ tokens of 
ages $x_i$ in places $p_i$. We 
define the size of a marking as $|M| = \sum_{p\in P}|M(p)|$ where 
$|M(p)|$ is the number of tokens located in the place $p$.

%A marked ETAPN 
%$(N,M_0)$ is a TAPN N together with an initial marking $M_0$ with all tokens of age $0$. 

\begin{definition}[(Enabledness)]
\label{def:enabledness}
 Let $N = \tapntuple$ be an ETAPN. 
We say that a transition $t \in T$ is \emph{enabled} in a marking $M$ by the 
multiset of timed tokens \newline 
$\inn = \multiset{(p,x_{p}^1), (p,x_{p}^2), \dots ,(p,x_{p}^{\wfunction ((p,t))})\mid 
p \in {}^\bullet t} \subseteq M$ and by the multiset of tokens
$\out = \multiset{ (p^{\prime},x_{p^{\prime}}^1),
           (p^{\prime},x_{p^{\prime}}^2),
\dots ,\allowbreak
(p^{\prime},x_{p^{\prime}}^{\wfunction ((t,p^{\prime}))}) 
\mid p^{\prime} \in t^\bullet }$ if
\begin{itemize}
\item for all input arcs except the inhibitor arcs, the tokens from $\inn$ satisfy the age guards of the arcs, i.e. 
%$$\forall(p,t) \in \ia, x_p^i \in \cfunction((p,t))\text{ for }1\leq i\leq w((p,t)) $$ 
$$\forall(p,t) \in \ia. \type((p,t)) \neq \inhib \Rightarrow  x_p^i \in \cfunction((p,t))\text{ for }1\leq i\leq w((p,t)) $$ 
%\item for each place $p$ in ${}^\bullet t$ there are $\wfunction ((p,t))$ tokens from $p$ in $\inn$, i.e. $$\forall p\in {}^\bullet t. \wfunction ((p,t))= n_{p} $$
%\item for each place $p^{\prime}$ in $t^\bullet $ there are $\wfunction ((t,p^{\prime}))$ tokens from $p^{\prime}$ in $\out$, i.e. $$\forall p^{\prime}\in t^\bullet . \wfunction ((t,p^{\prime}))= m_{p^{\prime}} $$
\item for any inhibitor arc pointing from a place $p$ to the
transition $t$, the number of tokens in $p$ is smaller than the weight of the arc, i.e.
$$\forall(p,t) \in \ia. \type((p,t)) = \inhib \Rightarrow|M(p)|<\wfunction ((p,t))$$ 
%$$\forall(p,t) \in \ia. \type((p,t)) = \inhib \Rightarrow \nexists x \in M(p). x \in \cfunction((p,t))$$
\item for all input arcs and output arcs which constitute a transport arc, 
the age of the input token must be equal to the age of the output token and satisfy the invariant of the output place, i.e.
\begin{eqnarray*}
&\forall(p,t) \in \ia. \forall(t,p^{\prime}) \in \oa.\type((p,t)) = \type((t,p^{\prime})) 
= \transporti \\
&\Rightarrow \big( x_p^i = x_{p^{\prime}}^i \wedge x_{p^{\prime}}^i \in 
\inv(p^{\prime})\big) \text{ for } 1\leq i \leq w((p,t))
\end{eqnarray*}
\item for all normal output arcs, the age of the output token is $0$, i.e. $$\forall(t,p^{\prime}) \in \oa. \type((t,p^{\prime})) = \normal \Rightarrow x_{p^{\prime}}^i = 0 \text{ for }1\leq i \leq w((p,t)).$$ 
\end{itemize}
\end{definition}

A given ETAPN $N$ %=\tapntuple$ 
defines a DTTS $T(N)\eqdef (\markingsof(N),T,\rightarrow)$
where states are the markings and the transitions are as follows: 
\begin{itemize}
\item If $t\in T$ is enabled in a marking $M$ by the  multisets of
tokens $\inn$ and $\out$ then $t$ can \emph{fire} and produce 
the marking $M^{\prime} = (M \smallsetminus \inn) \uplus \out$ 
where  $\uplus$ is the multiset sum operator and $\smallsetminus$ is the multiset 
difference operator; we write $M \trans{t} M^{\prime}$ for this 
switch transition.
\item A time \emph{delay} $d \in \nnul$ is allowed in $M$ if
\begin{itemize}
\item $(x+d) \in I(p)$ for all $p \in P$ and for all $x \in M(p)$, and
% i.e. by delaying $d$ time units no token violates any of the age invariants, 
%and
\item if $M \trans{t} M'$ for some $t \in \Turg$ then $d=0$.
 %there is at least one urgent transition enabled in $M$ then
 %     $d=0$, i.e. enabled urgent transitions disallow time passing.
\end{itemize}
By delaying $d$ time units in $M$ we reach the marking $M^{\prime}$ defined as
$M^{\prime}(p) = \multiset{x+d \mid x \in M(p)}$ for all $p \in P$; 
we write $M \trans{d} M^{\prime}$ for this delay transition.
\end{itemize}

%A computation of a net $N$ from the initial marking $M_0$ is
%$M_0 \rightarrow M_1\rightarrow \cdots \rightarrow M_n$ is 
%denoted by $\{M_i\}_{i=0}^{n}$ 
%and we call it a \emph{run}. If the sequence is infinite, we write 
%$\{M_i\}_{i\geq 0}$. Moreover, we write $M \Rightarrow^* M^{\prime}$ if  
%$M^{\prime}$ is reachable from $M$ and $[M\rangle$ represents the set of reachable markings of $M$.

\noindent Let 
$\trans{} \eqdef \bigcup_{t \in T} \trans{t} \cup \bigcup_{d \in \nnul} \trans{d}$.
Again, the set of all markings reachable %in the net $N$ 
from a given marking $M$ is denoted by 
$[M\rangle \eqdef \{ M' \mid M \trans{}^* M' \}$.
By $M \trans{d,t} M'$ we denote that there is a marking $M''$
such that $M \trans{d} M'' \trans{t} M'$.

A marking $M$ is a \emph{deadlock} if there is no $d \in \nnul$ and
no $t \in T$ such that $M \trans{d,t} M'$ 
for some marking $M'$.
A marking $M$ is \emph{divergent} if for any $d \in \nnul$
we have $M \trans{d} M'$ for some $M'$.


%\section{Finite Abstractions for Bounded ETAPNs}

In general, ETAPNs are infinite in two dimensions. The number of tokens
in reachable markings can be unbounded and even for bounded nets
the ages of tokens can be arbitrarily large. 


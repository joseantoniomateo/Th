\chapter{Conclusions, Contributions and Future Works}\label{chapter:c6}
\markboth{Chapter~\ref{chapter:c6} Conclusions, Contributions and Future Works}{}

This chapter presents the conclusions of this Thesis, reviews the contributions of this work, and suggests some possible future lines of research. It also includes a list of the publications obtained due to the work presented here as well as other contributions obtained as a result of collaborations with researchers of different research groups.

In this Thesis, two different works have been presented the results obtained in two closely related topics. On the one hand, the design of an operational semantics
for a brand new language, called BPELRF, was introduced as well as a visual formalism for it. On the other hand, a formal model, based on timed-arc Petri nets, suitable to model the workflow of a system was presented in the second part of the Thesis.  


Concerning the first part, we were aware of the recent boom in distributed systems with the advent of Cloud Computing. With our previous experience in modelling web services compositions (this work is focused on WS-BPEL, but other members of our research group have broad experience in WS-CDL), we considered that a natural step was to take advantage of this experience in order to provide a formal definition of this emerging technology. Thus, after some discussion and study of the state-of-the-art in this area, we decided to create our own language to model and analyse such kind of systems. Due to the absence of standardization, a standardised specification language (WSRF) was selected as basis of BPELRF language. This language is closely related to WSN and, therefore, the integration of a notification scheme was the natural evolution. Clearly, the next step was to provide the user with a visual formalism and a tool that supports the mathematical theory presented here. The benefit of using this language with respect to other specification languages such as UML was presented in the introduction.

Regarding to the second part of the Thesis, a formal model for workflow nets with time restrictions was developed. This model serves to calculate total execution times for different tasks as well as to impose some deadlines to these (or different) tasks. Moreover, a correctness criterion called \emph{soundness} was extended to the timed setting. In addition, we defined a new property called \emph{strong soundness} to deal with strong deadlines. The decidability of these properties was
studied with the discrete semantics. Some of these results can be extended to the continuous semantics. A tool supporting the creation and analysis of workflow nets
was developed. This tool was used in some interesting case studies. 

\section{Contributions}\label{contributions}
\markright{~\ref{contributions} Contributions}
%
In this section, the contributions resulting of the development of this Thesis are summarised in the following list:
%
\begin{itemize}
%
\item The definition from scratch of the syntax and semantics of the formal language BPELRF.
%
\item The definition of a visual model in terms of Petri nets for this language.
%
\item The development of a tool that supports this language.
%
\item The definition of a new formalism to model workflow nets to deal with time restrictions.
%
\item The definition of a new correctness criterion for this kind of nets.
%
\item The study of the decidability of this criterion as well as the other correctness property (soundness).
%
\item The development of a tool yielding all the theory of timed-arc Workflow nets.
\end{itemize}
%
\section{Future Works}\label{future}
\markright{~\ref{future} Future Works}
%
Obviously, the work presented in this Thesis has several possible directions. Some of them are the following:
%
\begin{itemize}
%
\item Concerning the language BPELRF, it has been left as future work the demonstration of the equivalence between both semantics: SOS semantics and Petri nets semantics.
%
\item Moreover, BPELRF tool can be extended to deal with the BNF grammar of the language. Now, it only transforms from WS-BPEL code to coloured Petri nets. Nevertheless, it could be useful to transform from WS-BPEL to the BNF grammar and, then, extract the Labelled Transition System. Moreover, it can be interesting to provide Labelled Transition System support.
%
\item As BPELRF is a first step to model Cloud systems, it is appealing to see how virtualisation can be included in it as well as how this complicates the models.
%
\item Regarding the workflow nets part, there are some extensions under development. First, as commented previously, most of the results obtained for the discrete semantics can be extended to continuous semantics, and, therefore, it is fairly interesting to set our scenario in a continuous setting. Nevertheless, there are some theorems  that are not valid for continuous time and, as a consequence, we are trying to find a valid solution for these cases.
%
\item Finally, two different future directions came out during the development of the theory for workflow nets. First, it can be observed that there is no special place in our formalism to model the presence of resources (for instance, specified via WSRF). This kind of nets are called Resource-constrained workflow nets and they were introduce by K. van Hee et al. in \cite{Hee05}. Nevertheless, to the best of our knowledge, no timed extension have been done so far. Moreover, there are some cases that are sound for one instance, but because of shared resources, can deadlock for several instances (see \cite{Juhas10}). We consider that the algorithm to detect this situation can be improved and, therefore, we are studying how to improve it.  
%
%
\end{itemize}
%
\section{Publications and Collaborations}\label{publications}
\markright{~\ref{publications} Publications}
%
Thanks to a great set of collaborators, the work done during these years led to several publications. They include in chronological order international journal papers, international conference papers, and a national conference paper. Two technical reports have also been published and works submitted for publication are mentioned too.
%
\subsection{Journal Papers}
%
\begin{itemize}
%
\item 
Rodr\'iguez, I., D\'iaz, G., Rabanal, P., Mateo, J.A. (2012). A centralized and a decentralized method to automatically derive choreography-conforming web service system. \emph{Journal of Logic and algebraic programming (JLAP)}, volume 81, number 2, pages 127-159.

\end{itemize}
%
\subsection{International Conference Papers}
%
\begin{itemize}
%
\item 
Mateo, J.A., D\'iaz, G., Mart\'inez, E. and Cambronero, M.E. (2010). Modeling Conference Contribution Management Using Web Services, in \emph{International Conference on Internet and Web Applications and Services (ICIW)}, pages. 463-468.
%
\item
Mateo, J.A., Valero, V., Mart\'inez, E. and D\'iaz, G. (2011). Analysis and Verification of Web Services Resource Framework (WSRF) specifications Using Timed Automata Modeling Conference Contribution Management Using Web Services, in \emph{International Conference on Internet and Web Applications and Services (ICIW)}, pages 222-227.
%
\item
Rabanal, P., Mateo, J.A., Rodr\'iguez, I., D\'iaz, G. (2011). Improving the Automatic Derivation of Choreography-conforming Web Services Systems, in    
\emph{}, pages 187-194.  
%
\item 
Mateo, J.A., Valero, V., D\'iaz, G. (2011). An Operational Semantics of BPEL Orchestrations Integrating Web Services Resource Framework, in \emph{International Workshop on Web Services and Formal Methods (WS-FM)},  pages 79-94.
%
\item
Rabanal, P.,  Rodr\'iguez, I., Mateo, J.A., D\'iaz, G. (2011). DIEGO: A Tool for DerIving chorEoGraphy-cOnforming Web Service Systems, in    
\emph{International Conference on Computational Science (ICCS)}, pages 449-458.
%
\item
D\'iaz, G., Llana, L., Valero, V., Mateo, J.A. (2012). Conformance Verification of Normative Specifications using C-O Diagrams, in 
\emph{Workshop on Formal Languages and Analysis of Contract-Oriented Software (FLACOS)}, EPTCS 94, pages 1-10.
%
\item
D\'iaz, M., Valero, V., Maci\`a, H., Mateo, J.A., D\'iaz, G. (2012). BPEL-RF Tool: An Automatic Translation from WS-BPEL/WSRF Specifications to Petri Nets, in
\emph{International Conference on Software Engineering Advances (ICSEA)}, pages 325-330.
%
\item Ruiz, M.C., Mateo, J.A., Maci\`a, H., Pardo , J.J., Olivares, T. (2012). Formal Modelling and Performance Evaluation of a novel Role-Based Routing Algorithm for Wireless Sensor Networks, in \emph{International Conference on Advanced Computing and Communications (ADCOM)}, pages 4-12.  
%
\item Mateo, J.A., Ruiz, M.C., Maci\`a, H., Pardo , J.J., Ortiz, A.M. (2013).
Formal Study of a Novel Network Role-based Routing Intelligent Algorithm, in \emph{International Conference on Computational Science (ICCS)}, 
pages 2525-2528.       
%
\item  Mateo, J.A., Srba, J., S{\o}rensen, M.G. (2014). Soundness of Timed-Arc Workflow Nets, in \emph{International Conference on
Application and Theory of Petri Nets and Concurrency (Petri Nets)} (to appear).
\end{itemize}
%
\subsection{National Conference Papers}
%
\begin{itemize}
%
\item Mateo, J.A., Valero, V., D\'iaz, G. (2011). BPEL-RF: A Formal Framework for BPEL Orchestrations Integrating Distributed Resources, in
\emph{Jornadas de Concurrencia y Sistemas Distribuidos (JCSD)}, pages .      

\item Ruiz, M.C., Mateo, J.A., Maci\`a, H., Pardo , J.J., Olivares, T. (2012). Formal Verification by Using Petri Nets of NORA. M.C. Ruiz, J.A. Mateo, H. Macià, J.J. Pardo, T. Olivares, in \emph{Jornadas de Concurrencia y Sistemas Distribuidos (JCSD)}, pages .

Macia Soler, Hermenegilda; Ruiz Delgado, Mª Del Carmen; Mateo Cortes, Jose Antonio; Francisco Javier Gomez, Calleja; Fernando Royo (2013). 
A rigorous study of a collision resolution protocol, in \emph{Jornadas de Concurrencia y Sistemas Distribuidos (JCSD)}, pages .  
\end{itemize}
%
\subsection{Technical Reports}
%
\begin{itemize}
%
\item Cambronero, M.E., Valero, V., D\'iaz, G. and Mart\'inez, E. (2009). Web Services Choreographies Verification, \emph{Technical Report DIAB-09-04-3}, Computing Systems Department, University of Castilla-La Mancha.
%
%\item Cambronero, M.E., D\'iaz, G., Mart\'inez, E. and Valero, V. (2009). A comparative study between WSCI, WS-CDL, and OWL-S, \emph{Technical Report DIAB-09-04-3}, Computing Systems Department, University of Castilla-La Mancha.
%
\end{itemize}
%
\subsection{Submitted Works}
%
\subsection{Collaborations}
%
International Collaborations and stays at international universities and research groups:
%
\begin{itemize}
%
\item Department of. 
%
\end{itemize}
%
\section{Funds and Grants}\label{funds}
\markright{~\ref{funds} Funds and Grants}
%
The present Thesis has been carried out thanks to the funds received from a number of projects and grants:
%
\begin{itemize}
%
\item Research project: Modelling and Analysis of Composed Web Services Using Formal Techniques (TIN2009-14312-C02-02)\newline
Research project funded by Spanish Ministry of Education \& Science.\newline
Participating Organizations: University de Castilla-La Mancha (Spain).
%
%\item The author of this thesis has been supported by a predoctoral grant from the \textit{Junta de Comunidades de Castilla-La Mancha}, as a part of the program ``\textit{Plan Regional de Investigaci\'{o}n Cient\'{\i}fica, Desarrollo Tecnol\'{o}gico e Innovaci\'{o}n 2005-2010 (PRINCET)}''.
%
\end{itemize}
\section{Proof of Theorem~\ref{thm:discvscont}} \label{app:cont}

Let $N$ be an extended timed-arc Petri net and let $M_0$ be its marking
with integer ages of tokens only.

Let $r$ be an execution 
\begin{equation}\label{eq:execution}
M_0 \trans{d_0,t_0} M_1 \trans{d_1,t_1} M_2 \trans{d_2,t_2} 
\ldots \trans{d_{n-1},t_{n-1}} M_{n}
\end{equation}
where $d_i \in \rnul$ for all $i$, $0 \leq i <n$.
Let $m=max\{|M_i| \text{ s.t. } 0 \leq i \leq n\}$ be the maximum number of
tokens in any of the intermediate markings.

In order to prove that the same sequence of transitions can be executed
also with integer time delays, we will first define a table
with $m$ rows and $n+1$ columns that represents how the tokens
in the net (rows) are moved around in the net by each transition firing
(column $i$ represents the marking $M_i$ that enables the firing of
transition $t_i$ and column $i+1$ the marking $M_{i+1}$ reached after its firing). 
The timing information is forgotten in the table, it merely represents
the consumption of tokens and whether they were produced by normal or
transport arcs. Based on such a table we will later define a difference
constraint system describing all possible delays that enable the firing
of the given transition sequence.

A table \itn{T} for the execution $r$ is a matrix with $m$ rows and
$n+1$ columns such that each element \tai \ of the table, where 
$1 \leq y \leq m$ and $0 \leq i \leq n$, contains either the value 
$\bot$ (unused token) or the pair $(p,f)$ where $p \in P$ represents the location of the token
and $f \in \nnul \cup \{\bullet\}$ is a flag signalling whether
the age of the given token was reset to some given value from $\nnul$
or whether it has not changed (the flag value $\bullet$).
We denote by \taip and \taif
the elements $p$ and $f$ of the pair of \tai, respectively.
If one element of the pair $(p,f)$ is not relevant for our considerations,
we simply write $(p,-)$ or $(-,f)$. We let $y$ to range over
the rows in the table (tokens) and $i$ over the columns in the table
(transition firing steps).

\begin{definition}[Valid table for run $r$] 
A table $T$ for the run $r$ is \emph{valid} if the following conditions
are met.
\begin{description}
\item[a)] Given the initial marking 
$M_0=\{(p_1,x_1),(p_2,x_2),\ldots,(p_k,x_k)\}$, 
the zero column of \itn{T} is defined as
%consists of \itn{T} includes a column \taj{y}{0} such that 
$\taj{y}{0}\eqdef(p_y,x_y)$ if $1\leq y \leq k$, and 
$\taj{y}{0}\eqdef\bot$ if $k <  y \leq m$.
\item[b)] For each column \itn{i} in the table:
\begin{itemize}
\item there is a set $\consumei\subseteq \jindex$ representing 
the $y$-indexes of the tokens  in column $i$
consumed by firing the transition $t_i$ such that for all 
$p \in {^\bullet}t_i$ we have 
$w(p,t_i)=|\{y \in \consumei \mid \taip=p\}|$,
and for all $p \in P \setminus \!^\bullet{t_i}$ we have 
$|\{y \in \consumei \mid \taip=p\}|=0$, 
\item there is a set $\producei\subseteq \jindex$ representing
the $y$-indexes of the tokens in column $i+1$ produced by firing 
the transition $t_i$ such that for all 
$p \in t_i^\bullet$ we have 
$w(t_i,p)=|\{y \in \producei \mid \tajp{y}{i+1}=p\}|$ 
and for all $p \in P\,\setminus\, t_i^\bullet$ we have 
$|\{y \in \producei \mid \tajp{y}{i+1}=p\}|=0$, and
\item there is a bijection 
$\itn{\mathscr{P}}:\jindex\rightarrow\jindex$ that relates the indexes 
of column $i$ with 
those presented in column $i+1$ such that
\begin{enumerate}[i)]
\item if $|\consumei| \leq |\producei|$ and $y \in \consumei$ then $\mathscr{P}(y) \in \producei$,
\item if $|\consumei| \geq |\producei|$ and $\mathscr{P}(y) \in \producei$ then $y \in \consumei$,
\item if $y \in \consumei$ and 
$\type((\taip,t_i))=\transporti= \type((t_i,p'))$ 
then $\mathscr{P}(y)=y$ and $\tajp{y}{i+1}=p'$,
\item if $y \in \jindex \setminus \consumei$ and $\tai\neq\bot$ then $\mathscr{P}(y)=y$ and $\tajp{y}{i+1}=\taip$,
\item if  $y \in \jindex\setminus \consumei$ and $\tai=\bot$ then either $\mathscr{P}(y) \in \producei$, or $\mathscr{P}(y)=y$ and $\taj{y}{i+1}=\bot$, and

%\item if $a \in \producei$ and $\type(t_i,\taip)=Transport_l$ and $\type(p^\prime,t_i)=Transport_l$ for some l then $\mathscr{P}(a)=a$,
\item if $T_{\mathscr{P}(y),i+1}=\bot$ then $y \in \consumei$ or $\tai=\bot$.
\end{enumerate}
\end{itemize}

\item[c)] For each column $i$ in the table:
\begin{itemize}
\item if $\type((p,t_i))=\inhib$ for some $p \in P$ 
then for all $y \in \jindex$, $\taip\neq p$,
\item for all $y \in \producei$, if
${\mathit \type((t_i,T^{place}_{y,i+1}))=\normal}$ 
then ${\mathit T^{flag}_{y,i+1}=0}$ else ${\mathit T^{flag}_{y,i+1}=\bullet}$,
and
%(${\mathit \type((t_i,T^{place}_{y,i+1}))=Transport_\ell}$ for some $\ell$), 
\item if $y \notin \producei$ and $\taj{y}{i+1} \neq \bot$ then ${\mathit T^{flag}_{y,i+1}=\bullet}$.
\end{itemize}
\end{description}
\end{definition}
Let us now assume a  computation 
$$M_0 \trans{0,start} M_1 \trans{3.5,book} M_2 \trans{4.6,pay}  
M_3 \trans{0,empty} M_4 \trans{0,empty}  M_5 \trans{0,empty} 
M_6 \trans{0,success}M_7$$ 
in our running example from Figure~\ref{fig:example}, where $M_0=\{(in,0)\}$.
One possible valid table $T$ for this computation is given
in Table~\ref{validtable}.

\begin{table}
\scriptsize
\begin{tabular}{|c|c|c|c|c|c|c|c|c|}
 \hline
 & $M_0$ & $M_1$ & $M_2$ & $M_3$ & $M_4$ & $M_5$ & $M_6$ & $M_7$\\
\hline
1 & (in,0) & (booking,0)   & (payment,$\bullet$) & (successful,0) & (successful,0) & (successful,0) & (successful,0) & (out,0)\\\hline
2 & $\bot$   & (attempts,0) & (attempts,$\bullet$) & (attempts,$\bullet$) & (attempts,$\bullet$) & (attempts,$\bullet$) & $\bot$ & $\bot$\\\hline
3 & $\bot$   & (attempts,0) & (attempts,$\bullet$) & (attempts,$\bullet$) & (attempts,$\bullet$) & $\bot$ & $\bot$ & $\bot$\\\hline
4 & $\bot$   & (attempts,0) & (attempts,$\bullet$) & (attempts,$\bullet$) & $\bot$ & $\bot$ & $\bot$ & $\bot$\\
\hline
\end{tabular}
\vspace{3mm}
\caption{A valid table for this run.}
\vspace{-3mm}
\label{validtable}
\end{table}

One can verify that there is a valid table for any computation of the
given net $N$. On the other hand, any valid table defines a legal computation 
on untimed markings represented by each column.

\begin{definition}[Untimed marking given by column $i$] \label{def:untmark}
Let $T$ be a valid table and let $0 \leq i < n$.
We define the untimed marking $M_i^u \eqdef \{ \tajp{y}{i} \in P \mid
1 \leq y \leq m  \}$  as a multiset of all places where a token is present
in the column $i$ of the table $T$.
\end{definition}

By the way a valid table is constructed, the reader can easily verify
the validity of the following lemma.
\begin{lemma}[Untimed consistency of a valid table $T$]\label{lemma:untmarking}
Let \itn{T} be a valid table.
Then $M^u_i \trans{t_i} M^{u}_{i+1}$ for all $i$, $0 \leq i < n$,
in the classical (untimed) Petri net semantics.
\end{lemma}

We shall now proceed with defining a set of difference constraint 
inequalities encoding in a sound and complete way 
the timings aspects of the computation given in Equation~(\ref{eq:execution}).

Let the \emph{execution time} of a transition $t_i$
in Equation~(\ref{eq:execution}) be denoted by the variable $e_i$
representing the total time elapsed from the 
initialization until the transition $t_{i}$ is fired. 
In order to construct the system of inequalities over the variables
$e_0, \ldots, e_{n-1}$, we need to define an expression describing
the age of a token $y$ just at the moment the transition $t_i$ is fired.

\begin{definition}[Token-age expression]\label{def:age}
Let $T$ be a valid table.
We define \agexi{y}{i} where $1 \leq y \leq m$ and $0 \leq i < n$
as the expression 
\begin{center}
``$e_i -e_{j-1} +d$'' 
\end{center}
where $j$, $j\leq i$,
is the largest number not greater than $i$
% entry time of the marking that contains the first element of T 
such that $\tajf{y}{j}=d\in \nnul$. 
%By definition, $e_{-1}=\agexi{y}{-1}=0$ for all $y \in \jindex$.
\end{definition}

The intuition is that $\agexi{y}{i}$ expresses, in terms of
the execution time variables, the current age
of token $y$ after the time delay $d_i$ and exactly when 
firing the transition $t_i$.
The correctness of the definition follows from the
requirements on a valid table and the fact that the first column (i=0) of 
any valid table contains only pairs $(p,f)$ where $f \not= \bullet$.

For example, in our running example in Table~\ref{validtable},
the age of token $1$ at the moment the transition \itn{pay} is fired
can be expressed as $e_2 - e_0 + 0$.


%{\bf Remark.} It is important to note that $\agexi{y}{i}$ always exists for all $y \in \jindex$ 
%and $i \in \{-1,0,\ldots,n\}$ since if $\type((t_{i-1}\taip))=\normal$  then
%$\taj{y}{i}=(-,0)$ and, therefore, $j=i$, whereas if $\type((t_{i-1}\taip))=Transport_\ell$ for some $\ell$,
%we will always find such a $j$ since, according to the definition of the bijective function $\mathscr{P}$ (condition iii)) and
%the fact that the multiplicity of the output arc of this transport arc is by definition 1, one can always search backward in the same
%row and find, at least, one element such that $\taif\neq \bullet$.   

We are now ready to define, for a given valid table, a system of inequalities
over the variables $e_i$ that expresses all the timing constraints on the 
firing of transitions, age invariants and urgency.
 
\begin{definition}[Constraint system]\label{def:consys}
Let \itn{T} be a valid table for a run $r$ from
Equation~(\ref{eq:execution}). The constraint system $\mathscr{C}$ for $T$ 
is the set of inequations over the variables
$e_0, e_1, \ldots, e_{n-1}$, containing the constraints 
$\{ e_0 \leq e_1,\ e_1 \leq e_2, \ldots, e_{n-2} \leq e_{n-1}\}$, 
and constructed 
such that for all $p \in P$, $y \in \jindex$ and $i \in \iindex$:
\begin{description}
\item[a)] if $\taip=p$ and $\inv(p)=[0,u]$ where $u \in \nnul$, 
we add the inequality $\itn{age(y,i)\leq u}$ to $\mathscr{C}$,
\item[b)] if $\taip=p$ and $y \in \consumei$ and $(p,t_i) \in \ia$ and $\cfunction((p,t_i))=[a,b]$, we add $a \leq \agexi{y}{i}$ and if $b \not=\infty$ also 
$\agexi{y}{i}\leq b$ to $\mathscr{C}$, and
\item[c)] if $M^u_i$ enables some $t \in \Turg$
then we add $e_i - e_{i-1} = 0$ to $\mathscr{C}$
where by definition $e_{-1}$ is replaced with $0$.
\end{description}
\end{definition}

In our running example (Figure \ref{fig:example}), the constraint system for the valid table
depicted in Table~\ref{validtable} is the following.
\begin{itemize}
\item We add the inequalities
$$e_0 \leq e_1,\ \ \  e_1 \leq e_2,\ \ \ \ldots,\ \ \ e_5 \leq e_6\ .$$
\item For the two nontrivial age invariants, we add
$\agexi{1}{1} \leq 5$ and $\agexi{1}{2} \leq 10$, in other words
we add the constraints
$$e_1-e_0 \leq 5,\ \ \  e_2-e_0 \leq 10 \ . $$
\item Regarding the guards on input arcs, we add for column $0$
the constraint $0 \leq\agexi{1}{0}$, for column $1$ the constraints
$2 \leq \agexi{1}{1}$ and
$\agexi{1}{1} \leq 5$, for column $2$ the constraints
$0 \leq \agexi{1}{2}$ and
$\agexi{1}{2} \leq 10$ and so on for the remaining columns (producing
only trivial constrains that are always satisfied). Hence the following
constraints (listing only the nontrivial ones) are added to the systems:
$$2 \leq e_1 - e_0,\ \ \ e_1 - e_0 \leq 5,\ \ \ e_2-e_0 \leq 10\ .$$
\item Finally, for each column that enables some urgent transition 
(columns $0$, $3$, $4$, $5$ and $6$), we add (recall that $e_{-1}=0$):
$$e_0=0, \ \ \ e_3=e_2, \ \ \ e_4=e_3, \ \ \ e_5=e_4, \ \ \ e_6=e_5\ .$$
\end{itemize}

Observe that the original delays in the trace from Equation~(\ref{eq:execution})
form a solution of the constructed constraint system: 
$e_0=0$,
$e_1=3.5$,
$e_2=8.1$,
$e_3=8.1$,
$e_4=8.1$,
$e_5=8.1$,
$e_6=8.1$.
In fact, there is also an integer solution to the constraint system
(this is not only a coincidence),
e.g.
$e_0=0$,
$e_1=2$,
$e_2=3$,
$e_3=3$,
$e_4=3$,
$e_5=3$,
$e_6=3$,
and such a sequence is executable in our running workflow example.

\begin{lemma} \label{lem:correct}
Let $r$ be a run $M_0 \trans{d_0,t_0} M_1 \trans{d_1,t_1} M_2 \trans{d_2,t_2} 
\ldots \trans{d_{n-1},t_{n-1}} M_{n}$ with $d_i \in \rnul$ in a workflow
net $N$. Then there is a valid table $T$ for $r$
and the corresponding constraint system $\mathscr{C}$ such that
$e_i= \sum_{j=0 \ldots i} d_j$ is a solution of $\mathscr{C}$. 
Moreover, $e_0, e_1, \ldots, e_{n-1}$ is a (real) solution of 
$\mathscr{C}$ if and only if
$M_0 \trans{e_0,t_0} M_1 \trans{e_1-e_0,t_1} M_2 \trans{e_2-e_1,t_2} 
\ldots \trans{e_{n-1}-e_{n-2},t_{n-1}} M_{n}$.
\end{lemma}
\begin{proof}[sketch]
By analysing the requirements for a valid table, we can see that if
a run $r$ can be performed in the net $N$ then we are able to 
design a table that satisfies all the requirements for the untimed
part of the run execution and uses the right tokens in the pairing
bijection such that the corresponding constraint system $\mathscr{C}$
gives the sufficient and necessary
conditions for the execution time variables $e_i$ to produce a valid
computation of the net $N$ with the given sequence of transitions firing. 
Hence the original execution times in the given run are one possible solution of 
the system but any such a solution actually provides a possible
timed execution of the run (note that if a transition $t_i$ is performed 
at time $e_i$ and transition $t_{i+1}$ is performed at time $e_{i+1}$ then
the delay between the execution of these two transitions is $e_{i+1}-e_{i}$).
\end{proof}

We can now summarise and conclude the proof of Theorem~\ref{thm:discvscont}. 
We assumed a run of the net with real delays as in Equation~(\ref{eq:execution}).
Based on this we know that there exists a valid table $T$ for such a run such that 
the constraint system $\mathscr{C}$ for the run, representing all possible 
delays that can execute the transition sequence in~(\ref{eq:execution}), has
a solution corresponding to the delays in (\ref{eq:execution}). This is due
to Lemma~\ref{lem:correct}. 

The constraint system $\mathscr{C}$ is an instance of linear programming 
problem where we used difference constraints only and we are therefore 
guaranteed that the matrix in the linear programming problem is totally
unimodular~\cite{umodmatrix1}. As the system has a solution, it also
has an optimal integral solutions---a general result guaranteed
for any linear programming problem with totally unimodular 
matrix~\cite{umodmatrix2}.
Hence using Lemma~\ref{lem:correct} we know that there is also an execution
of $N$ following the same transitions as in Equation~(\ref{eq:execution})
but with integral delays only. Moreover, the minimum and maximum
execution times can be also achieved by integral delays only. Hence
the proof of Theorem~\ref{thm:discvscont} is completed.



%tapn
%new macros
\newcommand{\Turg}{\ensuremath{T_{\mathit{urg}}}}
\newcommand{\sqrel}[2]{\ensuremath{{#1}\sqsubseteq{#2}}}
\newcommand{\cutsqrel}[2]{\ensuremath{{#1}\sqsubseteq_{cut}{#2}}}
\newcommand{\reachmark}[1]{\ensuremath{[#1\rangle}} % [M>
\newcommand{\itn}[1]{\ensuremath{{\mathit{#1}}}}
\newcommand{\jindex}{\ensuremath{{\mathit{\{1,\ldots,m\}}}}}
\newcommand{\iindex}{\ensuremath{{\mathit{\{0,\ldots,n\}}}}}
\newcommand{\consumei}{\ensuremath{{\mathit{Consume_i}}}}
\newcommand{\consumeij}[1]{\ensuremath{{\mathit{Consume_{#1}}}}}
\newcommand{\producei}{\ensuremath{{\mathit{Produce_i}}}}
\newcommand{\produceij}[1]{\ensuremath{{\mathit{Produce_{#1}}}}}
\newcommand{\tai}{\ensuremath{{\mathit{T_{y,i}}}}}
\newcommand{\taif}{\ensuremath{{\mathit{T^{\mathit{flag}}_{y,i}}}}}
\newcommand{\taip}{\ensuremath{{\mathit{T^{\mathit{place}}_{y,i}}}}}
\newcommand{\taj}[2]{\ensuremath{{\mathit{T_{#1,#2}}}}}
\newcommand{\tajp}[2]{\ensuremath{{\mathit{T^{\mathit{place}}_{#1,#2}}}}}
\newcommand{\tajf}[2]{\ensuremath{{\mathit{T^{\mathit{flag}}_{#1,#2}}}}}
\newcommand{\agexi}[2]{\ensuremath{{\mathit{age({#1,#2})}}}}
\newcommand{\fr}[1]{\ensuremath{{\mathit{fract(#1)}}}}
\newcommand{\inte}[1]{\ensuremath{{\mathit{int(#1)}}}}
\newcommand{\rankeq}{\ensuremath{{\mathit{\cong}}}}
\newcommand{\notrankeq}{\ensuremath{{\mathit{\ncong}}}}
\newcommand{\multiset}[1]{\ensuremath{\mathit{\{\!\!\{#1\}\!\!\}}}}
\newcommand{\alist}[1]{\ensuremath{\mathit{<\!\!#1\!\!>}}}
\newcommand{\rank}[2]{\ensuremath{{\mathit{rank_{#1}(#2)}}}}
\newcommand{\tokens}{\mathcal{T}}
\newcommand{\abstmarking}{\mathbb{M}}
\newcommand{\abstinn}{\mathscr{In}}
\newcommand{\abstout}{\mathscr{Out}}
\newcommand{\abstmarkingsof}[1]{\ensuremath{\mathscr{A}(#1)}}
\newcommand{\abstoken}{\ensuremath{{\mathit{(r,p,\ell,c)}}}}
\newcommand{\absmarktriple}[1]{\ensuremath{{\mathit{#1}}}}
\newcommand{\cupdiamond}{\mathbin{\mathaccent\diamond\cup}} % Added by Jose to denote the merge
\newcommand{\cupdot}{\mathbin{\mathaccent\cdot\cup}} % Added by Jose to denote the disjoint operator
\newcommand{\fire}{\ensuremath{{\mathit{fire((p,t))}}}}
\newcommand{\enable}{\ensuremath{{\mathit{enable((p,t))}}}}
\newcommand{\rankM}[1]{\ensuremath{{\mathit{rank(#1)}}}}
\newcommand{\fireM}{\ensuremath{{\mathscr{F}}}}
\newcommand{\transM}{\ensuremath{{\mathscr{Trans}}}}

\newcommand{\initstack}{\ensuremath{\mathit{InitStack()}}}
\newcommand{\noinf}{\ensuremath{\mathit{maxBound}}}
\newcommand{\sizeofmarking}{\ensuremath{\mathit{size}}}
\newcommand{\tapntuple}{\ensuremath{(P, T, \Turg, \ia,\allowbreak \oa, \cfunction, \wfunction, \type, \inv)}}
%\newcommand{\tawfntuple}{\ensuremath{(P_w \cup P_r, T, \ia,\allowbreak \oa, \cfunction, \wfunction)}} % Added by Jose to model Timed-Arc workflow nets
\newcommand{\rctawfntuple}{\ensuremath{(P, T, \ia,\allowbreak \oa, in, out, \cfunction, \wfunction, \type, \inv)}} % Added by Jose to model Resource Constrained Timed-Arc workflow nets
%\newcommand{\productiontuple}{\ensuremath{(P_p , T, \ia_p ,\allowbreak \oa_p , \cfunction, \wfunction)}} % Added by Jose to model the production net of Resource Constrained Timed-Arc workflow nets
\newcommand{\ptcpntuple}{\ensuremath{(P,T,A,\Sigma,V,G,E,\lambda,D,\pi)}}
\renewcommand{\int}{\ensuremath{\mathcal{I}}}
\newcommand{\intinv}{\int^{\text{inv}}}
\newcommand{\outup}[1]{\mathit{Out}^{#1}}
\newcommand{\inup}[1]{\mathit{In}^{#1}}
\newcommand{\proc}{\ensuremath{\mathit{S}}}
%\newcommand{\lab}{\ensuremath{\mathit{Act}}}
%\newcommand{\act}{\ensuremath{\mathit{Labels}}}
\newcommand{\act}{\ensuremath{\mathit{Act}}}
\newcommand{\queryset}{\mathcal{Q}}
\newcommand{\querytype}{QUERYTYPE}
\newcommand{\places}{\mathit{Places}}
\newcommand{\false}{\mathit{ff}}
\newcommand{\true}{\mathit{tt}}
\newcommand{\ef}{\mathit{EF}}
\newcommand{\cut}{\mathit{cut}}
\newcommand{\caufunc}{\mathit{cau}}
\newcommand{\typefour}{\mathit{IV}}
\newcommand{\typeone}{\ensuremath{\mathit{Inv}}}
\newcommand{\typetwo}{\ensuremath{\mathit{Dead}}}
\newcommand{\typethree}{\ensuremath{\mathit{Std}}}
\newcommand{\cat}{\ensuremath{\mathit{cat}}}
\newcommand{\ia}{\ensuremath{\mathit{IA}}}
\newcommand{\oa}{\ensuremath{\mathit{OA}}}
\newcommand{\cfunction}{\ensuremath{\mathit{g}}}
\newcommand{\wfunction}{\ensuremath{\mathit{w}}}
\newcommand{\type}{\ensuremath{\mathit{Type}}}
\newcommand{\smalltype}{\ensuremath{\mathit{type}}}
\newcommand{\inv}{\ensuremath{\mathit{I}}}
\newcommand{\types}{\mathbf{{Types}}}
\newcommand{\inn}{\ensuremath{\mathit{In}}}
\newcommand{\out}{\ensuremath{\mathit{Out}}}
\newcommand{\inhib}{\mathit{Inhib}}
\newcommand{\transporti}{\ensuremath{\mathit{Transport}_j}}
\newcommand{\transportone}{\ensuremath{\mathit{Transport}_1}}
\newcommand{\transport}{\mathit{Trans}}
\newcommand{\eqMarking}{\equiv}
\newcommand{\gMarking}{>}
\newcommand{\lMarking}{<} 
\newcommand{\enablfunc}{\hat{T}_N}
\newcommand{\nnul}{\mathbb{N}_{0}}
\newcommand{\rnul}{\mathbb{R}^{\geq 0}}
\newcommand{\nuptok}[1]{\mathbb{N}_{0}^{#1}}
\newcommand{\markingsof}{\mathcal{M}}
%\newcommand{\path}[2]{#1 \Rightarrow^* #2}
\newcommand{\qquestion}[3]{\framebox{ \minibox{{\bf Problem:}\ #1 \\ {\bf Instance:} #2 \\ {\bf Question: #3}}}}% Added by Jose to make question environment
\newcommand{\trans}[1]{\stackrel{#1}{\rightarrow}}
\newcommand{\transbdss}[1]{\stackrel{#1}{\rightarrow}_{\textrm{BDS}}}
\newcommand{\transbds}{\rightarrow_{\textrm{BDS}}}
\newcommand{\notrans}{\ensuremath{\nrightarrow}}
\newcommand{\eqdef}{\stackrel{\textrm{def}}{=}}
\newcommand{\powerset}{\mathcal{P}}
\newcommand{\rbigeq}{\mathbb{R}_{\geq 0}}
\newcommand{\cutmold}{\cut(M)_{>}}
\newcommand{\cutyoung}{\cut(M)_{\leq}}
\newcommand{\mold}{M_{>}}
\newcommand{\myoung}{M_{\leq}}
\newcommand{\moldone}{{M_{1}}_{>}}
\newcommand{\moldtwo}{{M_{2}}_{>}}
\newcommand{\real}{\mathbb{R}}
\newcommand{\nbig}{\mathbb{N}}
\newcommand{\causality}{\ensuremath{\mathit{Causality}}}
\newcommand{\mc}{\ensuremath{\mathit{MC}}}
\newcommand{\mcarc}{\ensuremath{\mathit{C}_{arc}}}
\newcommand{\nextInv}{\ensuremath{\mathit{NextInv}}}
\newcommand{\mcplace}{\ensuremath{\mathit{C}_{place}}}
\newcommand{\cmax}{\ensuremath{\mathit{C}_{max}}}
\newcommand{\normal}{\ensuremath{\mathit{Normal}}}
\newcommand{\waiting}{\ensuremath{\mathit{Waiting}}}
\newcommand{\passed}{\ensuremath{\mathit{Passed}}}
\newcommand{\trace}{\ensuremath{\mathit{Trace}}}
\newcommand{\addpw}{\ensuremath{\mathit{AddToPW}}}
\newcommand{\eq}{\ensuremath{_{\mathit{eq}}}}
%\renewcommand{\top}{\ensuremath{\mathit{top}()}}
\newcommand{\pop}{\ensuremath{\mathit{pop}()}}
%\newcommand{\note}[1]{{\color{red} NOTE: #1}}
\newcommand{\ceq}{:=}
\newcommand{\push}{\ensuremath{\mathit{push}}}
\newcommand{\emptystack}{\ensuremath{\mathit{isEmpty}()}}
\newcommand{\validSuc}{\ensuremath{\mathit{validSuccessors}}}
\newcommand{\successor}{\ensuremath{\mathit{successors}}}
\newcommand{\maxrunend}{\ensuremath{\mathit{endOfMaxRun}}}
\newcommand{\cpp}{C{}\texttt{++}}
\newcommand{\inmark}{\ensuremath{\mathit{M_{in}}}}
\newcommand{\outmark}{\ensuremath{\mathit{M_{out}}}}
%\newcommand{\question}{\framebox{}}


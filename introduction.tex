\section{Introduction}
Workflow nets~\cite{Aalst97,Aalst98} were introduced 
by Wil van der Aalst as a formalism
for modelling, analysis and verification of business workflow processes.
%to model, validate and verify business process, \emph{workflow nets} (wf-nets) \cite{Aalst97,Aalst98}. 
The formalism is based on Petri nets abstracting away most of the data 
while focusing on the possible flow in the system. 
Its intended use is in finding design errors 
like the presence of deadlocks, livelocks 
and other anomalies in workflow processes. Such correctness criteria can
be described via the notion of \emph{soundness} (see~\cite{AalstHHSVVW11}) that
requires the option to complete the workflow, guarantees proper termination
and optionally also the absence of redundant tasks. 

After the seminal work on workflow nets, researchers have 
invested much effort in defining new soundness criteria and/or 
improving the expressive power of the original model by adding new features 
and studying the related decidability and 
complexity questions
(it is not in the scope of this paper to list all these works but we
refer to~\cite{AalstHHSVVW11} for a recent overview).
%such as reset or inhibitor arcs, resources, time or multiple instances. 
%It is impossible and is also out of the scope of this paper to summarise here all
%of these works, and, therefore, we refer the interested reader to \cite{AalstHHSVVW11} for further information. 
%Nevertheless, it is worthwhile to
%mention that there is a common characteristic in all of this extended models: undecidability. Thus, soundness (or its different variants)
%is decidable for classical workflow nets \cite{Aalst97}, but, for instance, it is undecidable when we add reset or inhibitor arcs \cite{AalstHHSVVW11}.
In the present paper we consider a quantitative extension of workflow 
nets with timing features, allowing us to argue, among others, 
about the execution intervals of tasks, deadlines and urgent behaviour of workflow processes. Our workflow
model is based on timed-arc Petri nets~\cite{BLT:90,Hanisch:93} 
where tokens carry timing information
and arcs are labelled with time intervals restricting the available
ages of tokens used for transition firing. %We use an extended version
%of the model that allows us to capture advanced timing restrictions
%in a convenient and direct way. 
Let us first informally introduce 
the model on our running example.

The timed-arc workflow net in Figure~\ref{fig:example} 
describes a simple booking-payment workflow
where a web-service provides a booking form followed by online payment.
The whole booking-payment procedure cannot last for more than 10 mi\-nutes
and the booking phase takes at least 2 minutes and 
must be finished within the first 5 minutes. The process can
fail at any moment and the service allows for three additional attempts
before it will terminate with failure. The workflow net 
%in Figure~\ref{fig:example}
consists of six places drawn as circles and nine transitions
drawn as rectangles. Places can contain timed tokens, like the one
of age $0$ in the place $\mathit{in}$ (input place of the workflow).
The tokens present in the net form a marking. 
Places and transitions are connected by arcs
such that arcs from places to transitions contain time intervals restricting
the possible ages of tokens that can be consumed by transition firing.
For simplicity we do not draw time intervals of the form $[0,\infty]$
as they do not restrict the ages of tokens in any way.

\begin{figure}[t]
\begin{tikzpicture}[font=\scriptsize,xscale=1.9,yscale=1.4]
  \tikzstyle{arc}=[->,>=stealth,thick]
	\tikzstyle{transportArc}=[->,>=diamond,thick]
  \tikzstyle{every place}=[minimum size=6mm,thick]
  \tikzstyle{every transition}=[fill=black,minimum width=2mm,minimum height=5mm]
  \tikzstyle{every token}=[fill=white,text=black]
  \tikzstyle{urgency}=[place,fill=white,minimum size=2.0mm,thin]
	\tikzstyle{inhibArc}=[->,>=o,thick]
  \node at (0,0) {};
  \node[place,label=above:$\mathit{in}$] at (0.5,0) (in) {$0$};
	\node[place,label=above:inv: $\leq 5$,] at (1,1.6) (booking) {};
	\node [above] at (1,2.15) {$\mathit{booking}$};
	\node[place,label=above:inv: $\leq 10$] at (3,1.6) (payment) {};
	\node [above] at (3,2.15) {$\mathit{payment}$};
	\node[place,label=above:$\mathit{successful}$] at (5,1.6) (successful) {};
	\node[place,label=above:$\mathit{out}$] at (6,0) (out) {};
	\node[place,label=left:$\mathit{attempts}$] at (2,-1) (attempts) {};
  \node[transition,,rotate=90,] at (1,0) (start) {};
	\node [above] at (0.95,-0.4) {$\mathit{start}$};
  \node[urgency] at (1,0) {};
	\node [above] at (1.55,-0.5) {$3\times$};
  \node[transition,label=$\mathit{book}$] at (2,1.6) (book) {};
  \node[transition,label=$\mathit{pay}$] at (4,1.6) (pay) {}; 
  \node[transition,rotate=90,] at (2,0) (restart1) {}
	 edge [pre, bend right=10, thick,,>=stealth] (booking)
	 edge [pre, thick,>=stealth] (attempts);
 	\draw [->, thick,>=stealth] (restart1) to [bend left=10] (booking);
	\node [above] at (2.35,-0.4) {$\mathit{restart1}$};
	\node[transition,rotate=90,] at (3,0) (restart2) {}
	 edge [pre, thick,>=stealth] (payment)
	 edge [pre, thick,>=stealth] (attempts)
	 edge[post, bend right=10, thick,>=stealth] (booking);
	\node [above] at (3.25,-0.4) {$\mathit{restart2}$};
	\node[transition,rotate=90,] at (4,0) (empty) {}
	 edge [pre, bend left=15, thick,>=stealth] (successful)
	 edge [pre, bend left, thick,>=stealth] (attempts)
	 edge[post, bend right=15, thick,>=stealth] (successful);
	\node[urgency] at (4,0) {};
	\node [above] at (4.25,-0.45) {$\mathit{empty}$};
	\node[transition,rotate=90,] at (5,0) (success) {}
	edge[pre, bend left, >=o, thick](attempts);
  \node[urgency] at (5,0) {};
	\node [above] at (5.25,-0.4) {$\mathit{success}$};
	\node[transition,rotate=90,yshift=0, label=right:$\mathit{fail1\ }$] at (1,-1.5) (fail1) {}
	 edge[pre,>=o, thick](attempts)
	 edge[post, bend right, thick,>=stealth](out);
	\draw[arc] (booking) .. controls (0,1.5) and (0,-1.5) .. (fail1) {};
	\node [above] at (0.05,-0.25) {[5,5]};
	\node[transition,rotate=90,yshift=-0.5, label=right:$\mathit{fail2}$] at (5,-1.5) (fail2) {}
	 edge[pre, bend left=25, thick,>=stealth](payment)
	 edge[pre,>=o, bend left=20, thick](attempts)
	 edge[post, bend right=25, thick,>=stealth](out);
	\node [above] at (3.45,0.52) {[10,10]};
	\draw[arc] (in) -- (start) {};
	\draw[arc] (start) -- (booking) {};
	\draw[transportArc] (booking) -- (book) node[midway,above]{$[2,5]$:1}{};
	\draw[transportArc] (book) -- (payment)  node[midway,above]{:1} {};
	\draw[arc] (payment) -- (pay)  node[midway,above]{$[0,10]$} {};
	\draw[arc] (pay) -- (successful) {};
	\draw[arc] (successful) -- (success) {};
	\draw[arc] (success) -- (out) {};
	\draw[arc] (start) -- (attempts) {};
	%\draw[inhibArc] (attempts) .. controls +(right:3.95cm) and +(-0.2,-1) .. (5,-0.15) {};
		
\end{tikzpicture}
\caption{Booking-payment workflow with timing constraints}
\label{fig:example}
\end{figure}

In the initial marking of the net, the transition $\mathit{start}$
is enabled as it has a token %of an appropriate age 
in its input place.
The transition is urgent (marked with a filled circle), so no time
delay is possible once it gets enabled.
After the $\mathit{start}$ transition is fired, a new token of age 
$0$ arrives to the place $\mathit{booking}$ (initiating the booking phase) 
and three new tokens of age $0$ arrive to the place $\mathit{attempts}$
(in order to count the number of attempts we have before the service fails).
The transition $\mathit{fail1}$ is not enabled as 
the place $\mathit{attempts}$, connected to $\mathit{fail1}$
via the so-called inhibitor arc, contains tokens, inhibiting
$\mathit{fail1}$ from firing. The transition $\mathit{book}$ is not enabled
either as the token's age in the place \itn{booking} 
does not belong to the interval $[2,5]$.
However, after waiting for example $3$ minutes, $\mathit{book}$ %transition 
can fire.  This consumes the token of age $3$ from $\mathit{booking}$ and 
transports it to the place $\mathit{payment}$, preserving its age. This
is signalled by the use of transport arcs that contain the diamond-shaped
tips with index :$1$ (denoting how these arcs are paired). 
%The index is 
%important in a situation when more that one pair of transport arcs is 
%connected to the same transition. If two normal arcs (with standard arrow-shaped tips)
%are used instead of transport arcs, the age of the
%token would be reset to $0$ after being it transported to the place
%$\mathit{payment}$. As we, however, used transport arcs,
%we know that the transition $\mathit{pay}$ can be fired within 
%at most $10$ minutes since the whole booking-payment service started.

At any moment, the booking-payment part of the workflow can be
restarted by firing the transitions $\mathit{restart1}$ or
$\mathit{restart2}$. This will bring the token back to the place
$\mathit{booking}$, reset its age to $0$, and consume one attempt from
the place $\mathit{attempts}$. Once no more attempts are available
and the age of the token in the place $\mathit{booking}$ or
$\mathit{payment}$ reaches $5$ resp. $10$, we can fire
the transition $\mathit{fail1}$ resp. $\mathit{fail2}$ and terminate the
workflow by placing a token into the output place $\mathit{out}$.
Note that the places $\mathit{booking}$ and $\mathit{payment}$ contain
age invariants $\leq 5$ resp. $\leq 10$, meaning that the ages of tokens in 
these places should be at most $5$ resp. $10$. Hence if the service
did not succeed within the given time bound, the workflow will necessarily
fail.
Finally, if the payment transition was executed within 10 minutes
from the service initialization, the transition
$\mathit{empty}$ can now repeatedly remove any remaining tokens in the place
$\mathit{attempts}$ and the transition $\mathit{success}$ terminates 
the whole workflow. As both the transitions
$\mathit{empty}$ and $\mathit{success}$ are urgent, no further time delay
is allowed in this termination phase.
%after the execution of the transition $\mathit{pay}$.

We are concerned with the study of
soundness and strong soundness, intuitively meaning that from any
marking reachable from the initial one, it is always possible
to reach a marking (in case of strong soundness additionally within a fixed
amount of time), having just one token in the place $\itn{out}$.
Moreover, once a token appears in the place $\itn{out}$, it is mandatory
that the rest of the workflow net does not contain any remaining tokens.
One can verify (either manually or using our tool mentioned later)
that the workflow net of our running example is both sound and strongly sound.

\paragraph{Our contribution.}
We define a workflow theory based on timed-arc Petri nets,
extend the notion of soundness from~\cite{Aalst97} to deal with 
timing features and introduce a new notion of strong soundness
that guarantees time-bounded workflow termination. We study
the decidability/undecidability of soundness and strong soundness
and conclude that even though they are in general undecidable,
we can still design efficient verification algorithms for two important
subclasses: monotonic workflow nets 
(not using any inhibitor arcs, age invariants and urgent transitions)
and for the subclass of bounded nets. 
This contrasts to the fact that for example the reachability
question for monotonic workflow nets is already 
undecidable~\cite{RGE:reachability}. 
Moreover, our algorithms 
allow us to compute the minimum and maximum execution times of the
workflow. The theory is developed for
discrete-time semantics but in Section~\ref{sec:cont} we discuss its
relationship to the continuous-time semantics. Finally, we implement
the algorithms given in this paper within the open-source tool 
TAPAAL~\cite{DJJJMS:TACAS:12} and successfully demonstrate on a number of 
case studies the applicability of the theory in real-world scenarios.

\section*{Related Work}

Soundness for different extensions of Petri nets with e.g. 
inhibitor arcs, reset arcs and other features have been studied before,
leading often to undecidability results (for a detailed overview 
see~\cite{AalstHHSVVW11}). We shall now focus mainly on 
time extensions of Petri net workflow models.
%nets as modelling tool. On the one hand, there are a bunch of works 
%defining timed workflows nets in terms of time Petri nets, that is, 
%an interval is associated to the firing of a transition.
Ling and Schmidt~\cite{LS2000} defined timed workflow nets in terms 
of Time Elementary Nets (TENs). These nets are 1-bounded by definition 
and a net is sound iff it is live and the initial marking is a
home marking in a net that connects the output place of the workflow with
the input one. Du and Jiang~\cite{DuJ03} suggested Logical Time Workflow Nets 
(LTWN) and their compositional semantics. Here liveness together with 
boundedness is a necessary and sufficient condition for soundness.
Moreover, the soundness of a well-structured LTWN can be verified 
in polynomial time. Tiplea et al.~\cite{TipleaM05} 
introduced a variant of timed workflow nets in terms of timed Petri nets 
and showed the decidability of soundness for the bounded subclass.
In subsequent work~\cite{TipleaM06,TipleaM09} they studied the decidability 
of soundness under different firing strategies. The papers listed above
rely on the model of time Petri nets where timing information is
associated to transitions and not to tokens like in our case.
The two models are significantly different, in particular the 
number of timing parameters for time Petri nets is fixed, contrary
to the dynamic creation of tokens with their private clocks in
timed-arc Petri nets. We also see several modelling advantages
of having ages associated to tokens as we can for example track
the duration of sequentially composed tasks (via transport arcs) 
as demonstrated in our running example. We are not aware of other
works developing a workflow theory and the corresponding notions
of soundness based on timed-arc Petri nets. Finally, we
implement the soundness checks within a user-friendly tool
that permits easy GUI-based debugging of issues in workflows---something 
that is not that common for other workflow analysis tools 
(see~\cite{FF:AWPN:06} for more discussion).

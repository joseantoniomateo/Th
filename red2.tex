\chapter{Generalidades}

En este cap\'{\i}tulo presentamos una revisi\'{o}n de los principales
conceptos relacionados con las Redes de Petri.
Por tanto, el objetivo de este cap\'{\i}tulo ser\'{a} proporcionar
la base de trabajo de los siguientes, as\'{\i} como fijar la
notaci\'{o}n que va a ser utilizada a lo largo de la tesis.
Puesto que los
resultados presentados en este cap\'{\i}tulo son cl\'{a}sicos en la
teor\'{\i}a sobre Redes de Petri, algunas demostraciones son
omitidas, indicando \'{u}nicamente las citas bibliogr\'{a}ficas en las que
pueden encontrarse.

\section{Notaci\'{o}n}

La notaci\'{o}n empleada a lo largo de este trabajo ser\'{a} la siguiente:
\begin{enumerate}
\item {\bf N\'{u}meros}\\
Let $\nnul = \mathbb{N} \cup \{0\}$ and $\rnul$ be the set of the 
nonnegative integers and nonnegative real numbers, respectively,
and $\nnul^{\infty} = \nnul \cup \left\{\infty \right\}$. 
Denotaremos por $\nnul = \mathbb{N} \cup \{0\}$ los n\'{u}meros naturales
incluyendo el 0, por los naturales excluyendo el 0, y
denotaremos por $Z$ al conjunto de n\'{u}meros enteros.
Asimismo, denotaremos por $Q$ los n\'{u}meros racionales, y por
$Q^{+}$ los n\'{u}meros racionales positivos.
\item {\bf Conjuntos y Multiconjuntos}\\
Utilizaremos la notaci\'{o}n usual para trabajar con
conjuntos. El cardinal de un conjunto $A$
se denotar\'{a} de la forma $|A|$.

As\'{\i}, diremos
que $x \in R$, para un multiconjunto $R\,:\,A \longrightarrow \nnul$
sii $R(x) > 0$; asimismo utilizaremos el s\'{\i}mbolo
$\emptyset$ para denotar el
multiconjunto vac\'{\i}o ($R(x)=0,\,\forall x \in A$); mientras que,
por ejemplo,
un multiconjunto que consta de dos
instancias de un elemento $x$ y una instancia de un elemento $y$
ser\'{a} denotado en la forma $\{2x,1y\}$. Dado un conjunto
$A$, denotaremos por ${\cal B}(A)$ el conjunto de multiconjuntos
sobre $A$.

We start with the preliminaries and the definition of timed transition system. 
We use the delimiters \{\} for sets and $\multiset{}$ for multisets.
As usual, let $A$ be a set, a multiset or a list, 
we use $x \in A$ to denote the membership of the element  $x$ in $A$. 
%We consider only discrete time
%as for closed intervals the reachability/soundness problems
%are equivalent with the continuous time variant as discussed in 
%Section~\ref{sec:cont}.
Let $\nnul = \mathbb{N} \cup \{0\}$ and $\rnul$ be the set of the 
nonnegative integers and nonnegative real numbers, respectively,
and $\nnul^{\infty} = \nnul \cup \left\{\infty \right\}$. 
\item {\bf Relaciones}\\
Dado un conjunto $X$, una relaci\'{o}n sobre
$X$ ser\'{a} un conjunto $R \subseteq X \times X$. Llamaremos
dominio de $R$ al conjunto $dom(R)$ definido de la forma
siguiente:
\[dom(R) = \{ x \in X \,|\, \exists
y \in X\,:\, (x,y) \in R \}\]
y codominio de $R$ al conjunto $cod(R)$ definido como sigue:
\[cod(R) = \{ x \in X \,|\, \exists y \in X\,:\,(y,x) \in R\}\]
Dada una relaci\'{o}n $R$, se define el {\it cierre reflexivo y transitivo} de
$R$, $R^*$, como sigue:
\[ R^* = \{ (x,y)\,|\,x=y \,\vee\,
\exists x_1,\ldots,x_n,\,\,(x,x_1)\in R,\ldots,(x_n,y) \in R\}\]
Asimismo, se define el {\it cierre
transitivo} de $R$, $R^+$, como sigue:
\[R^+ = \{ (x,y)\,|\,
\exists x_1,\ldots,x_n,\,\,(x,x_1)\in R,\ldots,(x_n,y) \in R\}\]
\item {\bf Vectores}\\
La notaci\'{o}n empleada para representar los vectores
ser\'{a} la usual, mediante tuplas. En el caso de vectores con componentes
en $\nnul$, diremos que $v \geq w$ sii todas las componentes de $v$
son mayores o iguales que las correspondientes de $w$. Adem\'{a}s,
diremos que $v > w$ si $v \geq w$ y $v \neq w$.
\end{enumerate}

% % Labelled Transition system
\begin{definition}[Labelled transition system]

For a choreography $C$, we define the
semantics of $C$ as the labelled transition system
obtained by the application of rules in Table \ref{tab:coreo}, starting at the state ${s_0}_c$:
\[
  {\it lts}(C) = (\mathcal{Q}, {s_0}_c,  \rightarrow)
\]
%
\noindent
where $\mathcal{Q}$ is the set of reachable choreography states, ${s_0}_c$ is the initial state where all the variables in $\sigma$ are initialised to 0 and $\rho$ is empty, and
$\rightarrow \,\,= \,\,\rightarrow_1 \,\cup\, \{ \flecha{a}\,|\,$
for all basic activity $a$, or $a=\tau \,\}$.
\vspace{0.1cm}
\end{definition}

% % Discrete time TTS
A \emph{discrete timed transition system} (DTTS) 
is a triple $\left(\proc, \act,\rightarrow\right)$
where $\proc$ is the set of states, $\act$ is the set of actions
and $\rightarrow\: \subseteq \proc \times (\act \cup \nnul)  \times \proc$ is the 
transition relation written as $s \trans{a} s'$ whenever $(s,a,s') \in \rightarrow$.
If $a \in \act$ then we call it a \emph{switch transition}, if
$a \in \nnul$ we call it a \emph{delay transition}.
%By $\rightarrow^{*}$ we denote the reflexive and transitive closure of 
%the relation
%$\rightarrow  \eqdef \bigcup_{a \in \act} \trans{a} \; \cup \; \bigcup_{d \in \nnul} \trans{d}$. 
We also define the set of \emph{well-formed time intervals} as 
$\int \eqdef \{[a,b] \mid a \in \nnul,b\in \nnul^{\infty}, a\leq b \}$
and its subset $\intinv \eqdef \{[0,b] \mid b\in \nnul^{\infty} \}$
used in age invariants. 
We start with the preliminaries and the definition of timed transition system. 
We use the delimiters \{\} for sets and $\multiset{}$ for multisets.
As usual, let $A$ be a set, a multiset or a list, 
we use $x \in A$ to denote the membership of the element  $x$ in $A$. 
%We consider only discrete time
%as for closed intervals the reachability/soundness problems
%are equivalent with the continuous time variant as discussed in 
%Section~\ref{sec:cont}.

%Continuous time TTS
A \emph{timed transition system} (TTS) 
is a triple $\left(\proc, \act,\rightarrow\right)$
where $\proc$ is the set of states, $\act$ is the set of actions
and $\rightarrow\: \subseteq \proc \times (\act \cup \rnul)  \times \proc$ is the 
transition relation written as $s \trans{a} s'$ whenever $(s,a,s') \in \rightarrow$.
If $a \in \act$ then we call it a \emph{switch transition}, if
$a \in \rnul$ we call it a \emph{delay transition}.
%By $\rightarrow^{*}$ we denote the reflexive and transitive closure of 
%the relation
%$\rightarrow  \eqdef \bigcup_{a \in \act} \trans{a} \; \cup \; \bigcup_{d \in \nnul} \trans{d}$. 
We also define the set of \emph{well-formed time intervals} 
by: $$\int =\{ [a,a] \;|\; [a,b] \;|\; [a,b) \;|\; (a,b] \;|\; (a,b) \;|\; [a,\infty) \;|\; (a,\infty) \mid a,b \in \nnul,a < b\}$$
 Moreover, its subset: $$\intinv = \{ [0,0] \;|\; [0,b] \;|\; [0,b) \;|\; [0,\infty) \mid b \in \nnul\}$$
is used in age invariants. 

We say that a relation $\mathcal{R}$ is an untimed bisimulation for a timed transition system 
$T=\left(\proc, \act,\rightarrow\right)$ iff whenever $s_1 \mathcal{R} s_2$, $a \in \act$ and $d \in \rnul$:
\begin{enumerate}
\item if $s_1\trans{a}s^\prime_1$ then $s_2\trans{a}s^\prime_2$ such that $s^\prime_1 \mathcal{R} s^\prime_2$;
\item if $s_1\trans{d}s^\prime_1$ then $s_2\trans{d^\prime}s^\prime_2$ for some $d^\prime \in \rnul$ such that $s^\prime_1 \mathcal{R} s^\prime_2$;
\item if $s_2\trans{t}s^\prime_2$ then $s_1\trans{t}s^\prime_1$ such that $s^\prime_1 \mathcal{R} s^\prime_2$;
\item if $s_2\trans{d}s^\prime_2$ then $s_1\trans{d^\prime}s^\prime_1$ for some $d^\prime \in \rnul$ such that $s^\prime_1 \mathcal{R} s^\prime_2$.
\end{enumerate}

\section{Redes de Petri}

\begin{definition} (Redes de Petri Ordinarias)\\
Una terna $N=(P,T,F)$, formada por dos
conjuntos $P$ y $T$, y una relaci\'{o}n $F$ definida sobre
$P \,\cup\,T$, se dice que es una {\em Red de Petri Ordinaria} si
satisface:
\begin{enumerate}
\item $P \,\cap \,T = \emptyset$
\item $F \subseteq (P \times T) \,\cup\, (T \times P)$
\item $dom(F) \, \cup \, cod(F) = P \, \cup \, T$
\end{enumerate}

Al conjunto $P$ se le llama {\it conjunto de lugares}, al
conjunto $T$ {\it conjunto de transiciones} y a $F$
{\it relaci\'{o}n de flujo}. $F$ relaciona lugares
con transiciones en forma de arcos entre lugares y transiciones
o entre transiciones y lugares.
Salvo que se indique lo contrario, supondremos que las
redes con las que trabajaremos son finitas (los conjuntos
$P$ y $T$ son finitos).
Las redes de Petri pueden
representarse gr\'{a}ficamente por medio de grafos bipartitos, que son
grafos que constan de dos tipos de
nodos (lugares y transiciones). Los lugares se representan con
c\'{\i}rculos y las transiciones con rect\'{a}ngulos.

Sea el conjunto $X = P\,\cup\,T$. Entonces, para todo $x \in X$
se definen los conjuntos siguientes:
$\precond{x} = \{ y \in X \,|\, (y,x) \in F\}~$ (preconjunto de
x), $x^{\bullet} = \{ y \in X \,|\, (x,y) \in F\}~$ (postconjunto
de x).

Se dice que $N$ es $T$-restringida sii $\precond{t} = t^{\bullet} =
\emptyset\,\,\,\forall t \in T$.
\end{definition}

\begin{example} Sea $N=(P,T,F)$, siendo:
\[\begin{array}{l}
P = \{ p_1,\,p_2,\,p_3\}\\
T = \{ t_1,\,t_2\}\\
F = \{ (p_1,t_1),\,(p_2,t_1),\,(t_1,p_3),\,(p_3,t_2)\}
\end{array}\]

Esta red est\'{a} representada gr\'{a}ficamente en la figura \ref{fig201}.
\end{example}

\begin{figure}
\setlength{\unitlength}{0.0125in}
\begin{picture}(130,170)(46,638)
\thicklines
\put(267,722){\circle{16}}
\put(300,802){\circle{16}}
\put(241,803){\circle{16}}
\put(267,712){\vector( 0,-1){ 37}}
\put(267,755){\vector( 0,-1){ 25}}
\put(290,800){\line(-2,-3){ 10}}
\put(280,785){\vector(-1,-4){  5}}
\put(250,800){\line( 2,-3){ 10}}
\put(260,785){\vector( 0,-1){ 20}}
\put(255,666){\framebox(25,7){}}
\put(255,756){\framebox(25,7){}}
\put(279,730){\makebox(0,0)[lb]{\raisebox{0pt}[0pt][0pt]{ $p_3$}}}
\put(283,676){\makebox(0,0)[lb]{\raisebox{0pt}[0pt][0pt]{ $t_2$}}}
\put(284,764){\makebox(0,0)[lb]{\raisebox{0pt}[0pt][0pt]{ $t_1$}}}
\put(309,810){\makebox(0,0)[lb]{\raisebox{0pt}[0pt][0pt]{ $p_2$}}}
\put(220,812){\makebox(0,0)[lb]{\raisebox{0pt}[0pt][0pt]{ $p_1$}}}
\end{picture}




\caption{\label{fig201} Ejemplo de Red de Petri}
\end{figure}

\begin{definition} (Marcajes de Redes de Petri Ordinarias)\\
Sea $N=(P,T,F)$ una Red de Petri Ordinaria.
Una funci\'{o}n $M: P \longrightarrow
\nnul$ se llama un {\it marcaje de N}. Entonces,
$(P,T,F,M)$ se llama una {\it Red de Petri Ordinaria Marcada}.
\end{definition}

Los marcajes de las redes de Petri se representan gr\'{a}ficamente
incluyendo en cada lugar tantos puntos como tokens le corresponden,
o bien anotando cada lugar con el n\'{u}mero de tokens asociado al mismo.

\begin{example} En la Red de Petri del ejemplo \ref{fig201} podemos
considerar el siguiente marcaje:
\[ M(p_1) = 1,\,\,\,M(p_2)=1,\,\,\,M(p_3)=0 \]
La representaci\'{o}n gr\'{a}fica de la misma, correspondiente a la
primera variante, puede verse en la figura \ref{fig202}.
\end{example}

\begin{figure}
\setlength{\unitlength}{0.0125in}
\begin{picture}(130,170)(46,638)
\thicklines
\put(267,722){\circle{16}}
\put(300,802){\circle{16}}
\put(300,802){\circle*{2}}
\put(241,803){\circle{16}}
\put(241,803){\circle*{2}}
\put(267,712){\vector( 0,-1){ 37}}
\put(267,755){\vector( 0,-1){ 25}}
\put(290,800){\line(-2,-3){ 10}}
\put(280,785){\vector(-1,-4){  5}}
\put(250,800){\line( 2,-3){ 10}}
\put(260,785){\vector( 0,-1){ 20}}
\put(255,666){\framebox(25,7){}}
\put(255,756){\framebox(25,7){}}
\put(279,730){\makebox(0,0)[lb]{\raisebox{0pt}[0pt][0pt]{$p_3$}}}
\put(283,676){\makebox(0,0)[lb]{\raisebox{0pt}[0pt][0pt]{$t_2$}}}
\put(284,764){\makebox(0,0)[lb]{\raisebox{0pt}[0pt][0pt]{$t_1$}}}
\put(309,810){\makebox(0,0)[lb]{\raisebox{0pt}[0pt][0pt]{$p_2$}}}
\put(220,812){\makebox(0,0)[lb]{\raisebox{0pt}[0pt][0pt]{$p_1$}}}
\end{picture}




\caption{\label{fig202} Ejemplo de Red de Petri Marcada}
\end{figure}

La sem\'{a}ntica de una Red de Petri est\'{a} definida mediante la siguiente
regla de disparo, que define el marcaje alcanzado tras el disparo de una
transici\'{o}n.

\begin{definition} (Regla de Disparo)\\
Sea $N=(P,T,F,M)$ una Red de Petri Ordinaria Marcada. Una
transici\'{o}n $t \in T$
se dice que {\it est\'{a} permitida bajo el marcaje M},
lo cual se denota por $M[t\rangle$, si para todo
lugar $p \in P$ tal que $(p,t) \in F$ se verifica $M(p) > 0$.

El disparo de una transici\'{o}n permitida $t$ bajo el marcaje $M$
resulta en un nuevo marcaje de la red, $M'$, definido de la forma
siguiente:
\[M'(p) = M(p) - W_f(p,t) + W_f(t,p)~~~\forall p \in P\]
siendo $W_f(x) = 1$ si $x \in F$ y $W_f(x) = 0$, para $x \not\in F$,
para todo $x \in (T \times P) \, \cup \, (P \times T)$.
Ello se denota en la forma $M[t\rangle M'$.
\end{definition}

\begin{example} En el ejemplo \ref{fig202}, el disparo de la transici\'{o}n $t_1$
conduce la red al marcaje $M'$:
\[ M'(p_1) = 0,\,\,\,M'(p_2)=0,\,\,\,M'(p_3)=1 \]
\end{example}

\begin{definition} (Activaci\'{o}n Concurrente de Transiciones)\\
Sea $N= (P,T,F,M)$ una Red de Petri Ordinaria Marcada. Sea
$R \subseteq T$ un conjunto de transiciones de la red.
Se dice que {\it las transiciones de R est\'{a}n permitidas
concurrentemente}, lo que se denota por $M [ R \rangle$
sii $M(p) \geq \sum_{t \in R} W_f(p,t),~
\forall p\in P$, donde
$W_f(p,t)$ est\'{a} definido de la misma forma que en la definici\'{o}n anterior.

Podemos incluso extender esta definici\'{o}n a multiconjuntos,
permitiendo as\'{\i} la posibilidad de que varias instancias de una
misma transici\'{o}n sean disparadas en un solo paso.
De esta forma, diremos que un multiconjunto de transiciones
$R$ est\'{a} permitido bajo el marcaje $M$ sii $M(p)
\geq \sum_{t \in T} W_f(p,t) \cdot R(t)$, $\forall p \in P$.

El disparo de un multiconjunto de transiciones $R$ bajo el
marcaje $M$ conduce la red a un nuevo marcaje $M'$, dado por:
\[ M'(p) = M(p) - \sumas{t \in T} (W_f(p,t) - W_f(t,p)) \cdot R(t)\]
Esta evoluci\'{o}n de la red en un solo paso la denotaremos por
$M[ R \rangle M'$.
\end{definition}

\begin{definition} (Redes de Petri Generalizadas)\\
Una Red de Petri Generalizada (aunque en el futuro nos permitiremos
eventualmente la licencia de llamarlas simplemente Redes de Petri)
es una tupla $N=(P,T,F,K,W)$, donde:
\begin{enumerate}
\item $(P,T,F)$ es una Red de Petri Ordinaria.
\item $K\,:\,P \longrightarrow \nat \,\cup\, \{\infty\}$ es una
funci\'{o}n que indica el n\'{u}mero m\'{a}ximo de tokens que puede haber en
cada lugar ({\it funci\'{o}n de capacidad}).
\item $W\,:\,F \longrightarrow \nat^{+}$ es una funci\'{o}n que
indica la multiplicidad de cada arco ({\it pesos de los arcos}).
\end{enumerate}

Se suele omitir la funci\'{o}n de capacidad, por innecesaria,
si \'{e}sta es infinita para todos los lugares. Adem\'{a}s, es
usual extender la definici\'{o}n de $W$ a todo el universo de
posibles arcos, haci\'{e}ndola nula para pares
$(p,t)$ o $(t,p)$ que no est\'{e}n en $F$.
\end{definition}

\begin{definition} (Regla de Disparo para Redes de Petri Generalizadas)\\
Sea $N=(P,T,F,K,W)$ una Red de Petri Generalizada.
\begin{enumerate}
\item Una funci\'{o}n $M\,:\, P \longrightarrow \nat$ se dice que
es un {\it marcaje} de $N$ sii $M(p) \leq K(p)$, para todo
$p \in P$.
\item Una transici\'{o}n $t \in T$ est\'{a} permitida bajo un marcaje
$M$, lo que se denota por $M[ t \rangle$
sii $W(p,t) \leq M(p) \leq K(p) - W(t,p)$, para todo $p \in P$.
El disparo de $t$ resulta en un nuevo marcaje de la red $M'$, dado por:
$M'(p) = M(p) - W(p,t) + W(t,p)$, para todo $p \in P$.
De nuevo, esta evoluci\'{o}n la denotamos por $M[ t \rangle M'$.
\item Un multiconjunto de transiciones $R$ est\'{a} permitido bajo el
marcaje $M$, lo que se denota por $M [ R \rangle$, si y s\'{o}lo si
$M(p) \geq \sum_{t \in T} W(p,t) \cdot R(t)$. El disparo de $R$
resulta en un nuevo marcaje $M'$, dado por:
\[ M'(p) = M(p) - \sum_{t \in T} (W(p,t) - W(t,p)) \cdot R(t), \,\,
\forall p \in P\]
Lo que se denota por $M [ R \rangle M'$.
\end{enumerate}
\end{definition}

\begin{definition} Sea $N=(P,T,F,K,W,M_0)$ una Red de Petri Marcada.
\begin{enumerate}
\item Se dice que $\sigma = M_0 t_1 M_1 \ldots t_n M_n$ es una
secuencia de ocurrencia finita de $N$ si y s\'{o}lo si
$\forall i \in \{1,\ldots,n\},\,M_{i-1} [ t_i \rangle M_i$.
En algunas ocasiones denotaremos las secuencias de ocurrencia
en la forma $t_1 \ldots t_n$, ya que a partir del marcaje de partida
$M_0$ y a partir de la secuencia de transiciones disparadas a partir de \'{e}l,
es posible obtener los dem\'{a}s marcajes $M_i$.

Extendemos la notaci\'{o}n habitual a
secuencias de ocurrencia, obteniendo $M_0 [ \sigma \rangle M_n$.
Se denota mediante $L(N,M_0)$ al
conjunto de secuencias de ocu\-rren\-cia de $N$ que parten de $M_0$.
\item Se dice que $\sigma = M_0 R_1 M_1 \ldots R_n M_n$ es una
secuencia de pasos finita de $N$ si y s\'{o}lo si
$\forall i \in \{1,\ldots,n\},\,M_{i-1} [ R_i \rangle M_i$.
De nuevo se extiende la notaci\'{o}n habitual a secuencias de pasos:
$M_0 [ \sigma \rangle M_n$.
Se denota por $P(N,M_0)$ al conjunto de secuencias de pasos de $N$
que parten de $M_0$.
\end{enumerate}
\end{definition}

En lo sucesivo trabajaremos usualmente sobre la sem\'{a}ntica de
secuencias de ocurrencia, salvo que expl\'{\i}citamente se
indique lo contrario.

%\bdfn (Matrices de Incidencia)\\
%Sea $N=(P,T,F,W,M_0)$ una Red de Petri Marcada.
%\begin{enumerate}
%\item Se dice que $N$ es {\it pura} sii $\forall t \in T$,
%$\forall p \in P$, $W(t,p) \cdot W(p,t) = 0$.
%\item Si $N$ es una red pura, podemos definir su
%{\it matriz de incidencia previa}, $C^{-} = (c_{i,j}^{-})$,
%$i=1,\ldots,|P|\,;\,j=1,\ldots,|T|$, siendo
%$c_{i,j}^{-} = W(p_i,t_j)$, y su
%{\it matriz de incidencia posterior}
%$C^{+} = (c_{i,j}^{+})$,
%$i=1,\ldots,|P|\,;\,j=1,\ldots,|T|$, siendo
%$c_{i,j}^{+} = W(t_i,p_j)$.
%\item Si $N$ es una red pura se define su {\it matriz de
%incidencia} $C$ por medio de $C = C^+ - C^-$.
%\end{enumerate}
%\edfn

%La matriz de incidencia puede ser definida tambi\'{e}n sobre redes
%que no sean puras, pero en tal caso no caracteriza a las mismas, pues
%una misma matriz de incidencia corresponde a varias redes diferentes.
%
%\bex Es sencillo obtener dos Redes de Petri diferentes con la misma
%matriz de incidencia. Para ello basta tomar
%una Red de Petri Ordinaria y elegir un lugar y una transici\'{o}n
%no conectados inicialmente, y conectarlos formando un loop. Por
%ejemplo, las dos redes de la figura \ref{fig203} tienen la misma
%matriz de incidencia. De hecho, si se trabaja con Redes Generalizadas
%el a\~{n}adido se puede hacer sobre cualquier par.
%\eex
%
%\begin{figure}
%\input{fig28}
%\caption{\label{fig203} Dos Redes de Petri con la misma Matriz de Incidencia}
%\end{figure}

%\bprop (Ecuaci\'{o}n de Estado)\\
%Sea $N=(P,T,F,W,M_0)$ una Red de Petri Marcada Pura,
%$\sigma \in L(N,M_0)$, y $M_0 [ \sigma \rangle M$.
%Entonces se tiene: $M = M_0 + C \cdot \bar{\sigma}$, siendo
%$\bar{\sigma}$ el vector de Parikh asociado a la
%secuencia $\sigma$, que est\'{a} definido como
%$\bar{\sigma}(i) = $n\'{u}mero de ocurrencias de
%la transici\'{o}n $t_i$ en la secuencia $\sigma$.
%
%\proof Sea la secuencia de marcajes producida a lo
%largo de la ejecuci\'{o}n de $\sigma$: $M_0 t_{i_{1}} M_1 \ldots t_{i_{n}} M_n$.
%Entonces, de la regla de disparo se concluye que $M_1 = M_0 +
%C \cdot U_{i_{1}}$, siendo $U_{i_{1}}$ el vector cuyas componentes son todas
%nulas, salvo la $i_{1}$-\'{e}sima, que vale 1. En general, se obtiene
%que $M_k = M_{k-1} + C \cdot U_{i_{k}}$.
%
%Por tanto:
%\[ M_k = M_{k-2} + C \cdot (U_{i_{k-1}} + U_{i_{k}}) = \ldots =
%M_0 + C \cdot \sum_{j=1}^{k} U_{i_{j}}\]
%Ahora bien, $\suma{j=1}{k} U_{i_{j}} = \bar{\sigma}$, lo que
%termina la demostraci\'{o}n.
%\eprop

\section{An\'{a}lisis de Redes}

Cuando se plantea el dise\~{n}o de un sistema, aparte del inter\'{e}s que
suscita el tener un modelo gr\'{a}fico del mismo, resulta importante
disponer de herramientas que nos permitan obtener propiedades
a partir del modelo considerado. En el caso de los sistemas
concurrentes, la verificaci\'{o}n del cumplimiento de ciertas
propiedades se hace m\'{a}s dif\'{\i}cil que en el caso de los sistemas
secuenciales, por lo que dichas herramientas cobran un particular
inter\'{e}s. El an\'{a}lisis de la conducta de los sistemas tiene por objeto
determinar el cumplimiento de ciertas propiedades, como por ejemplo,
que el n\'{u}mero de procesos en una cierta cola del sistema no excede
cierta cantidad, que se garantiza la exclusi\'{o}n mutua en el acceso
a cierto recurso del sistema, etc.

En el caso de las Redes de Petri, junto con su notable inter\'{e}s para
el modelado de sistemas, por su naturaleza gr\'{a}fica que permite
tener un modelo f\'{a}cilmente comprensible de un sistema,
se tiene tambi\'{e}n una herramienta poderosa para analizar formalmente
el cumplimiento de determinadas propiedades de buena conducta
de los mismos,
como la ausencia de bloqueos, la alcanzabilidad de un cierto
estado, la posibilidad de alcanzar la situaci\'{o}n de partida
del sistema con independencia de su estado actual, etc.

\medskip
Habitualmente se dividen las propiedades de los sistemas en dos
clases:

\subsection{Propiedades de Seguridad}

Garantizan que el sistema no
alcanzar\'{a} nunca un conjunto de estados no deseados, o
que nunca ejecutar\'{a} una secuencia de pasos que pertenezca a un
determinado conjunto.

Son propiedades de Seguridad las
siguientes:
\begin{enumerate}
\item {\bf Propiedad de Alcance}. Un marcaje $M$ de una Red de Petri
Marcada $N= (P,T,F,W,M_0)$ se dice que es {\it alcanzable} en $N$
sii existe una secuencia de ocurrencia $\sigma \in L(N,M_0)$
tal que $M_0 [ \sigma \rangle M$. Denotamos por $[M_0\rangle$ al
conjunto de marcajes alcanzables en $N$ a partir del marcaje
inicial $M_0$, y por $[ M \rangle$ al conjunto de marcajes alcanzables
en $N$ a partir de un marcaje $M$.
\item {\bf Propiedad de Seguridad}. Una Red de Petri Marcada
$N=(P,T,F,W,$ \linebreak
$M_0)$ es {\it n-segura}, para un cierto $n \in \nat$ dado, si todo
marcaje alcanzable $M$ a partir de $M_0$ cumple la
propiedad $M(p) \leq n$, para todo $p \in P$. Las redes
1-seguras se llaman habitualmente {\it seguras}, y en ellas los
marcajes pueden identificarse con subconjuntos de lugares.

Bajo los mismos requisitos, se dice que un lugar $p \in P$ es
{\it n-seguro} si $M(p) \leq n$, para todo marcaje alcanzable
$M$ a partir de $M_0$.
\item {\bf Propiedad de Limitaci\'{o}n}. Sea
$N=(P,T,F,W,M_0)$ una Red de Petri Marcada.
Se dice que un lugar $p \in P$ es limitado si
existe un n\'{u}mero natural $n \in \nat$ tal que dicho lugar es
$n$-seguro; y se dice que $N$ es {\it limitada} si todos sus lugares
son limitados.
\item {\bf Ausencia de Bloqueos}. Dada una Red de Petri Marcada
$N=(P,T,F,W,$ \linebreak
$M_0)$, y un marcaje alcanzable $M$ a partir de $M_0$.
Se dice que $M$ es un {\it marcaje muerto} si no existe ninguna
transici\'{o}n permitida bajo dicho marcaje. La red $N$ se dice
que est\'{a} {\it libre de bloqueos} si no existe ning\'{u}n marcaje
alcanzable muerto.
\item {\bf Conservaci\'{o}n Respecto a un Vector de Pesos}.
Sea una Red de Petri Marcada $N=(P,T,F,W,M_0)$, con
$P = \{p_1,\ldots,p_n\}$. Se dice que $N$ es
{\it conservativa respecto a un vector de pesos w}, con
$w \in \nat^n$, si para todo marcaje alcanzable $M$
a partir de $M_0$ se cumple:
\[ \suma{i=1}{n} w_i \cdot M(p_i) =
\suma{i=1}{n} w_i \cdot M_0(p_i)\]
\item {\bf Cubrimiento de Marcajes}.
Sea una Red de Petri Mar\-cada $N=$ \linebreak $(P,T,F,W,
M_0)$. Dado un marcaje $M$ de $N$, se dice que $N$ cumple la propiedad
del cubrimiento de marcajes para el marcaje $M$ si existe $M' \in
[M_0 \rangle$ tal que $M' \geq M$.
\end{enumerate}

\noindent {\sc NOTA:} Para ser m\'{a}s precisos, la alcanzabilidad no es
exactamente una propiedad de seguridad, sino la negaci\'{o}n de la propiedad,
la no alcanzabilidad. An\'{a}logamente ocurre con el cubrimiento de marcajes.

\subsection{Propiedades de Actividad}

Garantizan que el sistema, con independencia de su estado actual,
podr\'{a} eventualmente alcanzar un determinado estado de un cierto conjunto
de estados, o podr\'{a} eventualmente ejecutar una cierta secuencia de eventos.
Son propiedades de actividad las siguientes:
\begin{enumerate}
\item {\bf Vivacidad}.
Sea $N=(P,T,F,W,M_0)$ una Red de Petri Marcada. Una transici\'{o}n $t \in T$
se dice que es {\it viva} si para todo marcaje alcanzable $M \in
[ M_0 \rangle$ existe una secuencia de ocurrencia $\sigma$ que parte
de $M$ tal que $\sigma = t_1 \ldots t_m$, con $t_m = t$. Se dice que
$N$ es {\it viva} si todas sus transiciones son vivas.
\item {\bf Home State}.
Sea $N=(P,T,F,W,M_0)$ una Red de Petri Marcada. Se dice que
un marcaje $M$ de $N$ es un {\it home state} si para todo
$M' \in [ M_0 \rangle$ se tiene $M \in [ M' \rangle$.
\item {\bf Home Space}. Sea $N=(P,T,F,W,M_0)$ una Red de Petri Marcada.
Se dice que un conjunto de marcajes ${\cal M}$ es un {\it home-space}
de $N$ si para todo marcaje $M' \in [ M_0 \rangle$ existe un marcaje
$M'' \in {\cal M}$ tal que $M'' \in [ M' \rangle$.
\item {\bf Ciclicidad}.
Sea $N=(P,T,F,W,M_0)$ una Red de Petri Marcada. Se dice que
$N$ tiene un {\it comportamiento c\'{\i}clico} si para todo marcaje
$M \in [ M_0 \rangle$ existe una secuencia de ocurrencia $\sigma$
que parte de $M$ tal que $M [ \sigma \rangle M_0$.
\end{enumerate}

\subsection{Decidibilidad del Alcance y la Vivacidad}

Al abordarse el estudio de la decidibilidad de algunas de
estas propiedades anteriormente citadas, se comprob\'{o} que en absoluto se
trataba de problemas sencillos, no encontr\'{a}ndose en principio en la
mayor parte de las ocasiones m\'{a}s que un me\-ca\-nis\-mo
de reducibilidad que nos permitir\'{\i}a resolver unos problemas
reduci\'{e}ndolos a otros en apariencia m\'{a}s sencillos.
De esta forma, la decidibilidad
de la propiedad de alcance apareci\'{o} como b\'{a}sica para resolver la
decidibilidad de las dem\'{a}s.
Sin embargo, pronto se constat\'{o} que la prueba de la decidibilidad del
alcance no era ni mucho menos sencilla. Por ello, se
plantearon las siguientes propiedades relacionadas con la
alcanzabilidad, a las que se consigui\'{o} reducir esta \'{u}ltima.
\begin{enumerate}
\item {\bf Propiedad de Alcance de Submarcajes}.
Sea $N=(P,T,F,W,M_0)$ una Red de Petri Marcada. Dado $P' \subseteq P$,
se dice que $N$ cumple la propiedad de Alcance de Submarcajes para $P'$
si existe un marcaje $M' \in [M_0 \rangle$
tal que $M'(p) = M(p)$, $\forall p \in P'$.
\item {\bf Propiedad de Alcance Nulo}.
Sea $N=(P,T,F,W,M_0)$ una Red de Petri Marcada.
$N$ cumple la propiedad de Alcance Nulo sii $0 \in [ M_0 \rangle$, donde
0 es el vector cuyas componentes son todas nulas.
\item {\bf Propiedad de Alcance Nulo de un Solo Lugar}.
Sea $N=(P,T,$ \linebreak $F,W,M_0)$ una Red de Petri Marcada.
Dado $p_0 \in P$, se dice que $N$ cumple la propiedad de Alcance Nulo
de un Solo Lugar para $p_0$ si existe un marcaje
$M \in [ M_0 \rangle$ tal que $M(p_0) = 0$.
\end{enumerate}

\begin{theorem} En lo que a su decidibilidad se refiere, son
equivalentes los siguientes problemas.
\begin{enumerate}
\item El problema de alcance.
\item El problema de alcance de submarcajes.
\item El problema de alcance nulo.
\item El problema de alcance nulo de un solo lugar.
\end{enumerate}

\proof Ver \cite{Pet81}.
\end{theorem}

Sin embargo, a pesar de este resultado, posteriormente se vio que la
dificultad de resolver cualquiera de los tres nuevos
problemas introducidos es la misma que la de resolver el
problema de alcance. En todo caso, aunque no se conoc\'{\i}a si esta
propiedad era decidible, s\'{\i} se obtuvieron
resultados que ya dejaban ver que en caso de que fuera decidible,
la complejidad del algoritmo empleado ser\'{\i}a prohibitiva
(por ejemplo, Lipton \cite{Lip76}
demostr\'{o} en 1976 que un algoritmo para resolver el problema de
alcance requerir\'{\i}a al menos una cantidad exponencial de espacio
de almacenamiento y de tiempo con respecto al n\'{u}mero de arcos de la red).
Finalmente, la decidibilidad del pro\-ble\-ma de
alcance fue resuelta en 1981 por
Mayr (ver \cite{May81}), y m\'{a}s tarde por Kosaraju (ver
\cite{Kos82}) que corrigi\'{o} algunos errores de la
demostraci\'{o}n original. Pueden encontrarse demostraciones m\'{a}s
legibles en \cite{Lam86,Reu90,Lam92}.

Una vez probada la decidibilidad del problema de
alcance, se obtuvo inme\-dia\-ta\-men\-te la decidibilidad del
problema de vivacidad, pues previamente \cite{Pet81} tambi\'{e}n se
hab\'{\i}a probado que dicho problema era equivalente al problema de alcance.
Posteriormente pudo hacerse lo propio con otras propiedades. As\'{\i}
\cite{Fru86} prob\'{o} la decidibilidad de la propiedad de ser {\it
home-state}, posteriormente extendida a {\it home-spaces} en
\cite{JoFr91}.

\subsection{T\'{e}cnicas de An\'{a}lisis de Propiedades}

Las t\'{e}cnicas de an\'{a}lisis de propiedades de Redes de Petri suelen
clasificarse en 4 grupos:
\begin{itemize}
\item An\'{a}lisis por enumeraci\'{o}n.
\item An\'{a}lisis por transformaci\'{o}n.
\item An\'{a}lisis estructural.
\item An\'{a}lisis por Simulaci\'{o}n.
\end{itemize}

%\subsubsection{An\'{a}lisis por enumeraci\'{o}n}
%
%En el an\'{a}lisis por enumeraci\'{o}n se
%construye un \'{a}rbol de marcajes, que
%expresa las posibles evoluciones de la red.
%El \'{a}rbol as\'{\i} obtenido se llama
%{\it \'{a}rbol de accesibilidad}, el cual es en general infinito.
%Por ello, a\'{u}n a costa de perder cierta informaci\'{o}n, se define
%el llamado {\it \'{a}rbol de cobertura}, que es siempre finito
%(para redes finitas), lo que permite estudiar sobre
%\'{e}l aquellas propiedades
%para las que dicho \'{a}rbol proporcione la informaci\'{o}n necesaria
%para establecer su decidibilidad.
%
%\bdfn (Arbol de Cobertura)\\
%Sea $N=(P,T,F,W,M_0)$ una Red de Petri Marcada. Se define el
%{\it \'{a}rbol de cobertura} de $N$, $T(N)$, de la forma siguiente:
%\begin{enumerate}
%\item La ra\'{\i}z se etiqueta con el marcaje inicial $M_0$.
%\item Un nodo $x$ etiquetado con un marcaje $M$ es una hoja del
%\'{a}rbol si ocurre una de las siguientes posibilidades:
%\begin{enumerate}
%\item O bien no existe ninguna transici\'{o}n permitida bajo $M$.
%\item O bien hay un nodo $y$, $y \neq x$,  en el camino que
%va desde la ra\'{\i}z a $x$, etiquetado tambi\'{e}n con $M$.
%\end{enumerate}
%\item Si $x$ es un nodo con etiqueta $M$, y no es una hoja, entonces
%para cada transici\'{o}n $t$ permitida por $M$ se introduce un nuevo arco
%en el \'{a}rbol, etiquetado con esa transici\'{o}n $t$,
%con destino en un nuevo nodo $y$, cuya etiqueta se obtiene en
%base a los pasos siguientes:
%\begin{enumerate}
%\item Para todo lugar $p \in P$ se calcula: $\tilde{M}(p) =
%M(p) + W(p,t) - W(t,p)$.
%\item Si en el camino de la ra\'{\i}z a $x$ hay un nodo $z$, $z \neq y$,
%con etiqueta $M"$ tal que: $\tilde{M}(p) \geq M"(p)$,
%para todo $p \in P$, entonces se etiqueta el nodo $y$ con $M'$, definido
%de la forma siguiente:
%\[M'(p) = \left\{
%\begin{array}{ll}
%\tilde{M}(p) & {\rm si~}\tilde{M}(p) = M"(p)\\
%\omega & {\rm en~caso~contrario}
%\end{array}
%\right.
%\]
%\item Si la condici\'{o}n anterior resulta falsa, se hace $M'= \tilde{M}$,
%y el nodo $y$ se etiqueta con $M'$.
%\end{enumerate}
%\end{enumerate}

%En esta definici\'{o}n se considera que $n < \omega$, $\forall n \in \nat$,
%y que $n+ \omega = \omega + n = \omega + \omega = \omega - n = \omega$,
%$\forall n \in \nat$.
%\edfn
%
%\bthm El \'{a}rbol de cobertura de toda Red de Petri Marcada $N$, $T(N)$,
%es finito.
%
%\proof Ver \cite{Pet81}.
%\ethm
%
%No es posible decidir mediante el \'{a}rbol de cobertura la
%propiedad de alcance, pues no puede establecerse si un
%marcaje concreto pertenece o no al alcance de la red.
%Sin embargo, s\'{\i} es posible decidir el cubrimiento de
%marcajes, as\'{\i} como la seguridad, limitaci\'{o}n
%y conservaci\'{o}n (ver \cite{Bes86,BeFe86,Pet81}).
%
%\medskip
%El siguiente teorema, extraido de \cite{Fru86}, establece que
%para toda red $N$ es posible
%encontrar un conjunto finito de marcajes accesibles con la
%propiedad de que cualquier marcaje accesible de $N$ cubre a alguno de los
%marcajes de este conjunto.
%
%\bthm{\label{tfru}} Sea $N=(P,T,F,W,M_0)$ una Red de Petri Marcada.
%Existe un conjunto finito
%de marcajes alcanzables en $N$, $\{ M^1_B,\ldots,M^b_B\}$, tales que
%$\forall M \in [ M_0 \rangle$ existe $i \in \{1,\ldots,b\}$
%tal que $M^i_B \subseteq M$.
%
%\proof Ver \cite{Fru86}.
%\ethm
%
%\subsubsection{An\'{a}lisis por transformaci\'{o}n}
%
%En el an\'{a}lisis por transformaci\'{o}n se aplican diversos mecanismos de
%reducci\'{o}n de redes que se aplican a una red para
%obtener redes m\'{a}s sencillas, que
%preservan algunas propiedades de la red de partida, lo que
%permite estudiar sobre estas \'{u}ltimas dichas propiedades.
%Las simplificaciones m\'{a}s usuales consisten en eliminar los
%lugares impl\'{\i}citos (ver \cite{Sif78,Esp90}), lo que conserva la
%vivacidad y limitaci\'{o}n de la red de partida, o bien el
%reemplazamiento de una subred por un \'{u}nico lugar
%(ver \cite{Bra86}), cuando \'{e}sta cumple ciertas condiciones.
%De nuevo, esta reducci\'{o}n preserva la limitaci\'{o}n y vivacidad.
%Otras t\'{e}cnicas de reducci\'{o}n (ver \cite{Ber80,Bra86})
%se basan en sustituir un lugar
%por un conjunto de transiciones, o en la eliminaci\'{o}n de
%algunas transiciones irrelevantes en el funcionamiento de la
%red, etc.
%
%\subsubsection{An\'{a}lisis Estructural Basado en el Algebra Lineal}
%
%Los m\'{e}todos de an\'{a}lisis estructural estudian las propiedades de las
%redes de Petri haciendo uso exclusivamente de su topolog\'{\i}a.
%Estos m\'{e}todos se dividen en dos grupos, con relaci\'{o}n a las
%herramientas que utilizan: M\'{e}todos basados en el \'{a}lgebra lineal y
%m\'{e}todos de cerrojos y trampas.
%
%Los m\'{e}todos de an\'{a}lisis estructural basados en el \'{a}lgebra lineal
%se apoyan en la ecuaci\'{o}n de estado de la red (trabajan por
%tanto sobre redes puras). Para ello definen los P-invariantes
%como P-vectores anuladores izquierdos de la matriz de incidencia.
%
%\bdfn (P-invariantes)\\
%Sea $N=(P,T,F,W,M_0)$ una Red de Petri Marcada Pura.
%Entonces, a los vectores
%$V\,:\,P \longrightarrow {\cal Z}$ se les llama {\it P-vectores}.
%
%\begin{enumerate}
%\item Un {\it P-vector} $I$ se llama un {\it P-invariante} de $N$ si
%$I^T \cdot C = 0$, siendo $0$ el vector cuyas componentes son todas
%nulas.
%\item El {\it soporte de un P-vector} $H \in {\cal Z}^P$ se define como
%sigue:
%\[P_H = \{ p \in P \,|\, H(p) \neq 0\}\]
%\item Un {\it P-invariante} $I$ de $N$ se dice que es {\it no-negativo}
%si todas sus componentes son mayores o iguales que 0.
%\item Un {\it P-invariante} no-negativo
%se dice que es {\it mi\-ni\-mal} si no
%hay otro {\it P-inva\-rian\-te} $I'$ no-negativo tal que $0 < I' <I$.
%\pagebreak
%\item Un {\it P-invariante positivo} (todas sus componentes son positivas
%o nulas, pero al menos una es estrictamente positiva) se llama un
%{\it P-semiflujo}.
%\end{enumerate}
%\edfn
%
%La ecuaci\'{o}n $I^T \cdot C = 0$ puede ser resuelta mediante
%el algoritmo de Farkas (ver \cite{MeVa86}), que proporciona un
%conjunto de generadores del conjunto de P-semiflujos
%soluciones. La siguiente proposici\'{o}n demuestra que los
%{\it P-semiflujos minimales} generan todas las soluciones
%de $Y^T \cdot C = 0$, con $Y$ un vector con coeficientes
%en $Q^+ \,\cup \,\{0\}$.
%
%\bprop Sea una Red de Petri Marcada Pura $N=(P,T,F,W,M_0)$ y
%$C$ su matriz de incidencia. Sea $Y$ un vector con
%coeficientes en $Q^+ \,\cup \,\{0\}$ tal que $Y^T \cdot C=0$.
%Entonces, $Y = \sumas{i} \lambda_i Y_i$, siendo los $Y_i$
%{\it P-semiflujos minimales} de $N$.
%
%\proof Ver \cite{Esp90}.
%\eprop
%
%La siguiente propiedad de los P-invariantes
%permite resolver el problema de la conservaci\'{o}n.
%
%\bprop Sea una Red de Petri Marcada Pura $N = (P,T,F,W,M_0)$ y
%sea $I$ un P-invariante suyo. Entonces
%$\forall M' \in [ M_0 \rangle,\,:\,I^T \cdot M = I^T \cdot M_0$.
%
%\proof Basta aplicar la definici\'{o}n de P-invariante y la
%ecuaci\'{o}n de estado, multiplicando dicha ecuaci\'{o}n por su
%izquierda por $I^T$.
%\eprop
%
%De esta propiedad se concluye que si $N$ es una Red de Petri
%Marcada Pura dotada de un P-semiflujo, entonces $N$ es
%conservativa respecto de dicho P-semiflujo. Adem\'{a}s,
%aunque dicha propiedad no permite resolver el problema
%de alcance, s\'{\i} permite demostrar que un determinado marcaje
%no est\'{a} en el alcance de la red. Es decir, un marcaje
%determinado $M$ no estar\'{a} en el alcance de $N$ si
%$I^T \cdot M \neq I^T \cdot M_0$. Otra propiedad que puede
%ser resuelta mediante el uso de P-semiflujos es la seguridad,
%pues una consecuencia de que el n\'{u}mero entero $n = I^T \cdot M$
%sea constante para todos los  marcajes alcanzables es que
%todos los lugares de $N$ asociados a componentes no nulas
%de $I$ (lugares del soporte de $I$) son n-seguros.
%
%Lamentablemente, la propiedad anterior de invarianza de la
%cantidad $I^T \cdot M$ no caracteriza plenamente a los
%{\it P-invariantes}, es decir, el rec\'{\i}proco no es en general
%cierto. Ello se debe a la posibilidad de que en el marcaje inicial
%no contemos con suficientes tokens como para poner a prueba todas
%las transiciones del sistema. Sin embargo, si fortalecemos un poco
%las condiciones del marcaje de partida, conseguimos un rec\'{\i}proco.
%
%\bthm Sea $N=(P,T,F,W,M_0)$ una Red de Petri Marcada Pura, con
%$M_0$ un marcaje que cumple la siguiente propiedad: $\forall t \in T
%\;\exists M_t \in [ M_0 \rangle\,$que permite la activaci\'{o}n de $t$.
%Sea adem\'{a}s $I$ un {\it P-vector} tal que $I^T \cdot M = I^T \cdot M_0$,
%para todo $M \in [ M_0 \rangle$. Entonces $I$ es un {\it P-invariante}
%de $N$.
%
%\proof Para cada transici\'{o}n $t$ sea el marcaje $M_t$. El disparo
%de $t$ bajo dicho marcaje proporciona el marcaje $M'_t$, dado por:
%\[ M'_t = M_t + C(-,t) \]
%Ahora bien, por la hip\'{o}tesis efectuada sobre $I$, $I^T \cdot M'_t =
%I^T \cdot M_t$, lo que implica: $I^T \cdot C(-,t) = 0$. Esto es
%v\'{a}lido para todo $t$, por tanto $I^T \cdot C = 0$.
%\ethm
%
%De forma totalmente sim\'{e}trica podemos definir los {\it T-invariantes},
%como anuladores derechos de la matriz de incidencia. Ellos nos
%permitir\'{a}n resolver tambi\'{e}n algunos de los problemas planteados.
%
%\bdfn (T-invariantes)\\
%Sea $N=(P,T,F,W,M_0)$ una Red de Petri Marcada Pura. Un vector
%de la forma $J\,:\, T \longrightarrow {\cal Z}$ se llama un
%{\it T-vector}.
%\begin{enumerate}
%\item Se dice que un {\it T-vector} $J$ es un {\it T-invariante} de $N$ sii
%$C \cdot J = 0$, siendo $C$ la matriz de incidencia de $N$.
%\item un {\it T-invariante} de $N$ se dice que es {\it no-negativo}
%sii todas sus componentes son mayores o iguales que 0.
%\item Un {\it T-invariante} no-negativo
%$J$ de $N$ se dice que es {\it minimal}
%si no existe otro {\it T-invariante} no-negativo $J'$ tal que
%$0 < J' < J$.
%\end{enumerate}
%\edfn
%
%El teorema siguiente relaciona los marcajes
%reproducibles ({\it home-states}) con
%los {\it T-invariantes}.
%
%\bthm Sea $N=(P,T,F,W,M_0)$ una Red de Petri Marcada Pura, y sea
%$M$ un {\it home-state} de $N$. Entonces, las transiciones de cada
%secuencia de ocurrencia $\sigma$ tal que $M [ \sigma \rangle M$
%constituyen el soporte de un {\it T-invariante} no-negativo de $N$.
%
%\proof Inmediata, a partir de la ecuaci\'{o}n de estado.
%\ethm
%
%Su rec\'{\i}proco es igualmente cierto, lo que transforma este teorema
%en una propiedad caracterizadora de los {\it T-invariantes}.
%
%\bthm Sea $N=(P,T,F,W,M_0)$ una Red de Petri Marcada Pura y
%sea $J$ un {\it T-invariante} no-negativo de $N$. Entonces existe un
%marcaje $M_J$, reproducible con el disparo de las
%transiciones del soporte de $J$, disparadas tantas veces como
%indique la correspondiente componente en $J$.
%
%\proof Basta considerar el marcaje $M_J(p) = \sumas{t \in p^{\bullet}}
%J(t) \cdot W(p,t)$, $\forall p \in P$. Su definici\'{o}n es correcta, al
%ser $T$ no-negativo. Basta entonces aplicar la ecuaci\'{o}n de estado para
%obtener $M_J + C \cdot J = M_J + 0 = M_J$.
%\ethm
%
%No obstante, la ciclicidad de la red no puede ser resuelta mediante
%el simple uso de {\it T-invariantes}. De hecho, tal cosa no puede
%garantizarse ni siquiera aunque cada estado alcanzable sea un
%{\it home-state}.
%
%\subsubsection{An\'{a}lisis Estructural Basado en Cerrojos y Trampas}
%
%El otro tipo de t\'{e}cnicas estructurales se apoyan en los conceptos de
%cerrojos y trampas, que son propiedades puramente
%estructurales, prescindiendo
%del marcaje de la red en un momento dado. Los cerrojos son conjuntos
%de lugares, que una vez que han perdido conjuntamente sus tokens permanecen
%desmarcados por siempre. Sim\'{e}tricamente,
%las trampas son conjuntos de lugares que
%permanecen marcados una vez que tienen un token.
%
%\bdfn (Cerrojos y Trampas)\\
%Sea $N=(P,T,F,M_0)$ una Red de Petri Ordinaria Marcada Pura Fuertemente
%Conexa (es decir, desde cualquier nodo, ya sea transici\'{o}n lugar, se puede
%ir a cualquier otro nodo).
%\begin{enumerate}
%\item $S \subseteq P$ se llama un {\it cerrojo} de $N$ sii
%$\,\precond{S} \subseteq S^{\bullet}$.
%\item $S \subseteq P$ se llama una {\it trampa} de $N$ sii
%$\,S^{\bullet} \subseteq \precond{S}$.
%\item Un cerrojo (resp. trampa) se dice que es {\it minimal} sii no existe
%otro cerrojo (resp. trampa) contenido en \'{e}l de forma estricta.
%\end{enumerate}
%\edfn
%
%El siguiente teorema expresa la interpretaci\'{o}n realizada de los cerrojos
%y trampas.
%
%\bthm Sea $N$ una Red de Petri Ordinaria Marcada Pura Fuer\-te\-men\-te
%Conexa.
%\begin{enumerate}
%\item Sea $S$ un cerrojo de $N$. Entonces:
%\[ (\forall s \in S\,:\,M(s) = 0) \,\Rightarrow\,
%\forall M' \in [ M \rangle \,:\,\forall s \in S\,:\,M'(s)=0\]
%Es decir, si en alg\'{u}n marcaje $M$ el cerrojo se vac\'{\i}a, entonces en todo
%marcaje posterior permanece vac\'{\i}o.
%\item Sea $S$ una trampa de $N$. Entonces:
%\[ (\exists s \in S\,:\,M(s)>0) \,\Rightarrow\,
%\forall M' \in [ M \rangle\,\exists s \in S\,:\,M'(s) > 0\]
%Es decir, si en alg\'{u}n marcaje $M$ la trampa adquiere un token, entonces
%en todo marcaje posterior la trampa permanece marcada.
%\end{enumerate}
%
%\proof Ver \cite{Lau86}.
%\ethm
%
%El teorema siguiente proporciona una condici\'{o}n suficiente para
%decidir la ausencia de bloqueos.
%
%\bthm Sea $N$ una Red de Petri Ordinaria en las condiciones de la
%definici\'{o}n anterior. Si todo cerrojo
%no vac\'{\i}o de $N$ contiene una trampa marcada en $M_0$, entonces
%$N$ no se bloquea.
%
%\proof Ver \cite{Lau86}.
%\ethm
%
%Sin embargo, el problema de la vivacidad tampoco puede resolverse en
%general con el uso de cerrojos y trampas, salvo en casos concretos,
%como son los grafos marcados, que son redes fuertemente conexas en las
%que todo lugar tiene una \'{u}nica
%transici\'{o}n precondici\'{o}n y una \'{u}nica
%transici\'{o}n postcondici\'{o}n.
%
\section{Extended Timed-Arc Petri Nets} \label{sec:def}
%We shall now give some preliminaries in order to define the model
%of extended timed-arc Petri nets. 
%We consider only discrete time
%as for closed intervals the reachability/soundness problems
%are equivalent with the continuous time variant as discussed in 
%Section~\ref{sec:cont}.
Let $\nnul = \mathbb{N} \cup \{0\}$ and 
$\nnul^{\infty} = \nnul \cup \left\{\infty \right\}$.
A \emph{discrete timed transition system} (DTTS) 
is a triple $\left(\proc, \act,\rightarrow\right)$
where $\proc$ is the set of states, $\act$ is the set of actions
and $\rightarrow\: \subseteq \proc \times (\act \cup \nnul)  \times \proc$ is the 
transition relation written as $s \trans{a} s'$ whenever $(s,a,s') \in \rightarrow$.
If $a \in \act$ then we call it a \emph{switch transition}, if
$a \in \nnul$ we call it a \emph{delay transition}.
%By $\rightarrow^{*}$ we denote the reflexive and transitive closure of 
%the relation
%$\rightarrow  \eqdef \bigcup_{a \in \act} \trans{a} \; \cup \; \bigcup_{d \in \nnul} \trans{d}$. 
We also define the set of \emph{well-formed time intervals} as 
$\int \eqdef \{[a,b] \mid a \in \nnul,b\in \nnul^{\infty}, a\leq b \}$
and its subset $\intinv \eqdef \{[0,b] \mid b\in \nnul^{\infty} \}$
used in age invariants. 


\begin{definition}[Extended timed-Arc Petri Net] \label{defetapn}  
An \emph{extended timed-arc Petri net} 
(ETAPN) is a 9-tuple $N = \tapntuple$ where 
\begin{itemize}
\item $P$ is a finite set of \emph{places},
\item $T$ is a finite set of \emph{transitions} 
such that $P \cap T = \emptyset$, 
\item $\Turg \subseteq T$ is the set of \emph{urgent transitions},
\item $\ia \subseteq P \times T$ is a finite set of \emph{input arcs},
\item $\oa \subseteq T \times P$ is a finite set of \emph{output arcs},
\item $\cfunction : \ia \rightarrow \int$ is a \emph{time constraint function} assigning 
guards %(time intervals) 
to input arcs,
\item $\wfunction : \ia\cup \oa \rightarrow \mathbb{N}$ is a function assigning \emph{weights} to input and output arcs,
\item $\type : \ia \cup \oa \rightarrow \types$ is a \emph{type function} assigning a type to all arcs where $\types = \{\normal, \inhib\} \cup \{\transporti \mid j \in \mathbb{N} \}$ such that  
\begin{itemize}
\item if $\type(a) = \inhib$ then $a \in \ia$ and $\cfunction(a)=[0,\infty]$, 
\item if $(p,t) \in \ia$ and $t \in \Turg$ then $\cfunction((p,t))=[0,\infty]$,
\item if $\type((p,t)) = \transporti$ for some $(p,t) \in \ia$ then there is exactly one $(t,p^{\prime}) \in \oa$ such that $\type((t,p^{\prime})) = 
\transporti$, 
%and moreover $\wfunction((p,t))=\wfunction((t,p^{\prime}))$,
\item if $\type((t,p^{\prime})) = \transporti$ for some $(t,p^{\prime}) \in \oa$ then there is exactly one $(p,t) \in \ia$ such that $\type((p,t)) = 
\transporti$, 
%and moreover $\wfunction((p,t))=\wfunction((t,p^{\prime}))$,
\item if $\type((p,t)) = \transporti = \type((t,p^{\prime}))$ 
then $\wfunction((p,t))=\wfunction((t,p^{\prime}))$,
\end{itemize}
\item $\inv : P \rightarrow \int^{inv}$ is a function assigning \emph{age invariants} to places.
\end{itemize}
\end{definition}

\begin{remark}
Note that for transport arcs we assume that they come in pairs (for
each type $\transporti$) so that their weights match.
Also for inhibitor arcs and for input arcs to urgent transitions, we
require that the guards are $[0,\infty]$. This restriction is important
for some of the results presented in this paper and it also guarantees that 
we can use DBM-based algorithms in the tool TAPAAL~\cite{DJJJMS:TACAS:12}.
\end{remark}

The ETAPN model is not monotonic, meaning
that adding more tokens to markings can disable time delays or
transition firing.
Therefore we define a subclass of 
ETAPN where the monotonicity breaking features are not allowed.
In the literature such nets are often considered as the standard
timed-arc Petri net model~\cite{BLT:90,Hanisch:93} but we add the 
prefix monotonic for clarity reasons. 

\begin{definition}[Monotonic timed-arc Petri net] \label{deftapn}
A \emph{monotonic timed-arc Petri net} 
(MTAPN) is an extended timed arc Petri net 
with no urgent transitions ($\Turg=\emptyset)$, no age invariants
($\inv(p) = [0,\infty]$ for all $p \in P$) and no 
inhibitor arcs ($\type(a) \not= \inhib$ for all $a \in \ia$).
\end{definition}


%Let $N = \tapntuple$ be a ETAPN and $P^\prime \subseteq P$, the projection $N|_{P^\prime}$ 
%is the net \ensuremath{(P^\prime, T, \ia^\prime,\allowbreak \oa^\prime, \cfunction^\prime, 
%\wfunction^\prime, \type^\prime, \inv^\prime)}, 
%where $\ia^\prime=\ia \cap (P^\prime \times T)$, $\oa^\prime=\oa \cap (T \times P^\prime)$,
%$\cfunction^\prime : \ia^\prime \rightarrow \int$, $\wfunction^\prime : \ia^\prime \cup \oa^\prime \rightarrow \mathbb{N}$,
%$\type^\prime : \ia^\prime \cup \oa^\prime \rightarrow \types$, and $\inv^\prime : P^\prime \rightarrow \int^{inv}$. From now on, we will denote by 
%$P_s$ the set of places shared by various nets. Then, let $N$, $N^\prime$ be two ETAPNs such that $P \cap P^\prime \subseteq P_s$, the disjoint union of $N$ and $N^\prime$ is a ETAPN \ensuremath{(P^{\prime\prime},T^{\prime\prime},\ia^{\prime\prime}, \oa^{\prime\prime},\cfunction^{\prime\prime},\wfunction^{\prime\prime},\type^{\prime\prime}, \inv^{\prime\prime})}, where $P^{\prime\prime}= P\cupdot P^\prime, T^{\prime\prime}=T\cupdot T^\prime, \ia^{\prime\prime}=\ia\cupdot \ia^\prime,\oa^{\prime\prime}=\oa \cupdot \oa^\prime,\cfunction^{\prime\prime}:\ia^{\prime\prime}\cup \oa^{\prime\prime}\rightarrow \int, \wfunction^{\prime\prime}: \ia^{\prime\prime}\cup \oa^{\prime\prime}\rightarrow \mathbb{N}, 
%\type^{\prime\prime} : \ia^{\prime\prime} \cup \oa^{\prime\prime} \rightarrow \types, \text{ and } \inv^{\prime\prime} : P^{\prime\prime} \rightarrow \int^{inv}$.

Before we give the formal semantics of the model, let us fix some notation.
Let $N = \tapntuple$ be an ETAPN. 
%Let $F\eqdef \ia \cup \oa$. 
We denote by ${}^\bullet x \eqdef 
\{y \in P \cup T \mid (y,x) \in (\ia \cup \oa),\ \type((y,x)) \neq \inhib \}$ 
the preset of a transition or a place $x$.
Similarly, the postset $x^\bullet$ is defined as 
$x^\bullet \eqdef \{y \in P \cup T \mid (x,y) \in (\ia \cup \oa) \}$.
Let $\mathcal{B}(\nnul)$ be the set 
of all finite multisets over $\nnul$. A \emph{marking} $M$ on $N$ 
is a function $M : P \longrightarrow \mathcal{B}(\nnul)$ 
where for every place $p \in P$ and 
every token $x \in M(p)$ we have $x \in \inv(p)$, in other words
all tokens have to satisfy the age invariants. 
%The projection of $P^\prime \subseteq P$ in $M$ is a function 
%$M|_{P^\prime} : P^\prime \longrightarrow \mathcal{B}(\nnul)$.
The set of all markings in a net $N$ 
is denoted by $\mathcal{M}(N)$.

We write $(p,x)$ to denote a token at a place $p$ with the 
age $x\in \nnul$. Then $M=\{(p_1,x_1),(p_2,x_2),\dots ,(p_n,x_n)\}$ 
is a multiset representing a marking $M$ with $n$ tokens of 
ages $x_i$ in places $p_i$. We 
define the size of a marking as $|M| = \sum_{p\in P}|M(p)|$ where 
$|M(p)|$ is the number of tokens located in the place $p$.

%A marked ETAPN 
%$(N,M_0)$ is a TAPN N together with an initial marking $M_0$ with all tokens of age $0$. 

\begin{definition}[Enabledness]
\label{def:enabledness}
 Let $N = \tapntuple$ be an ETAPN. 
We say that a transition $t \in T$ is \emph{enabled} in a marking $M$ by the 
multisets of tokens
$\inn = \{(p,x_{p}^1), (p,x_{p}^2), \dots ,(p,x_{p}^{\wfunction ((p,t))})\mid 
p \in {}^\bullet t\} \subseteq M$ and 
$\out = \{ (p^{\prime},x_{p^{\prime}}^1),
           (p^{\prime},x_{p^{\prime}}^2),
\dots ,\allowbreak
(p^{\prime},x_{p^{\prime}}^{\wfunction ((t,p^{\prime}))}) 
\mid p^{\prime} \in t^\bullet \}$ if
\begin{itemize}
\item for all input arcs except the inhibitor arcs, the tokens from $\inn$ satisfy the age guards of the arcs, i.e. 
%$$\forall(p,t) \in \ia, x_p^i \in \cfunction((p,t))\text{ for }1\leq i\leq w((p,t)) $$ 
$$\forall(p,t) \in \ia. \type((p,t)) \neq \inhib \Rightarrow  x_p^i \in \cfunction((p,t))\text{ for }1\leq i\leq w((p,t)) $$ 
%\item for each place $p$ in ${}^\bullet t$ there are $\wfunction ((p,t))$ tokens from $p$ in $\inn$, i.e. $$\forall p\in {}^\bullet t. \wfunction ((p,t))= n_{p} $$
%\item for each place $p^{\prime}$ in $t^\bullet $ there are $\wfunction ((t,p^{\prime}))$ tokens from $p^{\prime}$ in $\out$, i.e. $$\forall p^{\prime}\in t^\bullet . \wfunction ((t,p^{\prime}))= m_{p^{\prime}} $$
\item for any inhibitor arc pointing from a place $p$ to the
transition $t$, the number of tokens in $p$ is smaller than the weight of the arc, i.e.
$$\forall(p,t) \in \ia. \type((p,t)) = \inhib \Rightarrow|M(p)|<\wfunction ((p,t))$$ 
%$$\forall(p,t) \in \ia. \type((p,t)) = \inhib \Rightarrow \nexists x \in M(p). x \in \cfunction((p,t))$$
\item for all input arcs and output arcs which constitute a transport arc, 
the age of the input token must be equal to the age of the output token and satisfy the invariant of the output place, i.e.
\begin{eqnarray*}
&\forall(p,t) \in \ia. \forall(t,p^{\prime}) \in \oa.\type((p,t)) = \type((t,p^{\prime})) 
= \transporti \\
&\Rightarrow \big( x_p^i = x_{p^{\prime}}^i \wedge x_{p^{\prime}}^i \in 
\inv(p^{\prime})\big) \text{ for } 1\leq i \leq w((p,t))
\end{eqnarray*}
\item for all normal output arcs, the age of the output token is $0$, i.e. $$\forall(t,p^{\prime}) \in \oa. \type((t,p^{\prime})) = \normal \Rightarrow x_{p^{\prime}}^i = 0 \text{ for }1\leq i \leq w((p,t)).$$ 
\end{itemize}
\end{definition}

A given ETAPN $N$ %=\tapntuple$ 
defines a DTTS $T(N)\eqdef (\markingsof(N),T,\rightarrow)$
where states are the markings and the transitions are as follows. 
\begin{itemize}
\item If $t\in T$ is enabled in a marking $M$ by the  multisets of
tokens $\inn$ and $\out$ then $t$ can \emph{fire} and produce 
the marking $M^{\prime} = (M \smallsetminus \inn) \uplus \out$ 
where  $\uplus$ is the multiset sum operator and $\smallsetminus$ is the multiset 
difference operator; we write $M \trans{t} M^{\prime}$ for this 
switch transition.
\item A time \emph{delay} $d \in \nnul$ is allowed in $M$ if
\begin{itemize}
\item $(x+d) \in I(p)$ for all $p \in P$ and all $x \in M(p)$, and
% i.e. by delaying $d$ time units no token violates any of the age invariants, 
%and
\item if $M \trans{t} M'$ for some $t \in \Turg$ then $d=0$.
 %there is at least one urgent transition enabled in $M$ then
 %     $d=0$, i.e. enabled urgent transitions disallow time passing.
\end{itemize}
By delaying $d$ time units in $M$ we reach the marking $M^{\prime}$ defined as
$M^{\prime}(p) = \{x+d \mid x \in M(p)\}$ for all $p \in P$; 
we write $M \trans{d} M^{\prime}$ for this delay transition.
\end{itemize}

%A computation of a net $N$ from the initial marking $M_0$ is
%$M_0 \rightarrow M_1\rightarrow \cdots \rightarrow M_n$ is 
%denoted by $\{M_i\}_{i=0}^{n}$ 
%and we call it a \emph{run}. If the sequence is infinite, we write 
%$\{M_i\}_{i\geq 0}$. Moreover, we write $M \Rightarrow^* M^{\prime}$ if  
%$M^{\prime}$ is reachable from $M$ and $[M\rangle$ represents the set of reachable markings of $M$.

\noindent Let 
$\trans{} \eqdef \bigcup_{t \in T} \trans{t} \cup \bigcup_{d \in \nnul} \trans{d}$.
The set of all markings reachable %in the net $N$ 
from a given marking $M$ is denoted by 
$[M\rangle \eqdef \{ M' \mid M \trans{}^* M' \}$.
By $M \trans{d,t} M'$ we denote that there is a marking $M''$
such that $M \trans{d} M'' \trans{t} M'$.

A marking $M$ is a \emph{deadlock} if there is no $d \in \nnul$ and
no $t \in T$ such that $M \trans{d,t} M'$ 
for some marking $M'$.
A marking $M$ is \emph{divergent} if for any $d \in \nnul$
we have $M \trans{d} M'$ for some $M'$.


%\section{Finite Abstractions for Bounded ETAPNs}

In general, ETAPNs are infinite in two dimensions. The number of tokens
in reachable markings can be unbounded and even for bounded nets
the ages of tokens can be arbitrarily large. We shall now recall a 
few results that allow us to make finite abstractions for bounded
ETAPNs, i.e. for nets where the maximum number of tokens in any
reachable marking is bounded by a constant.

Let $N=\tapntuple$ be a given ETAPN.
In~\cite{ALSST:MEMICS:12} 
the authors provide an algorithm for computing 
a function $\cmax: P \rightarrow (\nnul \cup \{ -1 \})$ 
returning for each place $p \in P$ the maximum constant associated
to this place, meaning that the ages of tokens in place $p$ that
are strictly greater than $\cmax(p)$ are irrelevant. In particular,
places where $\cmax(p)=-1$ are the so-called \emph{untimed} places
where the age of tokens is not relevant at all, implying that all
the intervals on their ongoing arcs are $[0,\infty]$.

Let $M$ be a marking of $N$. We split it into 
two markings $\mold$ and $\myoung$ where 
$\mold (p)=\left\{ x\in M(p) \mid x>\cmax(p) \right\}$ 
and $\myoung (p)=\allowbreak\left\{ x\in M(p) \mid 
x\allowbreak\leq\allowbreak \cmax(p) \right\}$
for all places $p \in P$. Clearly,
$M = \mold \uplus \myoung$.

We say that two markings $M$ and $M'$ in the net $N$ are equivalent, 
written $M \eqMarking M^{\prime}$, 
if $\myoung=\myoung^{\prime}$
and for all $p \in P$ we have
$|\mold (p)|=|\mold^{\prime}(p)|$.
In other words $M$ and $M'$ agree on the tokens with ages below the
maximum constants and have the same number of tokens above the maximum
constant.
% (relevant only for places $p$ with $I(p)=[0,\infty]$ as
%places with nontrivial age invariants cannot have tokens older that 
%the maximum constant which is in this case equal to the invariant upper-bound).

The relation $\eqMarking$ is an equivalence relation and it is
also a timed bisimulation 
where delays and transition firings on one side can be matched by
exactly the same delays and transition firings on the other side
and vice versa. % (see e.g.~\cite{LY:97}).

\begin{theorem}[\cite{ALSST:MEMICS:12}]
\label{thm:bisim}
  The relation $\eqMarking$ is a timed bisimulation.
\end{theorem}

We can now define canonical representatives for each
equivalence class of $\eqMarking$. 

\begin{definition}[Cut]
\label{def:cut}
Let $M$ be a marking.
We define its canonical marking $\cut(M)$ by 
$\cut(M)(p)= \myoung(p)\uplus \big\{ \underbrace{ \cmax(p)+1,\dots ,\cmax(p)+1 }_{|\mold(p)| \text{ times}} \big\}$.
%\begin{equation*}
%  \cut(M)(p)=
%\begin{cases}
%\myoung(p)  \quad \text{if $p$ is invariant or dead-token place,} \\
%\myoung(p)\uplus \big\{ \underbrace{ \cmax(p)+1,\dots ,\cmax(p)+1 }_{|\mold(p)| \text{ times}} \big\}    \quad \text{if $p$ is a normal place.}
%\vspace{-.45cm}
%\end{cases}
%\vspace{.45cm}
%\end{equation*}
\end{definition}

\begin{lemma}[\cite{ALSST:MEMICS:12}]
\label{lemma:canon}
Let $M$, $M_1$ and $M_2$ be markings. Then
(i) $M \eqMarking \cut(M)$, and (ii)
$M_1 \eqMarking M_2$ if and only if $\cut(M_1)=\cut(M_2)$.
\end{lemma}

Let $M$ and $M^\prime$ be two markings. We say that $M^\prime$ \emph{covers} 
$M$, denoted by $M \sqsubseteq M^\prime$, if $M(p) \subseteq M^\prime(p)$ 
for all $p \in P$. We write $M \sqsubseteq_{cut} M^\prime$ 
if $cut(M) \sqsubseteq cut(M^\prime)$.

For monotonic timed-arc Petri nets we can now show that adding more tokens
to the net does not restrict its possible behaviour. 

\begin{lemma}
\label{lem:mono}
Let $N$ be an MTAPN and $M,M' \in \mathcal{M}(N)$
be two of its markings such that $M \sqsubseteq_{\cut} M'$. 
If $M \trans{d} M_1$ (resp. $M \trans{t} M_1$) then 
$M' \trans{d} M'_1$ (resp. $M' \trans{t} M'_1$) such that 
$M_1 \sqsubseteq_{cut} M'_1$ and 
$|M'|-|M|=|M'_1|-|M_1|$.
\end{lemma}
\begin{proof}
Let $M \trans{d} M_1$, resp. $M \trans{t} M_1$.
As $M \equiv \cut(M)$ by Lemma~\ref{lemma:canon}(i),
we can by Theorem~\ref{thm:bisim} conclude that also $\cut(M) \trans{d} M_2$,
resp. $\cut(M) \trans{t} M_2$,
such that $M_1 \equiv M_2$. Recall that $\cut(M) \sqsubseteq \cut(M')$
by the assumption of the lemma.
\begin{itemize}
\item Time delay case ($\cut(M) \trans{d} M_2$).
As the net does not contain any nontrivial age invariants
and there are no urgent transitions,
we know that also $\cut(M') \trans{d} M_3$ such that
$M_2 \sqsubseteq M_3$ as time delay preserves the $\sqsubseteq$-relation.
\item Transition firing case ($\cut(M) \trans{t} M_2$).
As the net does not have any inhibitor arcs,
we can see that also $\cut(M') \trans{t} M_3$ by consuming
exactly the same tokens in $\cut(M')$ as we did in $\cut(M)$.
Clearly, $M_2 \sqsubseteq M_3$.
\end{itemize}
Because $\cut(M') \equiv M'$ due to Lemma~\ref{lemma:canon}(i),
we know by Theorem~\ref{thm:bisim}
that $M' \trans{d} M'_1$, resp. $M' \trans{t} M'_1$, such that $M_3 \equiv M'_1$.
Hence $M_1 \equiv M_2 \sqsubseteq M_3 \equiv M'_1$.
By Lemma~\ref{lemma:canon}(ii) we get
$\cut(M_1)=\cut(M_2)$ and $\cut(M_3)=\cut(M'_1)$.
Observe now a simple fact that $M_2 \sqsubseteq M_3$ implies that
$\cut(M_2) \sqsubseteq \cut(M_3)$.
This all together implies that $\cut(M_1)=\cut(M_2) \sqsubseteq
\cut(M_3) = \cut(M'_1)$ which is another way of saying that
$M_1 \sqsubseteq_\cut M'_1$ as required by the lemma.
As time delays do not change the number of
tokens in $M$ and $M'$ and transition firing adds or removes an
equal number of tokens from both $M$ and $M'$,
we can also conclude that $|M'|-|M|=|M'_1|-|M_1|$.
\qed
\end{proof}






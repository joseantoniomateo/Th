\section{Extended Timed-Arc Petri Nets} \label{sec:def}
%We shall now give some preliminaries in order to define the model
%of extended timed-arc Petri nets. 
%We consider only discrete time
%as for closed intervals the reachability/soundness problems
%are equivalent with the continuous time variant as discussed in 
%Section~\ref{sec:cont}.
%Let $\nnul = \mathbb{N} \cup \{0\}$ and 
%$\nnul^{\infty} = \nnul \cup \left\{\infty \right\}$.
%A \emph{discrete timed transition system} (DTTS) 
%is a triple $\left(\proc, \act,\rightarrow\right)$
%where $\proc$ is the set of states, $\act$ is the set of actions
%and $\rightarrow\: \subseteq \proc \times (\act \cup \nnul)  \times \proc$ is the 
%transition relation written as $s \trans{a} s'$ whenever $(s,a,s') \in \rightarrow$.
%If $a \in \act$ then we call it a \emph{switch transition}, if
%$a \in \nnul$ we call it a \emph{delay transition}.
%%By $\rightarrow^{*}$ we denote the reflexive and transitive closure of 
%%the relation
%%$\rightarrow  \eqdef \bigcup_{a \in \act} \trans{a} \; \cup \; \bigcup_{d \in \nnul} \trans{d}$. 
%We also define the set of \emph{well-formed time intervals} as 
%$\int \eqdef \{[a,b] \mid a \in \nnul,b\in \nnul^{\infty}, a\leq b \}$
%and its subset $\intinv \eqdef \{[0,b] \mid b\in \nnul^{\infty} \}$
%used in age invariants. 

%%% Nice intro but repeats too much from introduction, do not remove it though
%We can now introduce the timed-arc Petri net model
%where each token carries its own
%age and input arcs to transitions are annotated with time intervals
%that restrict the ages of tokens usable for transition firing. Newly
%produced tokens become of age $0$, unless they were produced by
%a pair of so-called \emph{transport arcs} that describe
%a path along which the tokens travel from input to output places 
%while preserving the ages of the tokens moved along the path. In this paper
%we study an extended model with (i) \emph{age invariants}
%associated to places that restrict the maximal possible ages of tokens
%in the given places, (ii) \emph{inhibitor arcs} that as in normal Petri
%nets inhibit transition firing, and
%(iii) \emph{urgent transitions} that disable time delay whenever
%at least one of them is enabled.
%These additional extensions add a considerable expressive power
%(as also demonstrated by the decidability and undecidability results
%proved in this paper) and they are essential for convenient modelling 
%of systems with timing attributes.

%\begin{definition}[Extended timed-Arc Petri Net] \label{defetapn}  
%An \emph{extended timed-arc Petri net} 
%(ETAPN) is a 9-tuple $N = \tapntuple$ where 
%\begin{itemize}
%\item $P$ is a finite set of \emph{places},
%\item $T$ is a finite set of \emph{transitions} 
%such that $P \cap T = \emptyset$, 
%\item $\Turg \subseteq T$ is the set of \emph{urgent transitions},
%\item $\ia \subseteq P \times T$ is a finite set of \emph{input arcs},
%\item $\oa \subseteq T \times P$ is a finite set of \emph{output arcs},
%\item $\cfunction : \ia \rightarrow \int$ is a \emph{time constraint function} assigning 
%guards %(time intervals) 
%to input arcs,
%\item $\wfunction : \ia\cup \oa \rightarrow \mathbb{N}$ is a function assigning \emph{weights} to input and output arcs,
%\item $\type : \ia \cup \oa \rightarrow \types$ is a \emph{type function} assigning a type to all arcs where $\types = \{\normal, \inhib\} \cup \{\transporti \mid j \in \mathbb{N} \}$ such that  
%\begin{itemize}
%\item if $\type(a) = \inhib$ then $a \in \ia$ and $\cfunction(a)=[0,\infty]$, 
%\item if $(p,t) \in \ia$ and $t \in \Turg$ then $\cfunction((p,t))=[0,\infty]$,
%\item if $\type((p,t)) = \transporti$ for some $(p,t) \in \ia$ then there is exactly one $(t,p^{\prime}) \in \oa$ such that $\type((t,p^{\prime})) = 
%\transporti$, 
%%and moreover $\wfunction((p,t))=\wfunction((t,p^{\prime}))$,
%\item if $\type((t,p^{\prime})) = \transporti$ for some $(t,p^{\prime}) \in \oa$ then there is exactly one $(p,t) \in \ia$ such that $\type((p,t)) = 
%\transporti$, 
%%and moreover $\wfunction((p,t))=\wfunction((t,p^{\prime}))$,
%\item if $\type((p,t)) = \transporti = \type((t,p^{\prime}))$ 
%then $\wfunction((p,t))=\wfunction((t,p^{\prime}))$,
%\end{itemize}
%\item $\inv : P \rightarrow \int^{inv}$ is a function assigning \emph{age invariants} to places.
%\end{itemize}
%\end{definition}
%
%\begin{remark}
%Note that for transport arcs we assume that they come in pairs (for
%each type $\transporti$) so that their weights match.
%Also for inhibitor arcs and for input arcs to urgent transitions, we
%require that the guards are $[0,\infty]$. This restriction is important
%for some of the results presented in this paper and it also guarantees that 
%we can use DBM-based algorithms in the tool TAPAAL~\cite{DJJJMS:TACAS:12}.
%\end{remark}
%
%The ETAPN model is not monotonic, meaning
%that adding more tokens to markings can disable time delays or
%transition firing.
%Therefore we define a subclass of 
%ETAPN where the monotonicity breaking features are not allowed.
%In the literature such nets are often considered as the standard
%timed-arc Petri net model~\cite{BLT:90,Hanisch:93} but we add the 
%prefix monotonic for clarity reasons. 
%
%\begin{definition}[Monotonic timed-arc Petri net] \label{deftapn}
%A \emph{monotonic timed-arc Petri net} 
%(MTAPN) is an extended timed arc Petri net 
%with no urgent transitions ($\Turg=\emptyset)$, no age invariants
%($\inv(p) = [0,\infty]$ for all $p \in P$) and no 
%inhibitor arcs ($\type(a) \not= \inhib$ for all $a \in \ia$).
%\end{definition}


%Let $N = \tapntuple$ be a ETAPN and $P^\prime \subseteq P$, the projection $N|_{P^\prime}$ 
%is the net \ensuremath{(P^\prime, T, \ia^\prime,\allowbreak \oa^\prime, \cfunction^\prime, 
%\wfunction^\prime, \type^\prime, \inv^\prime)}, 
%where $\ia^\prime=\ia \cap (P^\prime \times T)$, $\oa^\prime=\oa \cap (T \times P^\prime)$,
%$\cfunction^\prime : \ia^\prime \rightarrow \int$, $\wfunction^\prime : \ia^\prime \cup \oa^\prime \rightarrow \mathbb{N}$,
%$\type^\prime : \ia^\prime \cup \oa^\prime \rightarrow \types$, and $\inv^\prime : P^\prime \rightarrow \int^{inv}$. From now on, we will denote by 
%$P_s$ the set of places shared by various nets. Then, let $N$, $N^\prime$ be two ETAPNs such that $P \cap P^\prime \subseteq P_s$, the disjoint union of $N$ and $N^\prime$ is a ETAPN \ensuremath{(P^{\prime\prime},T^{\prime\prime},\ia^{\prime\prime}, \oa^{\prime\prime},\cfunction^{\prime\prime},\wfunction^{\prime\prime},\type^{\prime\prime}, \inv^{\prime\prime})}, where $P^{\prime\prime}= P\cupdot P^\prime, T^{\prime\prime}=T\cupdot T^\prime, \ia^{\prime\prime}=\ia\cupdot \ia^\prime,\oa^{\prime\prime}=\oa \cupdot \oa^\prime,\cfunction^{\prime\prime}:\ia^{\prime\prime}\cup \oa^{\prime\prime}\rightarrow \int, \wfunction^{\prime\prime}: \ia^{\prime\prime}\cup \oa^{\prime\prime}\rightarrow \mathbb{N}, 
%\type^{\prime\prime} : \ia^{\prime\prime} \cup \oa^{\prime\prime} \rightarrow \types, \text{ and } \inv^{\prime\prime} : P^{\prime\prime} \rightarrow \int^{inv}$.

%Before we give the formal semantics of the model, let us fix some notation.
%Let $N = \tapntuple$ be an ETAPN. 
%%Let $F\eqdef \ia \cup \oa$. 
%We denote by ${}^\bullet x \eqdef 
%\{y \in P \cup T \mid (y,x) \in (\ia \cup \oa),\ \type((y,x)) \neq \inhib \}$ 
%the preset of a transition or a place $x$.
%Similarly, the postset $x^\bullet$ is defined as 
%$x^\bullet \eqdef \{y \in P \cup T \mid (x,y) \in (\ia \cup \oa) \}$.
%Let $\mathcal{B}(\nnul)$ be the set 
%of all finite multisets over $\nnul$. A \emph{marking} $M$ on $N$ 
%is a function $M : P \longrightarrow \mathcal{B}(\nnul)$ 
%where for every place $p \in P$ and 
%every token $x \in M(p)$ we have $x \in \inv(p)$, in other words
%all tokens have to satisfy the age invariants. 
%%The projection of $P^\prime \subseteq P$ in $M$ is a function 
%%$M|_{P^\prime} : P^\prime \longrightarrow \mathcal{B}(\nnul)$.
%The set of all markings in a net $N$ 
%is denoted by $\mathcal{M}(N)$.
%
%We write $(p,x)$ to denote a token at a place $p$ with the 
%age $x\in \nnul$. Then $M=\{(p_1,x_1),(p_2,x_2),\dots ,(p_n,x_n)\}$ 
%is a multiset representing a marking $M$ with $n$ tokens of 
%ages $x_i$ in places $p_i$. We 
%define the size of a marking as $|M| = \sum_{p\in P}|M(p)|$ where 
%$|M(p)|$ is the number of tokens located in the place $p$.
%
%%A marked ETAPN 
%%$(N,M_0)$ is a TAPN N together with an initial marking $M_0$ with all tokens of age $0$. 
%
%\begin{definition}[Enabledness]
%\label{def:enabledness}
% Let $N = \tapntuple$ be an ETAPN. 
%We say that a transition $t \in T$ is \emph{enabled} in a marking $M$ by the 
%multisets of tokens
%$\inn = \{(p,x_{p}^1), (p,x_{p}^2), \dots ,(p,x_{p}^{\wfunction ((p,t))})\mid 
%p \in {}^\bullet t\} \subseteq M$ and 
%$\out = \{ (p^{\prime},x_{p^{\prime}}^1),
%           (p^{\prime},x_{p^{\prime}}^2),
%\dots ,\allowbreak
%(p^{\prime},x_{p^{\prime}}^{\wfunction ((t,p^{\prime}))}) 
%\mid p^{\prime} \in t^\bullet \}$ if
%\begin{itemize}
%\item for all input arcs except the inhibitor arcs, the tokens from $\inn$ satisfy the age guards of the arcs, i.e. 
%%$$\forall(p,t) \in \ia, x_p^i \in \cfunction((p,t))\text{ for }1\leq i\leq w((p,t)) $$ 
%$$\forall(p,t) \in \ia. \type((p,t)) \neq \inhib \Rightarrow  x_p^i \in \cfunction((p,t))\text{ for }1\leq i\leq w((p,t)) $$ 
%%\item for each place $p$ in ${}^\bullet t$ there are $\wfunction ((p,t))$ tokens from $p$ in $\inn$, i.e. $$\forall p\in {}^\bullet t. \wfunction ((p,t))= n_{p} $$
%%\item for each place $p^{\prime}$ in $t^\bullet $ there are $\wfunction ((t,p^{\prime}))$ tokens from $p^{\prime}$ in $\out$, i.e. $$\forall p^{\prime}\in t^\bullet . \wfunction ((t,p^{\prime}))= m_{p^{\prime}} $$
%\item for any inhibitor arc pointing from a place $p$ to the
%transition $t$, the number of tokens in $p$ is smaller than the weight of the arc, i.e.
%$$\forall(p,t) \in \ia. \type((p,t)) = \inhib \Rightarrow|M(p)|<\wfunction ((p,t))$$ 
%%$$\forall(p,t) \in \ia. \type((p,t)) = \inhib \Rightarrow \nexists x \in M(p). x \in \cfunction((p,t))$$
%\item for all input arcs and output arcs which constitute a transport arc, 
%the age of the input token must be equal to the age of the output token and satisfy the invariant of the output place, i.e.
%\begin{eqnarray*}
%&\forall(p,t) \in \ia. \forall(t,p^{\prime}) \in \oa.\type((p,t)) = \type((t,p^{\prime})) 
%= \transporti \\
%&\Rightarrow \big( x_p^i = x_{p^{\prime}}^i \wedge x_{p^{\prime}}^i \in 
%\inv(p^{\prime})\big) \text{ for } 1\leq i \leq w((p,t))
%\end{eqnarray*}
%\item for all normal output arcs, the age of the output token is $0$, i.e. $$\forall(t,p^{\prime}) \in \oa. \type((t,p^{\prime})) = \normal \Rightarrow x_{p^{\prime}}^i = 0 \text{ for }1\leq i \leq w((p,t)).$$ 
%\end{itemize}
%\end{definition}
%
%A given ETAPN $N$ %=\tapntuple$ 
%defines a DTTS $T(N)\eqdef (\markingsof(N),T,\rightarrow)$
%where states are the markings and the transitions are as follows. 
%\begin{itemize}
%\item If $t\in T$ is enabled in a marking $M$ by the  multisets of
%tokens $\inn$ and $\out$ then $t$ can \emph{fire} and produce 
%the marking $M^{\prime} = (M \smallsetminus \inn) \uplus \out$ 
%where  $\uplus$ is the multiset sum operator and $\smallsetminus$ is the multiset 
%difference operator; we write $M \trans{t} M^{\prime}$ for this 
%switch transition.
%\item A time \emph{delay} $d \in \nnul$ is allowed in $M$ if
%\begin{itemize}
%\item $(x+d) \in I(p)$ for all $p \in P$ and all $x \in M(p)$, and
%% i.e. by delaying $d$ time units no token violates any of the age invariants, 
%%and
%\item if $M \trans{t} M'$ for some $t \in \Turg$ then $d=0$.
% %there is at least one urgent transition enabled in $M$ then
% %     $d=0$, i.e. enabled urgent transitions disallow time passing.
%\end{itemize}
%By delaying $d$ time units in $M$ we reach the marking $M^{\prime}$ defined as
%$M^{\prime}(p) = \{x+d \mid x \in M(p)\}$ for all $p \in P$; 
%we write $M \trans{d} M^{\prime}$ for this delay transition.
%\end{itemize}
%
%%A computation of a net $N$ from the initial marking $M_0$ is
%%$M_0 \rightarrow M_1\rightarrow \cdots \rightarrow M_n$ is 
%%denoted by $\{M_i\}_{i=0}^{n}$ 
%%and we call it a \emph{run}. If the sequence is infinite, we write 
%%$\{M_i\}_{i\geq 0}$. Moreover, we write $M \Rightarrow^* M^{\prime}$ if  
%%$M^{\prime}$ is reachable from $M$ and $[M\rangle$ represents the set of reachable markings of $M$.
%
%\noindent Let 
%$\trans{} \eqdef \bigcup_{t \in T} \trans{t} \cup \bigcup_{d \in \nnul} \trans{d}$.
%The set of all markings reachable %in the net $N$ 
%from a given marking $M$ is denoted by 
%$[M\rangle \eqdef \{ M' \mid M \trans{}^* M' \}$.
%By $M \trans{d,t} M'$ we denote that there is a marking $M''$
%such that $M \trans{d} M'' \trans{t} M'$.
%
%A marking $M$ is a \emph{deadlock} if there is no $d \in \nnul$ and
%no $t \in T$ such that $M \trans{d,t} M'$ 
%for some marking $M'$.
%A marking $M$ is \emph{divergent} if for any $d \in \nnul$
%we have $M \trans{d} M'$ for some $M'$.


%\section{Finite Abstractions for Bounded ETAPNs}

In general, ETAPNs are infinite in two dimensions. The number of tokens
in reachable markings can be unbounded and even for bounded nets
the ages of tokens can be arbitrarily large. We shall now recall a 
few results that allow us to make finite abstractions for bounded
ETAPNs, i.e. for nets where the maximum number of tokens in any
reachable marking is bounded by a constant.

Let $N=\tapntuple$ be a given ETAPN.
In~\cite{ALSST:MEMICS:12} 
the authors provide an algorithm for computing 
a function $\cmax: P \rightarrow (\nnul \cup \{ -1 \})$ 
returning for each place $p \in P$ the maximum constant associated
to this place, meaning that the ages of tokens in place $p$ that
are strictly greater than $\cmax(p)$ are irrelevant. In particular,
places where $\cmax(p)=-1$ are the so-called \emph{untimed} places
where the age of tokens is not relevant at all, implying that all
the intervals on their ongoing arcs are $[0,\infty]$.

Let $M$ be a marking of $N$. We split it into 
two markings $\mold$ and $\myoung$ where 
$\mold (p)=\left\{ x\in M(p) \mid x>\cmax(p) \right\}$ 
and $\myoung (p)=\allowbreak\left\{ x\in M(p) \mid 
x\allowbreak\leq\allowbreak \cmax(p) \right\}$
for all places $p \in P$. Clearly,
$M = \mold \uplus \myoung$.

We say that two markings $M$ and $M'$ in the net $N$ are equivalent, 
written $M \eqMarking M^{\prime}$, 
if $\myoung=\myoung^{\prime}$
and for all $p \in P$ we have
$|\mold (p)|=|\mold^{\prime}(p)|$.
In other words $M$ and $M'$ agree on the tokens with ages below the
maximum constants and have the same number of tokens above the maximum
constant.
% (relevant only for places $p$ with $I(p)=[0,\infty]$ as
%places with nontrivial age invariants cannot have tokens older that 
%the maximum constant which is in this case equal to the invariant upper-bound).

The relation $\eqMarking$ is an equivalence relation and it is
also a timed bisimulation 
where delays and transition firings on one side can be matched by
exactly the same delays and transition firings on the other side
and vice versa. % (see e.g.~\cite{LY:97}).

\begin{theorem}[\cite{ALSST:MEMICS:12}]
\label{thm:bisim}
  The relation $\eqMarking$ is a timed bisimulation.
\end{theorem}

We can now define canonical representatives for each
equivalence class of $\eqMarking$. 

\begin{definition}[Cut]
\label{def:cut}
Let $M$ be a marking.
We define its canonical marking $\cut(M)$ by 
$\cut(M)(p)= \myoung(p)\uplus \big\{ \underbrace{ \cmax(p)+1,\dots ,\cmax(p)+1 }_{|\mold(p)| \text{ times}} \big\}$.
%\begin{equation*}
%  \cut(M)(p)=
%\begin{cases}
%\myoung(p)  \quad \text{if $p$ is invariant or dead-token place,} \\
%\myoung(p)\uplus \big\{ \underbrace{ \cmax(p)+1,\dots ,\cmax(p)+1 }_{|\mold(p)| \text{ times}} \big\}    \quad \text{if $p$ is a normal place.}
%\vspace{-.45cm}
%\end{cases}
%\vspace{.45cm}
%\end{equation*}
\end{definition}

\begin{lemma}[\cite{ALSST:MEMICS:12}]
\label{lemma:canon}
Let $M$, $M_1$ and $M_2$ be markings. Then
(i) $M \eqMarking \cut(M)$, and (ii)
$M_1 \eqMarking M_2$ if and only if $\cut(M_1)=\cut(M_2)$.
\end{lemma}

Let $M$ and $M^\prime$ be two markings. We say that $M^\prime$ \emph{covers} 
$M$, denoted by $M \sqsubseteq M^\prime$, if $M(p) \subseteq M^\prime(p)$ 
for all $p \in P$. We write $M \sqsubseteq_{cut} M^\prime$ 
if $cut(M) \sqsubseteq cut(M^\prime)$.

For monotonic timed-arc Petri nets we can now show that adding more tokens
to the net does not restrict its possible behaviour. 

\begin{lemma}
\label{lem:mono}
Let $N$ be an MTAPN and $M,M' \in \mathcal{M}(N)$
be two of its markings such that $M \sqsubseteq_{\cut} M'$. 
If $M \trans{d} M_1$ (resp. $M \trans{t} M_1$) then 
$M' \trans{d} M'_1$ (resp. $M' \trans{t} M'_1$) such that 
$M_1 \sqsubseteq_{cut} M'_1$ and 
$|M'|-|M|=|M'_1|-|M_1|$.
\end{lemma}
\begin{proof}
Let $M \trans{d} M_1$, resp. $M \trans{t} M_1$.
As $M \equiv \cut(M)$ by Lemma~\ref{lemma:canon}(i),
we can by Theorem~\ref{thm:bisim} conclude that also $\cut(M) \trans{d} M_2$,
resp. $\cut(M) \trans{t} M_2$,
such that $M_1 \equiv M_2$. Recall that $\cut(M) \sqsubseteq \cut(M')$
by the assumption of the lemma.
\begin{itemize}
\item Time delay case ($\cut(M) \trans{d} M_2$).
As the net does not contain any nontrivial age invariants
and there are no urgent transitions,
we know that also $\cut(M') \trans{d} M_3$ such that
$M_2 \sqsubseteq M_3$ as time delay preserves the $\sqsubseteq$-relation.
\item Transition firing case ($\cut(M) \trans{t} M_2$).
As the net does not have any inhibitor arcs,
we can see that also $\cut(M') \trans{t} M_3$ by consuming
exactly the same tokens in $\cut(M')$ as we did in $\cut(M)$.
Clearly, $M_2 \sqsubseteq M_3$.
\end{itemize}
Because $\cut(M') \equiv M'$ due to Lemma~\ref{lemma:canon}(i),
we know by Theorem~\ref{thm:bisim}
that $M' \trans{d} M'_1$, resp. $M' \trans{t} M'_1$, such that $M_3 \equiv M'_1$.
Hence $M_1 \equiv M_2 \sqsubseteq M_3 \equiv M'_1$.
By Lemma~\ref{lemma:canon}(ii) we get
$\cut(M_1)=\cut(M_2)$ and $\cut(M_3)=\cut(M'_1)$.
Observe now a simple fact that $M_2 \sqsubseteq M_3$ implies that
$\cut(M_2) \sqsubseteq \cut(M_3)$.
This all together implies that $\cut(M_1)=\cut(M_2) \sqsubseteq
\cut(M_3) = \cut(M'_1)$ which is another way of saying that
$M_1 \sqsubseteq_\cut M'_1$ as required by the lemma.
As time delays do not change the number of
tokens in $M$ and $M'$ and transition firing adds or removes an
equal number of tokens from both $M$ and $M'$,
we can also conclude that $|M'|-|M|=|M'_1|-|M_1|$.
\qed
\end{proof}




We say that two markings $M$ and $M'$ in the net $N$ are equivalent, 
written $M \eqMarking M^{\prime}$, 
if $\myoung=\myoung^{\prime}$
and for all $p \in P$ we have
$|\mold (p)|=|\mold^{\prime}(p)|$.
In other words $M$ and $M'$ agree on the tokens with ages below the
maximum constants and have the same number of tokens above the maximum
constant.
% (relevant only for places $p$ with $I(p)=[0,\infty]$ as
%places with nontrivial age invariants cannot have tokens older that 
%the maximum constant which is in this case equal to the invariant upper-bound).

The relation $\eqMarking$ is an equivalence relation and it is
also a timed bisimulation 
where delays and transition firings on one side can be matched by
exactly the same delays and transition firings on the other side
and vice versa. % (see e.g.~\cite{LY:97}).

\begin{theorem}[\cite{ALSST:MEMICS:12}]
\label{thm:bisim}
  The relation $\eqMarking$ is a timed bisimulation.
\end{theorem}

We can now define canonical representatives for each
equivalence class of $\eqMarking$. 

\begin{definition}[Cut]
\label{def:cut}
Let $M$ be a marking.
We define its canonical marking $\cut(M)$ by 
$\cut(M)(p)= \myoung(p)\uplus \big\{ \underbrace{ \cmax(p)+1,\dots ,\cmax(p)+1 }_{|\mold(p)| \text{ times}} \big\}$.
%\begin{equation*}
%  \cut(M)(p)=
%\begin{cases}
%\myoung(p)  \quad \text{if $p$ is invariant or dead-token place,} \\
%\myoung(p)\uplus \big\{ \underbrace{ \cmax(p)+1,\dots ,\cmax(p)+1 }_{|\mold(p)| \text{ times}} \big\}    \quad \text{if $p$ is a normal place.}
%\vspace{-.45cm}
%\end{cases}
%\vspace{.45cm}
%\end{equation*}
\end{definition}

\begin{lemma}[\cite{ALSST:MEMICS:12}]
\label{lemma:canon}
Let $M$, $M_1$ and $M_2$ be markings. Then
(i) $M \eqMarking \cut(M)$, and (ii)
$M_1 \eqMarking M_2$ if and only if $\cut(M_1)=\cut(M_2)$.
\end{lemma}

Let $M$ and $M^\prime$ be two markings. We say that $M^\prime$ \emph{covers} 
$M$, denoted by $M \sqsubseteq M^\prime$, if $M(p) \subseteq M^\prime(p)$ 
for all $p \in P$. We write $M \sqsubseteq_{cut} M^\prime$ 
if $cut(M) \sqsubseteq cut(M^\prime)$.

For monotonic timed-arc Petri nets we can now show that adding more tokens
to the net does not restrict its possible behaviour. 

\begin{lemma}
\label{lem:mono}
Let $N$ be an MTAPN and $M,M' \in \mathcal{M}(N)$
be two of its markings such that $M \sqsubseteq_{\cut} M'$. 
If $M \trans{d} M_1$ (resp. $M \trans{t} M_1$) then 
$M' \trans{d} M'_1$ (resp. $M' \trans{t} M'_1$) such that 
$M_1 \sqsubseteq_{cut} M'_1$ and 
$|M'|-|M|=|M'_1|-|M_1|$.
\end{lemma}
\begin{proof}
Let $M \trans{d} M_1$, resp. $M \trans{t} M_1$.
As $M \equiv \cut(M)$ by Lemma~\ref{lemma:canon}(i),
we can by Theorem~\ref{thm:bisim} conclude that also $\cut(M) \trans{d} M_2$,
resp. $\cut(M) \trans{t} M_2$,
such that $M_1 \equiv M_2$. Recall that $\cut(M) \sqsubseteq \cut(M')$
by the assumption of the lemma.
\begin{itemize}
\item Time delay case ($\cut(M) \trans{d} M_2$).
As the net does not contain any nontrivial age invariants
and there are no urgent transitions,
we know that also $\cut(M') \trans{d} M_3$ such that
$M_2 \sqsubseteq M_3$ as time delay preserves the $\sqsubseteq$-relation.
\item Transition firing case ($\cut(M) \trans{t} M_2$).
As the net does not have any inhibitor arcs,
we can see that also $\cut(M') \trans{t} M_3$ by consuming
exactly the same tokens in $\cut(M')$ as we did in $\cut(M)$.
Clearly, $M_2 \sqsubseteq M_3$.
\end{itemize}
Because $\cut(M') \equiv M'$ due to Lemma~\ref{lemma:canon}(i),
we know by Theorem~\ref{thm:bisim}
that $M' \trans{d} M'_1$, resp. $M' \trans{t} M'_1$, such that $M_3 \equiv M'_1$.
Hence $M_1 \equiv M_2 \sqsubseteq M_3 \equiv M'_1$.
By Lemma~\ref{lemma:canon}(ii) we get
$\cut(M_1)=\cut(M_2)$ and $\cut(M_3)=\cut(M'_1)$.
Observe now a simple fact that $M_2 \sqsubseteq M_3$ implies that
$\cut(M_2) \sqsubseteq \cut(M_3)$.
This all together implies that $\cut(M_1)=\cut(M_2) \sqsubseteq
\cut(M_3) = \cut(M'_1)$ which is another way of saying that
$M_1 \sqsubseteq_\cut M'_1$ as required by the lemma.
As time delays do not change the number of
tokens in $M$ and $M'$ and transition firing adds or removes an
equal number of tokens from both $M$ and $M'$,
we can also conclude that $|M'|-|M|=|M'_1|-|M_1|$.
\end{proof}




\section{Soundness for Continuous Time Workflow Nets} \label{sec:cont}

We have so far considered only discrete time workflow nets as this
is often sufficient for the practical applications. Nevertheless,
we shall now also discuss how the soundness
notion changes under the assumption of continuous time. 
For continuous time workflow nets,
we have marking as a function $M : P \longrightarrow \mathcal{B}(\rnul)$ 
where $\mathcal{B}(\rnul)$ is the set of all finite multisets over
nonnegative real numbers. Hence for every place $p \in P$ and 
every token $x \in M(p)$ we have $x \in \rnul$. We allow delay
transitions not only for integers but for arbitrary real numbers.
Otherwise the definition of continuous timed-arc workflow nets
is the same as in the case of the discrete semantics given earlier.

It is well-known that for closed 
timed automata (without any strict inequalities in guards), 
the location reachability problems for continuous and
discrete time delays coincide (see e.g.~\cite{AMP:CONCUR:98,danny_VLiet}).
We prove the same result also for extended timed-arc Petri
nets, following the idea from~\cite{danny_VLiet} where
the problem is reduced to an instance of 
linear programming, though with additional technical challenges 
for the Petri net case.
The result also implies that
%that the sets of untimed reachable markings
%in the continuous and discrete semantics coincide. Moreover, 
the discrete time semantics is sufficient for finding the minimum
and maximum execution times.

\begin{theorem}\label{thm:discvscont}
Let $N$ be an ETAPN and let $M_0$ be a marking on $N$ with all tokens
of integer ages. For any computation 
$M_0 \trans{d_0,t_0} M_1 \trans{d_1,t_1} M_2 \trans{d_2,t_2} 
\ldots \trans{d_{n-1},t_{n-1}} M_{n}$ in $N$ 
with $d_i \in \rnul$ for all $i$, $0 \leq i < n$, there is a computation 
$M_0 \trans{d^\prime_0,t_0} M^\prime_1 \trans{d^\prime_1,t_1} M^\prime_2 \trans{d^\prime_2,t_2} 
\ldots \trans{d^\prime_{n-1},t_{n-1}} M^\prime_{n}$ in $N$ such that
$d^\prime_i \in \nnul$ for all $i$, $0 \leq i < n$, and
$|M_n(p)|=|M^\prime_n(p)|$ for all $p \in P$.
Moreover if $N$ is a workflow net then there exists a computation with 
integer delays that achieves
the minimum, and if it exists also the maximum, execution time. 
\end{theorem}
\begin{proof}
Given a computation with real-time delays, we construct a set of linear
inequalities that describe all possible delays allowed for the execution
of the given trace. We only need difference constraints
for this purpose and hence the corresponding matrix 
%in the linear programming 
is totally unimodular~\cite{umodmatrix1}. As the instance 
has a real solution, it has also an optimal integral 
solution~\cite{umodmatrix2}. 
%The optimality  means 
%that maximising or minimising
%the accumulated delay on such a computation can be achieved via integer delays
%only. 
\end{proof}

As expected, continuous soundness now implies soundness in the discrete case.
Moreover, the discrete and continuous soundness coincide for a subclass
of workflow nets that do not enforce any urgent behaviour.

\begin{theorem} \label{thm:implication}
Let $N$ be an ETAWFN. If $N$ is sound in the continuous semantics
then it is sound in the discrete semantics.
\end{theorem}
\begin{proof}
Let $N$ be sound in the continuous semantics. Let $M$ be a marking
reachable from the initial marking $M_\mathit{in}$ in the discrete semantics.
As $M$ is clearly reachable also in the continuous semantics,
condition $b)$ of Definition~\ref{def:soundness} is clearly satisfied.
Regarding condition $a)$ of the definition of soundness, we know
that some final marking $M_\mathit{out}$ is reachable from $M$ in the continuous
semantics. However, using Theorem~\ref{thm:discvscont} we can conclude
that a marking $M'_\mathit{out}$ that has the same distribution of tokens
as $M_\mathit{out}$ is reachable from $M$ also in the discrete semantics,
and hence $N$ is sound w.r.t. the discrete semantics.
\end{proof}

If we moreover consider only workflow nets where time delays are not restricted
by neither age invariants nor urgency, then both the continuous and discrete
semantics coincide with respect to soundness. 

\begin{theorem} \label{thm:contdiscok}
Let $N$ be an ETAWFN with no age invariants and no urgent transitions
(inhibitor arcs are allowed). Then $N$ is sound in the
continuous semantics if and only if $N$ is sound in the discrete semantics.
\end{theorem}
\begin{proof}
``$\Rightarrow$'': Follows from Theorem~\ref{thm:implication}.\\
\indent ``$\Leftarrow$'': Let $N$ be sound in the discrete semantics.
Let $M$ be a marking
reachable from the initial marking $M_\mathit{in}$ in the continuous
semantics.
%We can w.l.o.g. assume that $M_\mathit{in}$ is a marking
%where the token in the place \itn{in} has an integer age (if not then
%we start with the integer part of the token age and as the first
%transition delay until the fractional part is achieved; this is always
%possible as the net does not contain any age invariants or urgent transitions).
Now we can use Theorem~\ref{thm:discvscont} to argue that a marking $M'$
is reachable from $M_\mathit{in}$ in the discrete semantics (with integer
delays only) such that $|M(p)|=|M'(p)|$ for all $p \in P$.
As $M$ and $M'$ have the same number of tokens in all the places
and $N$ is sound in the discrete semantics, it follows that condition $b)$
in Definition~\ref{def:soundness} holds for $M'$ and hence also for $M$.
Let us now argue for condition $a)$.
Let $M \trans{d} M_1$ where $d$ is an integer
greater than any constant used in any interval on any input arc.
Such a delay is possible as the net does not contain
any age invariants or urgent transitions. Clearly, $M' \trans{d} M'_1$
is possible also in the discrete semantics. However, now
all tokens in $M_1$ and $M'_1$ are greater than any constant appearing
in the net and hence these two markings are timed bisimilar.
Because $N$ is sound in the discrete semantics, we know that
there is some final marking $M_\mathit{out}$ such that
$M_\mathit{out} \in [M'_1\rangle$. As $M_1$ is timed bisimilar with
$M'_1$, it can also reach a final marking (bisimilar to $M_\mathit{out}$)
and so does $M$. Hence condition $a)$ of Definition~\ref{def:soundness}
is established and we can conclude that
$N$ is sound also in the continuous semantics.
\end{proof}


\begin{figure}[!t]
%\subfloat[][Using invariants]{
\begin{minipage}[c]{0.5\textwidth}
\begin{tikzpicture}[font=\scriptsize,xscale=1.9,yscale=1.6]
  \tikzstyle{arc}=[->,>=stealth,thick]
	\tikzstyle{transportArc}=[->,>=diamond,thick]
  \tikzstyle{every place}=[minimum size=6mm,thick]
  \tikzstyle{every transition}=[fill=black,minimum width=2mm,minimum height=5mm]
  \tikzstyle{every token}=[fill=white,text=black]
  \tikzstyle{urgency}=[place,fill=white,minimum size=2.0mm,thin]
%        \node[] at (0,-0.8) (space) { };
	\tikzstyle{inhibArc}=[->,>=o,thick]
\node[place,label=above:$\itn{in}$,structured tokens={0},] at (0.6,0) (in) {};
\node[place,label=above:$\itn{out}$,] at (3.25,0) (out) {};
\node[place,label=above:$\itn{waiting}$,] at (1.2,1) (waiting) {};
\node[place,label=left:$\itn{deadline}$,label=below:inv: $\leq  1$,] at (1.2,-0.9) (deadline) {};
\node[place,label=above:$\itn{finished}$,] at (2.25,1) (finished) {};
\node[transition,label=right:$\itn{init}$] at (1.2,0) (init) {};
\node[transition,label=below:$\itn{service}$] at (1.75,1) (service) {};
\node[transition,label=below:$\itn{late}$] at (2.25,0) (late) {};
\node[transition,label=below:$\itn{early}$] at (2.75,-0.9) (early) {};
\draw[arc] (in) -- (init) node[midway,auto] {$$} {};
\draw[arc] (init) -- (waiting)  {};
\draw[arc] (init) -- (deadline)  {};
\draw[arc] (waiting) -- (service) node[midway,auto] {$$} {};
\draw[arc] (service) -- (finished)  {};
\draw[arc] (finished) -- (late) node[midway,auto] {$[0,0]$} {};
\draw[arc] (deadline) -- (late) node[midway,auto] {$$} {};
\draw[arc] (late) -- (out)  {};
\draw[arc] (deadline) -- (early) node[midway,auto] {$$} {};
\draw[arc] (finished) -- (2.75,1) -- (early) node[very near start,auto] {$[1,1]$} {};
\draw[arc] (early) -- (3.25,-0.9) -- (out)  {};
\end{tikzpicture}
\captionsetup{type=subfigure}
\caption{Using invariants.}
\label{fig:count1}
\end{minipage}
\hspace{0mm}
%\subfloat[][Using urgency]{
\begin{minipage}[c]{0.5\textwidth}
\begin{tikzpicture}[font=\scriptsize,xscale=1.9,yscale=1.6]
  \tikzstyle{arc}=[->,>=stealth,thick]
	\tikzstyle{transportArc}=[->,>=diamond,thick]
  \tikzstyle{every place}=[minimum size=6mm,thick]
  \tikzstyle{every transition}=[fill=black,minimum width=2mm,minimum height=5mm]
  \tikzstyle{every token}=[fill=white,text=black]
  \tikzstyle{urgency}=[place,fill=white,minimum size=2.0mm,thin]
	\tikzstyle{inhibArc}=[->,>=o,thick]
	\tikzstyle{urgency}=[place,fill=white,minimum size=2.0mm,thin]
 \node at (0,0) {};
\node[place,label=above:$\itn{in}$,structured tokens={0},] at (0,1) (in) {};
\node[place,label=below:$\itn{idle}$,] at (1.25,1) (P1) {};
\node[place,label=below:$\itn{out}$,] at (2.4,1) (out) {};
\node[transition,label=above:$\itn{start}$] at (0.65,1) (start) {};
\node[transition,label=above:$\itn{loop}$] at (1.25,2.55) (loop) {};
\node[urgency] at (1.25,2.55) {};
\node[transition,label=above:$\itn{end1}$] at (1.9,1) (end1) {};
\node[transition,label=above:$\itn{end2}$] at (2.4,2.55) (end2) {};
\draw[arc] (P1) -- (end1) node[midway,below,xshift=-1] {$[0,0]$} {};
\draw[arc] (P1) -- (end2) node[midway,above] {} {};
\draw[arc] (end1) -- (out)  {};
\draw[arc] (end2) -- (out)  {};
\draw[transportArc] (in) -- (start) node[midway,auto] {:1} {};
\draw[transportArc] (start) -- (P1) node[midway,auto] {:1} {};
\draw[transportArc] (P1) to [bend right=25] (loop) node[near start,auto] {};
\draw[transportArc] (loop) to [bend right=25](P1) node[near end,auto] {};
\node [] at (0.9,1.85) {:1};
\node [] at (1.6,2) {:1};
\node [] at (1.94,2.29) {$[1,\infty)$};
\end{tikzpicture}
\vspace{-0.4cm}
\captionsetup{type=subfigure}
\caption{Using urgency.}\label{fig:count2}
\end{minipage}
\caption{Sound nets in the discrete semantics and unsound in the continuous one}
\label{fig:counts}
\end{figure}

%We know by Theorem~\ref{thm:discvscont} that the
%same untimed markings are reachable both in the discrete and continuous
%semantics, however, 
For extended timed-arc 
workflow nets the notion of soundness for the discrete and continuous 
semantics are, perhaps surprisingly, different.
Consider Figure~\ref{fig:counts} where both given workflow nets are sound
in the discrete semantics but not in the continuous one.

%. However, in the continuous semantics we can execute in the net
%from Figure~\ref{fig:count1} the sequence
%``\itn{init}, delay $0.5$, \itn{service}, delay $0.5$'', bringing us
%into a deadlock situation. In the net from Figure~\ref{fig:count2},
%we can perform the sequence ``delay $0.5$, \itn{start}'', ending up
%in a situation where only the urgent transition
%\itn{loop} is enabled. Hence the nets are not sound w.r.t. the continuous semantics.

\begin{theorem} \label{thm:counterex}
There is an ETAWFN (with either age invariants
or urgent transitions) sound in the
discrete semantics but unsound in the continuous one.
\end{theorem}
\begin{proof}
Consider the nets in Figure~\ref{fig:counts} (as before we do not
draw the $[0,\infty]$ intervals). It is easy to verify that
both of them are sound w.r.t. the discrete semantics. Indeed,
in Figure~\ref{fig:count1} the age of the token in the place
\itn{finished} can be either $1$ or $0$, depending on whether the
service was executed early or late, and then either the transition
\itn{early} or \itn{late} will be enabled and allow us to reach a
final marking. Similarly in Figure~\ref{fig:count2} the age of the token
in the place \itn{idle} will be of integer age so even though the transition
\itn{loop} disables any time delay, we can still terminate the workflow
by firing either the transition \itn{end1} or \itn{end2}.

However, in the continuous semantics we can execute in the net
from Figure~\ref{fig:count1} the sequence
``\itn{init}, delay $0.5$, \itn{service}, delay $0.5$'', bringing us
into a deadlock situation. In the net from Figure~\ref{fig:count2},
we can perform the sequence ``delay $0.5$, \itn{start}'', ending up
in a situation where only the urgent transition
\itn{loop} is enabled but at the same time disallows for any time delay.
In both cases the nets are not sound w.r.t. the continuous semantics.
\end{proof}

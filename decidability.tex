%!TEX root=main.tex
%\section{Decidability of Soundness} \label{sec:decidability}

Next we show that soundness for extended timed-arc workflow nets
is undecidable.
% In fact any of the additional 
%features of inhibitor arcs, age invariants and urgent transition is
%enough on its own to make the problem undecidable. 
The result has been 
known for the extension with inhibitor arcs~\cite{AalstHHSVVW11}, we prove it 
also for urgent transitions and age invariants. 


\begin{theorem}\label{thm:undecid}
Soundness is undecidable for extended timed-arc workflow nets.
This is the case also for MTAWFNs that contain additionally only
inhibitor arcs, age invariants or urgent transitions but not necessarily
all of them together.
\end{theorem}

\begin{proof}
The proofs are by reduction from the Minsky machine.
A \emph{Minsky machine} with two nonnegative counters $c_1$ and $c_2$ is
a sequence of labelled instructions
$$1:\mathsf{inst}_1;\ 2:\mathsf{inst}_2;\ \ldots, n:\mathsf{inst}_n$$
where $\mathsf{inst}_n = \mathsf{HALT}$ 
and each $\mathsf{inst}_i$, $1\le i < n$, is of one of the following forms
\begin{itemize}
\item
{(Inc)} \hspace{5mm} {\tt $i$: $c_j$++; goto $k$}
\item
{(Dec)} \hspace{4mm} {\tt $i$: if $c_j$=$0$ then goto $k$ else 
 ($c_j$--; goto $\ell$)}
\end{itemize}
for $j\in \{1,2\}$ and $1\le k,\ell \le n$.

Instructions of type (Inc) are called \emph{increment} instructions and
those of type (Dec) are called \emph{test and decrement} instructions.
A configuration is a triple $(i, v_1, v_2)$ where $i$ is the current
instruction and $v_1$ and $v_2$ are the values of the counters $c_1$
and $c_2$, respectively. A computation step between configurations is
defined in the natural way. If starting from the initial
configuration $(1,0,0)$ the machine reaches the instruction $\mathsf{HALT}$
then we say it \emph{halts}, otherwise it \emph{loops}.
%
The problem whether a given Minsky machine halts 
is undecidable~\cite{Minsky:book}. W.l.o.g. we assume
that the machine halts only when both counters are empty
(we can add a few instructions that will always empty the counters
before reaching the halting instruction).

We shall now reduce reachability in Minsky machines into the soundness
problem on ETAWFN. Counters $c_1$ and $c_2$ will be simulated
by two places $p_{c_1}$ and $p_{c_2}$ such that the number of
tokens in those places represents the value of the counters.
For every instruction label $i$, $1 \leq i \leq n$, we
add a new control place $p_i$. At any moment exactly one of the $p_i$
places will be marked by a token, representing the instruction
to be executed in the next step.

If we allow urgent transitions, we can create for any given Minsky machine
a workflow net constructed according to the patterns given in Figure~\ref{fig:urgent-reduction} (we only show
the encoding of the instructions that manipulate the first counter;
the encoding for the second counter is completely analogous).
We also postulate that the input place is $in=p_1$ and the output place is
$out=p_n$. Now, given the initial marking with one token in $p_1$,
the net will faithfully simulate the (deterministic) computation of
the Minsky machine. This is clear for the increment instruction
as the control token moves from $p_i$ to $p_k$ and the number of tokens
in $p_{c_1}$ is increased by one. For the test and decrement instruction,
if $p_{c_1}$ contains at least one token then the transition
$t_i^{\mathit{dec}}$ will be fired with no delay (the transition is urgent),
decreasing the counter by one and moving the control token to $p_\ell$
as required. Only if the counter $c_1$ is empty (there are no tokens
in $p_{c_1}$), we are allowed to delay one time unit and fire the transition
$t_i^{\mathit{zero}}$ such that the control token is moved to $p_k$.
Hence the test and decrement instruction is also faithfully simulated
and there is no possibility of any deadlock situation, meaning that
either $t_i^{\mathit{dec}}$ or $t_i^{\mathit{zero}}$ can always fire.
It is now an easy observation that the workflow net is sound if and only
if the Minsky machine halts.

%A very similar reduction can be done for workflow nets that can, instead
%of urgent transitions, employ inhibitor arcs. In order to guarantee
%that the simulation is fair, we simply add an inhibitor arc from the
%counter place $p_{c_1}$ to the transition $t_i^{\mathit{zero}}$.
%Otherwise the net is completely untimed and again, the workflow net is
%sound if and only if the Minsky machine halts.

The reduction for workflow nets that contain only age invariants
is more complicated. The reduction idea is based on~\cite{memics_paper}, 
however, it had to be nontrivially modified in order to avoid the
large number of possible deadlocks introduced in the reduction.

The counters are now modelled by two places that contain age invariants
ensuring that no tokens can get older than $2$, see 
Figure~\ref{fig:register_simulation}. The intuition is that
before and after the simulation of any instruction, all tokens in
the places $p_{c_1}$ and $p_{c_2}$ are exclusively of age $1$. 
As before the input place is $\itn{in}=p_1$ and the output place
is $\itn{out}=p_n$.

\begin{figure}[!h]
  
\begin{minipage}[c]{0.5\textwidth}
\hspace{1.5cm}
\begin{tikzpicture}[font=\scriptsize,xscale=1.5,yscale=1.5]
   \tikzstyle{arc}=[->,>=stealth,thick]
   \tikzstyle{transportArc}=[->,>=diamond,thick]
   \tikzstyle{every place}=[minimum size=6mm,thick]
   \tikzstyle{every transition}=[fill=black,minimum width=2mm,minimum height=5mm]
   \tikzstyle{every token}=[fill=white,text=black]
  \node[place,label=above:$p_{i}$] at (0,0) (pi) {};
  \node[place,label=above:$p_{k}$] at (2,0) (pk) {};
  \node[place,label=above:$p_{c_1}$] at (1,1) (pc1) {};
  \node[transition,label=60:$t_i$] at (1,0) (t) {};
  \draw[arc] (pi) -- (t) node[midway,above]{$[0,\infty]$} {};
  \draw[arc] (t) -- (pk) {};
  \draw[arc] (t) -- (pc1) {};
\end{tikzpicture}
\label{fig:inc}
\captionsetup{type=subfigure}
\caption{\tt $i$: $c_1$++; goto $k$}
\end{minipage}
%\hspace{0mm}
\begin{minipage}[c]{0.5\textwidth}
\vspace{0.5cm}
\begin{tikzpicture}[font=\scriptsize,xscale=1.9,yscale=1.5]
  \tikzstyle{arc}=[->,>=stealth,thick]
  \tikzstyle{transportArc}=[->,>=diamond,thick]
  \tikzstyle{every place}=[minimum size=6mm,thick]
  \tikzstyle{every transition}=[fill=black,minimum width=2mm,minimum height=5mm]
  \tikzstyle{every token}=[fill=white,text=black]
  \tikzstyle{urgency}=[place,fill=white,minimum size=2.0mm,thin]
  \node at (-0.85,0) {};
  \node at (2.8,0) {};
  \node[place,label=above:$p_{i}$] at (0,0) (pi) {};
  \node[place,label=above:$p_{k}$] at (2,0) (pk) {};
  \node[place,label=above:$p_{\ell}$] at (2,1) (pl) {};
  \node[place,label=above:$p_{c_1}$] at (0,1) (pc1) {};
  \node[transition,label=60:$t_i^{\mathit{dec}}$] at (1,1) (tdec) {};
  \node[urgency] at (1,1) {};
  \node[transition,label=60:$t_i^{\mathit{zero}}$] at (1,0) (tzero1) {};
 \draw[arc] (pi) -- (tdec) node[midway,left]{$[0,\infty]$} {};
 \draw[arc] (pi) -- (tzero1) node[midway,above]{$[1,\infty]$} {};
 \draw[arc] (pc1) -- (tdec) node[midway,above,pos=0.5]{$[0,\infty]$} {}; 
 \draw[arc] (tdec) -- (pl) {};
 \draw[arc] (tzero1) -- (pk) {};
\end{tikzpicture}
\vspace{-0.4cm}
\label{fig:dec}
\captionsetup{type=subfigure}
\caption{{\tt $i$: if $c_1$=$0$ then goto $k$ else ($c_1$--; goto $\ell$)}}
\end{minipage}
\caption{Simulation of (Inc) and (Dec) instructions with urgent transitions\label{fig:urgent-reduction}}
\end{figure}

Let us now observe
that this invariant is preserved after simulating the 
increment instruction (Figure~\ref{fig:increment_simulation}).
Assume that all tokens in the counter places are of age $1$ and that
the place $p_i$ contains one token of age $0$.
%(initially created by the initialization transition in Figure~\ref{fig:start}
%that just resets the token age to $0$). 
Before $t_i$ can be fired, one time unit must pass and this guarantees
that all tokens in the counter places will become of age $2$. After firing
of $t_i$, we also add one token of age $0$ to $p_{c_1}$ 
and moreover, a token of age $0$ in the place $q_i$ is created.
Before we can proceed and delay one time unit 
and then fire $t_i^{\mathit goto}$, we must fire the
transitions $t_{c_1}^{\mathit reset}$ and $t_{c_2}^{\mathit reset}$ once
for every token of age $2$ in the counter places in order to reset 
them all to the age $0$, otherwise the age invariant $\leq 2$ in the token 
places disables the delay of one time unit. Clearly, after the transition 
$t_i^{\mathit goto}$ is fired, all counter tokens are again of age $1$
(including the one added to $p_{c_1}$) and we argued for a faithful simulation
of the increment instruction.

\begin{figure}[h]
\captionsetup[figure]{position=above}
\caption{Simulation of a Minsky machine by a workflow net with invariants.\label{fig:cont}}
\begin{center}
\begin{minipage}[c]{\textwidth}
\centering
\begin{tikzpicture}[font=\scriptsize, xscale=.9,yscale=1.6]
   \tikzstyle{arc}=[->,>=stealth,thick]
   \tikzstyle{transportArc}=[->,>=diamond,thick]
   \tikzstyle{every place}=[minimum size=6mm,thick]
   \tikzstyle{every transition}=[fill=black,minimum width=2mm,minimum height=5mm]
   \tikzstyle{every token}=[fill=white,text=black]
  \node[place,label=above:$p_{c_1}$,label=270:{inv: $\leq 2$}] at (0,0) (p1) {};
  \node[place,label=above:$p^{\mathit{reset}}_{c_1}$](p2) at (4,0) {};
  \node[transition] at (2,0)[label=above:$t^{\mathit{reset}}_{c_1}$] {} 
    edge [pre,bend right=10]  node[above,xshift=0mm,pos=0.5]{$[2,2]$} (p1) 
    edge [post,bend right=10] (p2)
    edge [pre,bend left=10] node[above,xshift=0mm,pos=0.5]{$[0,\infty]$}(p2) 
    edge [post,bend left=10] (p1);

 \node[place,label=above:$p_{c_2}$,label=270:{inv: $\leq 2$}] at (6.5,0) (p1c) {};
  \node[place,label=above:$p^{\mathit{reset}}_{c_2}$](p2c) at (10.5,0) {};
  \node[transition] at (8.5,0)[label=above:$t^{\mathit{reset}}_{c_1}$] {}
    edge [pre,bend right=10]  node[above,xshift=0mm,pos=0.5]{$[2,2]$} (p1c)
    edge [post,bend right=10] (p2c)
    edge [pre,bend left=10] node[above,xshift=0mm,pos=0.5]{$[0,\infty]$}(p2c)
    edge [post,bend left=10] (p1c);
\end{tikzpicture}
    \captionof{subfigure}{Simulation of the counters $c_1$ and $c_2$.\label{fig:register_simulation}}
  \end{minipage}

  \begin{minipage}[c]{\textwidth}
%    \input{figures/simu_increment}
\centering
\begin{tikzpicture}[font=\scriptsize,scale=.9]
   \tikzstyle{arc}=[->,>=stealth,thick]
   \tikzstyle{transportArc}=[->,>=diamond,thick]
   \tikzstyle{every place}=[minimum size=6mm,thick]
   \tikzstyle{every transition}=[fill=black,minimum width=2mm,minimum height=5mm]
   \tikzstyle{every token}=[fill=white,text=black]
  \node[place,label=above:$p_{i}$,label=below:{inv: $\leq 1$}] at (0,0) (pj) {};
  \node[place,label=-90:$p^{\mathit{reset}}_{c_1}$](presetri) at (5,-1.4) {};
  \node[place,label=below:{inv: $\leq 1$},label={[label distance=-0mm]90:$q_{i}$}](q1) at (5,0) {};
  \node[place,label=90:$p^{\mathit{reset}}_{c_2}$](presetr3i) at (5,1.4) {};
  \node[place,label=above:$p_{c_1}$](pri) at (0,1.4) {};
  \node[place,label=above:$p_k$,label=below:{inv: $\leq 1$}](pk) at (10,0) {};

  \node[transition] at (2.5,0)[label=above:$t_i$] {}
    edge [pre] node[above,pos=0.6]{$[1,1]$} (pj) 
    edge [post] (q1) 
    edge [post, bend right=20] (pri) 
    edge [post, bend right] (presetri) 
    edge [post, bend left] (presetr3i);
  \node[transition] at (7.5,0)[label={[yshift=1mm,xshift=1mm]above:$t^{\mathit{goto}}_i$}] {}
    edge [pre, bend left] node[above,sloped,pos=0.6] {$[0,\infty]$} (presetri) 
    edge [pre] node[above,sloped,pos=0.55] {$[1,1]$} (q1) 
    edge [pre, bend right] node[above,sloped,pos=0.7] {$[0,\infty]$} (presetr3i) 
    edge [post] (pk) {};
\end{tikzpicture}
    \captionof{subfigure}{{\tt $i$: $c_1$++; goto $k$}\label{fig:increment_simulation}}
      
  \end{minipage} 
  \begin{minipage}[c]{\textwidth}
    %\input{figures/simu_test_decrement}
\centering
\begin{tikzpicture}[font=\scriptsize,xscale=0.9,yscale=1.5]
   \tikzstyle{arc}=[->,>=stealth,thick]
   \tikzstyle{transportArc}=[->,>=diamond,thick]
   \tikzstyle{every place}=[minimum size=6mm,thick]
   \tikzstyle{every transition}=[fill=black,minimum width=2mm,minimum height=5mm]
   \tikzstyle{every token}=[fill=white,text=black]

  \node[place,label=above:$p_{i}$,label=below:{inv: $\leq 1$}] at (0,1) (pj) {};
  \node[place,label=above:$p_{c_1}$,label=220:](pri) at (0,-1.5) {};
  \node[place,label=below:{inv: $\leq 1$},label={[label distance=-1mm]60:$q_{i}$}](q1) at (5,0.3) {};
  \node[place,label=below:{inv: $\leq 1$},label={[label distance=-1mm]60:$q'_{i}$}](q1p) at (5,-0.7) {};
  \node[place,label=above:$p^{\mathit{reset}}_{c_{2}}$](presetr3i) at (5,1) {};  
  \node[place,label=below:$p^{\mathit{reset}}_{c_{1}}$](prs) at (5,-1.5) {};  
  \node[place,label=right:$p_k$,label=below:{inv: $\leq 1$}](pl) at (10,1) {};
  \node[place,label=right:$p_\ell$,label=below:{inv: $\leq 1$}](pk) at (10,-1.5) {};

  \node[transition, label=above:$t_i$] at (2.5,1) {}
    edge [pre] node[auto,swap,yshift=0mm]{$[1,1]$} (pj)
    edge [post,bend right=15] (q1)
    edge [post] (presetr3i);
  \node[transition, label=above:$t^{\mathit{zero}}_i$] at (7.5,1) {}
    edge [pre, bend left=15] node[right,pos=0.4,xshift=0.9mm]{$[1,1]$} (q1)
    edge [pre] node[auto,swap,xshift=-0mm,yshift=-.0mm]{$[0,\infty]$} (presetr3i)
    edge [post] (pl);
  \node[transition, label=below:$t^{\mathit{dec}}_i$] at (2.5,-1.5) {}
    edge [pre] node[above,yshift=.0mm]{$[2,2]$} (pri)
    edge [pre,bend left=22] node[auto,yshift=-1.4mm]{$[0,0]$} (q1)
    edge [post,bend left=15] (q1p)
    edge [post] (prs);
\node[transition, label=below:$t^{\mathit{end}}_i$] at (7.5,-1.5) {}
    edge [pre,bend left=0] node[below,yshift=.0mm]{$[0,\infty]$} (prs)
    edge [pre,bend right=15] node[above,yshift=0.5mm]{$[1,1]$} (q1p)
    edge [pre,bend right=30] node[right,yshift=.0mm,pos=0.3]{$[0,\infty]$} (presetr3i)
    edge [post] (pk);
\end{tikzpicture}
      \captionof{subfigure}{{\tt $i$: if $c_1$=$0$ then goto $k$ else 
           ($c_1$--; goto $\ell$)}.\label{fig:test_decrement_simulation}}
  \end{minipage}
\end{center}
\end{figure}

Let us consider now the test and decrement instruction simulated by
the net in Figure~\ref{fig:test_decrement_simulation}. Again, let us assume
that all counter tokens are of age $1$ and that there is one token
of age $0$ in $p_i$. First we wait one time unit and then fire $t_i$,
meaning again that all counter tokens are of age $2$. Now all
tokens in the place $p_{c_2}$ can be reset to $0$ and if $p_{c_1}$
does not contain any tokens, we can wait one time unit and fire
$t_i^{\mathit zero}$, implying that we place a token to $p_k$
as expected and all counter tokens are again of age $1$.
If on the other hand $p_{c_1}$ contains some tokens (all of age $2$
as we already mentioned), we must without any delay fire
$t_i^{\mathit dec}$ consuming one token from $p_{c_1}$ while marking
the places $q'_i$ and $p^{\mathit reset}_{c_1}$, allowing
now all counter tokens to be reset to $0$. After this we can wait
one time unit and fire the transition $t_i^{\mathit end}$, while
continuing with the execution of the instruction with label $\ell$.
All tokens in the counter places are now again of age $1$.
Notice that these two scenarios are deterministically determined
by the presence or absence of tokens in $p_{c_1}$ and that there are
no deadlock situations possible during the simulation.

As a result
%, and assuming that the output place of the workflow net is $out = p_n$, 
we can see that the net is sound if and only if
the Minsky machine halts. This completes the undecidability proof
also for the situation where we use only age invariants.
\end{proof}

\begin{algorithm}[th!]
\caption{Soundness checking for timed-arc workflow nets\label{algthm:1}}
%\dontprintsemicolon
\SetKwInOut{Input}{Input}
\SetKwInOut{Output}{Output}

\Input{An MTAWFN or an ETAWFN with a positive integer bound $k$
       $N=\tapntuple$ where $\mathit{in}, \mathit{out} \in P$.}
\Output{``Not $k$-bounded'' if the workflow net is not monotonic 
         and not $k$-bounded;
        ``true'' and the minimum execution time if $N$ is sound; 
         ``false'' if $N$ is not sound.}
\Begin{
A marking $M$ has an (initially empty) set of its parents
$M.\mathit{parents}$ and a minimum execution time $M.min$ (initially $\infty$); \label{initially}
%$\mathit{Waiting}:= \{ cut(M) \mid \text{ $M$ is a marking s.t. }
% |M|=1 \text{ and } |M(\mathit{in})|=1 \}$\; 
$M_\mathit{in}:=\{(\mathit{in},0)\}$\;
$\mathit{Waiting}:= \{M_\mathit{in}\}$; $M_\mathit{in}.\mathit{min}=0$;
\label{waitinit1} \label{waitinit2}
% \textbf{foreach} $M \in \mathit{Waiting}$ \textbf{do} $M.min = 0$\; 
$\mathit{Reached}:=\mathit{Waiting}$; \label{Reachedinit}
$\mathit{Final}:= \emptyset$\; \label{Finalinit}
\While{$\mathit{Waiting} \neq \emptyset$ \label{initWhile}}{ 
Remove some marking $M$ from $\mathit{Waiting}$ with the smallest $M.min$\; \label{takeminmarking}
\ForEach{$M^\prime \mbox{s.t.}\, M \trans{1} M' \mbox{ or }
 M \trans{t} M' \mbox{ for some } t \in T$ \label{foreach}}{
\textbf{if} {$N$ is not monotonic and $|M'|>k$} \textbf{then} {\Return ``Not $k$-bounded''\; \label{monoandnotkbounded}} 
$M'_c:=cut(M^\prime)$;  \label{cut}
$M'_{c}.\mathit{parents}:=M'_{c}.\mathit{parents} \cup \{M\}$\; \label{addtoparents}
\textbf{if} $M \trans{1} M'$ \textbf{then} $M'_c.min := \mathtt{MIN}(M'_c.min, M.min+1)$\label{updateMin1}\;
%\hspace{18mm}
\textbf{else} $M'_c.min = \mathtt{MIN}(M'_c.min, M.min)$\label{updateMin2}\;
\eIf{$|M'_c(out)| \geq 1$}
{ \textbf{if} {$M'_c \text{ is a final marking}$} \textbf{then}
 {$\mathit{Final}:=\mathit{Final} \cup \{M'_c\}$\;\label{addtoFinal}} 
 %\hspace{35mm}
\textbf{else} {\Return false\;\label{notproper}} 
}
{
\If{$M'_c  \notin \mathit{Reached}$\label{notinReached}} 
{
\textbf{if} {$M'_c$ is a deadlock \label{isDeadlock}} \textbf{then} {\Return false\label{deadlock}\;}
\If{$N$ is monotonic and $\exists M'' \in \mathit{Reached}.\ M'' \sqsubseteq_{cut} M'_c$ \label{ifmonoandunbounded}}
{\Return false\;\label{monoandunbounded}} 
$\mathit{Reached}:= \mathit{Reached} \cup \{M'_c\}$; \label{addtoReached} $\mathit{Waiting}:= \mathit{Waiting} \cup \{M'_c\}$\; \label{addtoWaiting}
}
}
}
}
$\mathit{Waiting}:= \mathit{Final}$\; \label{endphase1}
\While{$\mathit{Waiting} \neq \emptyset$}{ \label{initphase2}
Remove some marking $M$ from $\mathit{Waiting}$\;
$\mathit{Waiting}:= \mathit{Waiting} \cup (M.\mathit{parents} \cap \mathit{Reached})$\;
$\mathit{Reached}:= \mathit{Reached} \smallsetminus M.\mathit{parents}$\;
\label{endwhile2}} 
\eIf{$\mathit{Reached} = \emptyset$}{
$\mathit{time} := \infty$; \textbf{foreach} $M \in \mathit{Final}$ \textbf{do} $\mathit{time} = \mathtt{MIN}(\mathit{time}, M.min)$\; \label{minexectime}
\Return true and $\mathit{time}$\label{returnTrue}\;\label{Nissound}} 
{\Return false;\label{endphase2}} 
}
\end{algorithm}

We now prove decidability of soundness for workflow nets without 
any inhibitor arcs, age invariants and urgency. This contrasts to
undecidability of reachability 
for this subclass~\cite{RGE:reachability}. 
%Ruiz:1999:NRT:826035.826838}.
Algorithm~\ref{algthm:1} shows how to efficiently check soundness
on this subclass and on the subclass of bounded nets.
The correctness proof is presented next. We will refer 
to phase 1 (lines \ref{initially}-\ref{endphase1}) 
and phase 2 (lines \ref{initphase2}-\ref{endphase2}) of Algorithm \ref{algthm:1}.

\begin{lemma}[Termination]
	\label{algthm:1:termination}
	Algorithm \ref{algthm:1} terminates on any legal input.
\end{lemma}

\begin{proof}%[Lemma \ref{algthm:1:termination}]
	 There is one while-loop in each phase of the algorithm. The loop in the first phase is executed as long as \itn{Waiting} is not empty. 
	We notice that initially \itn{Waiting} and \itn{Reached} are initialised to the same value.
	For each iteration of the loop, we remove a marking from \itn{Waiting} 
        and newly discovered cut markings are always added to both 
        \itn{Waiting} and \itn{Reached}
        (line \ref{addtoReached}) but only if they 
        are not already in \itn{Reached} (line \ref{notinReached}).
        Markings are never removed from the set \itn{Reached} and
        each canonical marking can appear in \itn{Waiting} at most once.

	For non-monotonic nets, only canonical markings with at most
        $k$ tokens are added (line~\ref{monoandnotkbounded}) 
        and therefore the set of canonical markings is finite. 
        Thus the algorithm will terminate as the set \itn{Waiting} 
        eventually becomes empty.

        For monotonic nets, the net
        could be unbounded and, therefore, the set \itn{Reached} would grow above any
        bound. In this case, we know by similar arguments 
        like in the proof of Theorem~\ref{thm:soundness} that must 
        exist $M$, $M^\prime \in [ M_{in}\rangle$ such that 
        \cutsqrel{M}{M^\prime} and $|M| < |M^\prime|$. However,
        the algorithm will detect such a situation 
        at line~\ref{ifmonoandunbounded} and terminate.

	For the loop in the second phase, 
        notice that $\mathit{Waiting} = \mathit{Final}$. 
	For each iteration, a marking is removed from \itn{Waiting} and 
        the intersection of the set of parents of the marking $M$, 
        \emph{M.parents} and the set \itn{Reached} is then added to 
        \itn{Waiting}. In addition, the set \emph{M.parents} is removed 
        from \itn{Reached}. Thus, any marking can only be added to 
        \itn{Waiting} once, and as the set \itn{Reached}
        is finite when entering the loop and a marking is removed from 
        \itn{Waiting} in each iteration, 
        eventually $\itn{Waiting} = \emptyset$ and the algorithm terminates.	
\qed
\end{proof}



\begin{lemma}[Invariant]
	\label{lemma:invariants}
	The while-loop in lines \ref{initWhile}-\ref{addtoWaiting} of Algorithm \ref{algthm:1} satisfies the following loop-invariants:
	\begin{description}
		\item[a)] $\itn{Waiting} \subseteq \itn{Reached}$, 
		\item[b)] for all marking $M^\prime_c \in \itn{Reached} \cup \itn{Final}$, there exists a computation of the net $M_{in} \rightarrow^* M^\prime$ such that $M^\prime_c=cut(M^\prime)$ and the accumulated delay
on the computation $M_{in} \rightarrow^* M^\prime$ 
is equal to $M'_c.\itn{min}$, and
		\item[c)] for any marking $M'_c \in \itn{Reached} \cup \itn{Final}$ and any $M \in M'_c.\itn{parents}$ there is a transition 
   $M \rightarrow M'$ such that $M^\prime_c=cut(M^\prime)$.
		%\item[c)] For all reachable marking $M^\prime$ in $N$, there exists a one-step computation $M \rightarrow M^\prime$ such that $M \in \itn{Reached}$.
	\end{description}
\end{lemma}

\begin{proof}%[Lemma \ref{algthm:1:invariant}]
	The claims a), b) and c) of the invariant are trivially satisfied the first time the while-loop is entered. 
	Let us assume the invariant holds before the execution of the body of the while-loop. 
	
	The claim a) is easily proved, as markings are only added to \emph{Waiting} in line \ref{addtoWaiting} of the loop body, 
	and the same marking is also added to \emph{Reached} in line \ref{addtoReached} and no markings are removed from \itn{Reached} in the body of the loop.
	
	For claim b) notice that in line~\ref{takeminmarking} we remove 
        a marking $M$ from \itn{Waiting} and for any successor $M^\prime$ of 
        $M$, the marking $M$ is added to the set of parents of 
        $M^\prime_c = \cut(M^\prime)$ (line~\ref{addtoparents}). 
	Due to the first invariant, $M$ was already in \itn{Reached} 
        in the previous iteration of the loop.
	Hence there exists a computation $M_{in} \rightarrow^* M_1$
        such that $M = \cut(M_1)$ and the accumulated delay of this
        computation is $M.\itn{min}$.
        Because $M \rightarrow M^\prime$ (line~\ref{foreach}) then 
        also $M_1 \rightarrow M_2$ 
	such that $M^\prime_c=cut(M_2)$ (line \ref{cut}).
        Hence $M_\itn{in} \rightarrow^* M_2$  and $M^\prime_c=cut(M_2)$
        as required. The accumulated delay is updated according to the
        type of the transition $M_1 \rightarrow M_2$ at
        lines~\ref{updateMin1} and~\ref{updateMin2}. If the value
        $M'_c.\itn{min}$ changed after this update then the computation
        $M_\itn{in} \rightarrow^* M_2$ achieves this accumulated delay,
        otherwise the minimum delay was achieved in some previous
        run of the body of the while-loop and it is hence valid due 
        to the loop invariant.  
	%Finally, this marking $M^\prime_c$ is added to the set of 
        %final markings (line~\ref{addtoFinal}) or to the set of reached 
        %markings (line \ref{addtoReached}).	

Finally, the claim c) follows from the fact that markings
to $M'_c.\itn{parents}$ are only added at line~\ref{addtoparents} and
such markings clearly satisfy the invariant claim.
\qed
\end{proof}

\begin{lemma}[Not k-bounded]
	\label{algthm:1:notkbounded}
	Let $N$ be a MTAWFN or an ETAWFN and $k > 0$. If Algorithm \ref{algthm:1} returns ``Not k-bounded'' then  $N$ is not k-bounded. 
\end{lemma}

\begin{proof}%[Lemma \ref{algthm:1:soundness}]
The algorithm returns ``Not k-bounded''  only at line~\ref{monoandnotkbounded},
provided that the net is not monotonic and there is a 
  marking $M^\prime$ reachable from $M \in \itn{Waiting}$ such
	that $|M^\prime| > k$. 
	By claim b) of Lemma~\ref{lemma:invariants}, we know 
	that there is a computation from $M_{in}$ to $M_1$
    such that  $M=\cut(M_1)$ and we also know that 
		$M \rightarrow M'$ (line~\ref{foreach}). 
              Therefore also $M_1 \rightarrow M_2$ and $M_2$ 
         is reachable from $M_{in}$ and at the same time $|M_2| > k$
    as cut preservers the number of tokens in a marking.
     Hence if the algorithm returns ``Not k-bounded'' then 
     the net is not $k$-bounded.
\qed
\end{proof}

\begin{lemma}[Cut markings]
	\label{alg1:reachcutmarkings}
	After the first while loop (lines \ref{initWhile}-\ref{addtoWaiting}) of Algorithm~\ref{algthm:1} is finished, we have at line~\ref{endphase1} 
that $\itn{Reached} \cup \itn{Final}=\{cut(M^\prime) \mid M_\itn{in} \rightarrow^* M^\prime\}$. Moreover, if $M_\itn{in} \rightarrow^* M'$ then the accumulated delay
of this computation is greater or equal to $\cut(M').\itn{min}$ and there
is at least one such a computation where the accumulated delay is
equal to $\cut(M').\itn{min}$. 
\end{lemma}
\begin{proof}%[Lemma \ref{algthm:1:invariant}] 
Let us first argue for the fact  $\itn{Reached} \cup \itn{Final}=\{cut(M^\prime) \mid M_\itn{in} \rightarrow^* M^\prime\}$.
``$\subseteq$'': follows directly from the proof of claim b) 
of Lemma~\ref{lemma:invariants}. ``$\supseteq$'': follows from the fact
that we search all possible successors of $M_\itn{in}$; we do not provide 
further arguments as this is the standard graph searching algorithm.
The optimality of the computation of the minimum delay follows from the fact
that we explore the graph from the nodes with the smallest \itn{min} value
(line~\ref{takeminmarking}) and this is (up to the cut-equivalence) essentially 
the Dijkstra's algorithm for shortest path in the graph. The fact
that $\cut(M').\itn{min}$ can be realized follows from
claim b) of Lemma~\ref{lemma:invariants}.
\qed
\end{proof}

\begin{lemma}[Return value false]\label{algthm:1:false}
	Let $N$ be a MTAWFN or ETAWFN and $k > 0$. If Algorithm \ref{algthm:1} returns false then $N$ is not sound. 
\end{lemma}
\begin{proof}%[Lemma \ref{algthm:1:soundness}]
	Reading the algorithm, one can notice that it returns false in four lines (lines \ref{notproper}, \ref{deadlock}, \ref{monoandunbounded} and \ref{endphase2}). 
	Therefore, we have to demonstrate that the net is not sound in any
of those cases.
	\begin{itemize}
	\item Starting with line \ref{notproper}, the algorithm returns false if it finds a marking $M^\prime$ 
	from the initial marking of $N$ such that $M^\prime_c=cut(M^\prime)$ and  $M^\prime_c$ has at least one token in the output place \itn{out} while
  it is not a final marking (contains some additional tokens in other places 
  too). Clearly, it is possible to reach from $M_\itn{in}$ 
 this marking (up to cut-equivalence) by claim b) of 
 Lemma~\ref{lemma:invariants} and it breaks condition b) of the definition
 of soundness (Definition~\ref{def:soundness}). Therefore the net is not sound. 

  \item In line \ref{deadlock}, the algorithm returns false if we have found a marking  $M^\prime$ reachable from $M_\itn{in}$ (here we use implicitly
claim b) of Lemma~\ref{lemma:invariants})
        such that $M^\prime_c=cut(M^\prime)$
        and  $M^\prime_c$ is a deadlock. As $M'_c$ is a deadlock, 
				$M'$ is also a deadlock (by Theorem~\ref{thm:bisim}).  
							The marking $M'$ is not a final marking and hence 
        breaks condition a) of Definition \ref{def:soundness}.
        Therefore the net is not sound.

	\item Let us continue with line \ref{monoandunbounded}. Here, we have reached a marking  $M^\prime$ 
	from the initial marking of the monotonic net $N$  such that $M^\prime_c=cut(M^\prime)$  
	and there exists a marking $M^{\prime\prime} \in \itn{Reached}$
	such that \cutsqrel{M^{\prime\prime}}{M^\prime_c} and $|M^{\prime\prime}| < |M^{\prime}_c|$. Important to remark here that 
	this situation $|M^{\prime\prime}| = |M^{\prime}_c|$ cannot happen since it is avoided due to the if-condition at line \ref{notinReached}.
	Let us suppose that N is sound. We know due to condition a) of Definition~\ref{def:soundness} that
	there is path from $M^{\prime\prime}$ to the final marking of N ($M_{out}$). However, by applying Lemma~\ref{lem:mono} repeatedly (again we reason up to the cut-equivalence), we can follow the same
	path from $M^\prime_c$ to a marking $M^\prime_{out}$ such that $|M^\prime_c|-|M^{\prime\prime}|=|M^\prime_{out}|-|M_{out}|$. As $|M^{\prime\prime}| < |M^{\prime}_c|$, we also know
	that $|M_{out}| < |M^\prime_{out}|$, which breaks condition b) of Definition \ref{def:soundness}, and contradicts that $N$ is sound.
	
	\item Finally, we take a look to line \ref{endphase2}. Let us recall that \itn{Reached} contains the set of cut markings of the reachable markings 
	that are not final (Lemma \ref{alg1:reachcutmarkings}). Moreover, in the second while-loop, we 
	remove from \itn{Reached} the parents of all the cut markings in \itn{Waiting} until \itn{Waiting} is empty, running a backward search from the set
\itn{Final}. Thus, if \itn{Reached} is not empty after this backward search 
terminates means that there is $M^\prime_c \in \itn{Reached}$ 
such that $M^\prime_c=cut(M^\prime)$ for some reachable marking
$M'$ from $M_\itn{in}$ and there is no path from $M^\prime$ to any
final marking. This breaks condition a) of Definition \ref{def:soundness} and
the net is not sound.\qed
	\end{itemize}	
\end{proof}

\begin{lemma}[Return value true]\label{algthm:1:true}
	Let $N$ be a MTAWFN or ETAWFN and $k > 0$. If Algorithm \ref{algthm:1} returns true then $N$ is sound and the return value is the minimum execution
time of $N$.
\end{lemma}
\begin{proof}
Assume that the algorithm returned true at line~\ref{returnTrue} 
and we shall argue
that $N$ satisfies conditions a) and b) of Definition \ref{def:soundness}.
Condition b) is straightforward as by Lemma~\ref{alg1:reachcutmarkings}
we know that $\itn{Reached} \cup \itn{Final}$ is the set of cut-markings 
of all the reachable markings of $N$ and if some of them
marks the place \itn{out} then this must be a final marking, otherwise
the algorithm would return false at line~\ref{notproper}.
For condition a) we realise that the set \itn{Final} contains
all final markings reachable from $M_\itn{in}$ and 
in the second-phase of the algorithm we run a backward search
and remove from the reachable state-space all markings that have
a computation leading to one of the final markings. We
return true only if \itn{Reached} is empty, meaning that all reachable markings
have a computation to some final marking. This corresponds to
condition a).
Correctness of the minimum execution time follows from Lemma~\ref{alg1:reachcutmarkings}.
\end{proof}


\begin{theorem}\label{thm:decid}
Soundness is decidable for monotonic timed-arc workflow nets and
for bounded extended timed-arc workflow nets.
\end{theorem}

%\subsection{Minimum execution time} \label{subsec:minmax}
Given a sound ETAWFN $N=\tapntuple$, we can reason about its
execution times (the accumulated time that is used to
move a token from the place $\mathit{in}$ into the place $\mathit{out}$).
Let $M_\itn{in}$ be the initial marking of $N$ and
$\mathcal{F}(N)$ be the set of all final markings of $N$.
Let $\mathcal{T}(N)$ be the set of all execution times by which we can 
get from the initial marking to some final marking. Formally,
$$\mathcal{T}(N)\eqdef 
 \{ \sum_{i=0}^{n-1}{d_i} \mid
M_\itn{in}= M_0 
\trans{d_0,t_0} M_1 \trans{d_1,t_1} M_2 \trans{d_2,t_2} \cdots 
\trans{d_{n-1},t_{n-1}} M_n \in \mathcal{F}(N) \} \ .$$
%M_\itn{in}= M_0 
%\trans{d_0} M'_0 \trans{t_0} M_1 \trans{d_1} M'_1 \trans{t_1} M_2
%\trans{d_2} \cdots \trans{t_{n-1}} M_n \in \mathcal{F}(N) \} \ .$$

The set $\mathcal{T}(N)$ is nonempty for any sound net $N$
and the \emph{minimum execution time} of $N$, defined by $\min \mathcal{T}(N)$,
is computable by Algorithm~\ref{algthm:1}.

\begin{theorem}
Let $N$ be a sound MTAWFN or a sound and bounded ETAWFN.
The minimum execution time of $N$ is computable.
\end{theorem}

Notice that 
the set $\mathcal{T}(N)$ can be infinite for general timed-arc
workflow nets, meaning that the \emph{maximum
execution time} of $N$, given by $\max \mathcal{T}(N)$, is not
always well defined. This issue is discussed in the next section.
%In the next section, we shall introduce
%the notion of strong soundness that will guarantee also the existence
%of the maximum execution time. 

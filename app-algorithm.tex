%!TEX root=main.tex
\section{Proofs Related to Algorithm~\ref{algthm:1}}\label{algthm:1:proofs}

We will refer to phase 1 (lines \ref{initially}-\ref{endphase1}) and phase 2 (lines \ref{initphase2}-\ref{endphase2}) of Algorithm \ref{algthm:1}.

\begin{lemma}[Termination]
	\label{algthm:1:termination}
	Algorithm \ref{algthm:1} terminates on any legal input.
\end{lemma}

\begin{proof}%[Lemma \ref{algthm:1:termination}]
	 There is one while-loop in each phase of the algorithm. The loop in the first phase is executed as long as \itn{Waiting} is not empty. 
	We notice that initially \itn{Waiting} and \itn{Reached} are initialised to the same value.
	For each iteration of the loop, we remove a marking from \itn{Waiting} 
        and newly discovered cut markings are always added to both 
        \itn{Waiting} and \itn{Reached}
        (line \ref{addtoReached}) but only if they 
        are not already in \itn{Reached} (line \ref{notinReached}).
        Markings are never removed from the set \itn{Reached} and
        each canonical marking can appear in \itn{Waiting} at most once.

	For non-monotonic nets, only canonical markings with at most
        $k$ tokens are added (line~\ref{monoandnotkbounded}) 
        and therefore the set of canonical markings is finite. 
        Thus the algorithm will terminate as the set \itn{Waiting} 
        eventually becomes empty.

        For monotonic nets, the net
        could be unbounded and, therefore, the set \itn{Reached} would grow above any
        bound. In this case, we know by similar arguments 
        like in the proof of Theorem~\ref{thm:soundness} that must 
        exist $M$, $M^\prime \in [ M_{in}\rangle$ such that 
        \cutsqrel{M}{M^\prime} and $|M| < |M^\prime|$. However,
        the algorithm will detect such a situation 
        at line~\ref{ifmonoandunbounded} and terminate.

	For the loop in the second phase, 
        notice that $\mathit{Waiting} = \mathit{Final}$. 
	For each iteration, a marking is removed from \itn{Waiting} and 
        the intersection of the set of parents of the marking $M$, 
        \emph{M.parents} and the set \itn{Reached} is then added to 
        \itn{Waiting}. In addition, the set \emph{M.parents} is removed 
        from \itn{Reached}. Thus, any marking can only be added to 
        \itn{Waiting} once, and as the set \itn{Reached}
        is finite when entering the loop and a marking is removed from 
        \itn{Waiting} in each iteration, 
        eventually $\itn{Waiting} = \emptyset$ and the algorithm terminates.	
\qed
\end{proof}



\begin{lemma}[Invariant]
	\label{lemma:invariants}
	The while-loop in lines \ref{initWhile}-\ref{addtoWaiting} of Algorithm \ref{algthm:1} satisfies the following loop-invariants:
	\begin{description}
		\item[a)] $\itn{Waiting} \subseteq \itn{Reached}$, 
		\item[b)] for all marking $M^\prime_c \in \itn{Reached} \cup \itn{Final}$, there exists a computation of the net $M_{in} \rightarrow^* M^\prime$ such that $M^\prime_c=cut(M^\prime)$ and the accumulated delay
on the computation $M_{in} \rightarrow^* M^\prime$ 
is equal to $M'_c.\itn{min}$, and
		\item[c)] for any marking $M'_c \in \itn{Reached} \cup \itn{Final}$ and any $M \in M'_c.\itn{parents}$ there is a transition 
   $M \rightarrow M'$ such that $M^\prime_c=cut(M^\prime)$.
		%\item[c)] For all reachable marking $M^\prime$ in $N$, there exists a one-step computation $M \rightarrow M^\prime$ such that $M \in \itn{Reached}$.
	\end{description}
\end{lemma}

\begin{proof}%[Lemma \ref{algthm:1:invariant}]
	The claims a), b) and c) of the invariant are trivially satisfied the first time the while-loop is entered. 
	Let us assume the invariant holds before the execution of the body of the while-loop. 
	
	The claim a) is easily proved, as markings are only added to \emph{Waiting} in line \ref{addtoWaiting} of the loop body, 
	and the same marking is also added to \emph{Reached} in line \ref{addtoReached} and no markings are removed from \itn{Reached} in the body of the loop.
	
	For claim b) notice that in line~\ref{takeminmarking} we remove 
        a marking $M$ from \itn{Waiting} and for any successor $M^\prime$ of 
        $M$, the marking $M$ is added to the set of parents of 
        $M^\prime_c = \cut(M^\prime)$ (line~\ref{addtoparents}). 
	Due to the first invariant, $M$ was already in \itn{Reached} 
        in the previous iteration of the loop.
	Hence there exists a computation $M_{in} \rightarrow^* M_1$
        such that $M = \cut(M_1)$ and the accumulated delay of this
        computation is $M.\itn{min}$.
        Because $M \rightarrow M^\prime$ (line~\ref{foreach}) then 
        also $M_1 \rightarrow M_2$ 
	such that $M^\prime_c=cut(M_2)$ (line \ref{cut}).
        Hence $M_\itn{in} \rightarrow^* M_2$  and $M^\prime_c=cut(M_2)$
        as required. The accumulated delay is updated according to the
        type of the transition $M_1 \rightarrow M_2$ at
        lines~\ref{updateMin1} and~\ref{updateMin2}. If the value
        $M'_c.\itn{min}$ changed after this update then the computation
        $M_\itn{in} \rightarrow^* M_2$ achieves this accumulated delay,
        otherwise the minimum delay was achieved in some previous
        run of the body of the while-loop and it is hence valid due 
        to the loop invariant.  
	%Finally, this marking $M^\prime_c$ is added to the set of 
        %final markings (line~\ref{addtoFinal}) or to the set of reached 
        %markings (line \ref{addtoReached}).	

Finally, the claim c) follows from the fact that markings
to $M'_c.\itn{parents}$ are only added at line~\ref{addtoparents} and
such markings clearly satisfy the invariant claim.
\qed
\end{proof}

\begin{lemma}[Not k-bounded]
	\label{algthm:1:notkbounded}
	Let $N$ be a MTAWFN or an ETAWFN and $k > 0$. If Algorithm \ref{algthm:1} returns ``Not k-bounded'' then  $N$ is not k-bounded. 
\end{lemma}

\begin{proof}%[Lemma \ref{algthm:1:soundness}]
The algorithm returns ``Not k-bounded''  only at line~\ref{monoandnotkbounded},
provided that the net is not monotonic and there is a 
  marking $M^\prime$ reachable from $M \in \itn{Waiting}$ such
	that $|M^\prime| > k$. 
	By claim b) of Lemma~\ref{lemma:invariants}, we know 
	that there is a computation from $M_{in}$ to $M_1$
    such that  $M=\cut(M_1)$ and we also know that 
		$M \rightarrow M'$ (line~\ref{foreach}). 
              Therefore also $M_1 \rightarrow M_2$ and $M_2$ 
         is reachable from $M_{in}$ and at the same time $|M_2| > k$
    as cut preservers the number of tokens in a marking.
     Hence if the algorithm returns ``Not k-bounded'' then 
     the net is not $k$-bounded.
\qed
\end{proof}

\begin{lemma}[Cut markings]
	\label{alg1:reachcutmarkings}
	After the first while loop (lines \ref{initWhile}-\ref{addtoWaiting}) of Algorithm~\ref{algthm:1} is finished, we have at line~\ref{endphase1} 
that $\itn{Reached} \cup \itn{Final}=\{cut(M^\prime) \mid M_\itn{in} \rightarrow^* M^\prime\}$. Moreover, if $M_\itn{in} \rightarrow^* M'$ then the accumulated delay
of this computation is greater or equal to $\cut(M').\itn{min}$ and there
is at least one such a computation where the accumulated delay is
equal to $\cut(M').\itn{min}$. 
\end{lemma}
\begin{proof}%[Lemma \ref{algthm:1:invariant}] 
Let us first argue for the fact  $\itn{Reached} \cup \itn{Final}=\{cut(M^\prime) \mid M_\itn{in} \rightarrow^* M^\prime\}$.
``$\subseteq$'': follows directly from the proof of claim b) 
of Lemma~\ref{lemma:invariants}. ``$\supseteq$'': follows from the fact
that we search all possible successors of $M_\itn{in}$; we do not provide 
further arguments as this is the standard graph searching algorithm.
The optimality of the computation of the minimum delay follows from the fact
that we explore the graph from the nodes with the smallest \itn{min} value
(line~\ref{takeminmarking}) and this is (up to the cut-equivalence) essentially 
the Dijkstra's algorithm for shortest path in the graph. The fact
that $\cut(M').\itn{min}$ can be realized follows from
claim b) of Lemma~\ref{lemma:invariants}.
\qed
\end{proof}

\begin{lemma}[Return value false]\label{algthm:1:false}
	Let $N$ be a MTAWFN or ETAWFN and $k > 0$. If Algorithm \ref{algthm:1} returns false then $N$ is not sound. 
\end{lemma}
\begin{proof}%[Lemma \ref{algthm:1:soundness}]
	Reading the algorithm, one can notice that it returns false in four lines (lines \ref{notproper}, \ref{deadlock}, \ref{monoandunbounded} and \ref{endphase2}). 
	Therefore, we have to demonstrate that the net is not sound in any
of those cases.
	\begin{itemize}
	\item Starting with line \ref{notproper}, the algorithm returns false if it finds a marking $M^\prime$ 
	from the initial marking of $N$ such that $M^\prime_c=cut(M^\prime)$ and  $M^\prime_c$ has at least one token in the output place \itn{out} while
  it is not a final marking (contains some additional tokens in other places 
  too). Clearly, it is possible to reach from $M_\itn{in}$ 
 this marking (up to cut-equivalence) by claim b) of 
 Lemma~\ref{lemma:invariants} and it breaks condition b) of the definition
 of soundness (Definition~\ref{def:soundness}). Therefore the net is not sound. 

  \item In line \ref{deadlock}, the algorithm returns false if we have found a marking  $M^\prime$ reachable from $M_\itn{in}$ (here we use implicitly
claim b) of Lemma~\ref{lemma:invariants})
        such that $M^\prime_c=cut(M^\prime)$
        and  $M^\prime_c$ is a deadlock. As $M'_c$ is a deadlock, 
				$M'$ is also a deadlock (by Theorem~\ref{thm:bisim}).  
							The marking $M'$ is not a final marking and hence 
        breaks condition a) of Definition \ref{def:soundness}.
        Therefore the net is not sound.

	\item Let us continue with line \ref{monoandunbounded}. Here, we have reached a marking  $M^\prime$ 
	from the initial marking of the monotonic net $N$  such that $M^\prime_c=cut(M^\prime)$  
	and there exists a marking $M^{\prime\prime} \in \itn{Reached}$
	such that \cutsqrel{M^{\prime\prime}}{M^\prime_c} and $|M^{\prime\prime}| < |M^{\prime}_c|$. Important to remark here that 
	this situation $|M^{\prime\prime}| = |M^{\prime}_c|$ cannot happen since it is avoided due to the if-condition at line \ref{notinReached}.
	Let us suppose that N is sound. We know due to condition a) of Definition~\ref{def:soundness} that
	there is path from $M^{\prime\prime}$ to the final marking of N ($M_{out}$). However, by applying Lemma~\ref{lem:mono} repeatedly (again we reason up to the cut-equivalence), we can follow the same
	path from $M^\prime_c$ to a marking $M^\prime_{out}$ such that $|M^\prime_c|-|M^{\prime\prime}|=|M^\prime_{out}|-|M_{out}|$. As $|M^{\prime\prime}| < |M^{\prime}_c|$, we also know
	that $|M_{out}| < |M^\prime_{out}|$, which breaks condition b) of Definition \ref{def:soundness}, and contradicts that $N$ is sound.
	
	\item Finally, we take a look to line \ref{endphase2}. Let us recall that \itn{Reached} contains the set of cut markings of the reachable markings 
	that are not final (Lemma \ref{alg1:reachcutmarkings}). Moreover, in the second while-loop, we 
	remove from \itn{Reached} the parents of all the cut markings in \itn{Waiting} until \itn{Waiting} is empty, running a backward search from the set
\itn{Final}. Thus, if \itn{Reached} is not empty after this backward search 
terminates means that there is $M^\prime_c \in \itn{Reached}$ 
such that $M^\prime_c=cut(M^\prime)$ for some reachable marking
$M'$ from $M_\itn{in}$ and there is no path from $M^\prime$ to any
final marking. This breaks condition a) of Definition \ref{def:soundness} and
the net is not sound.\qed
	\end{itemize}	
\end{proof}

\begin{lemma}[Return value true]\label{algthm:1:true}
	Let $N$ be a MTAWFN or ETAWFN and $k > 0$. If Algorithm \ref{algthm:1} returns true then $N$ is sound and the return value is the minimum execution
time of $N$.
\end{lemma}
\begin{proof}
Assume that the algorithm returned true at line~\ref{returnTrue} 
and we shall argue
that $N$ satisfies conditions a) and b) of Definition \ref{def:soundness}.
Condition b) is straightforward as by Lemma~\ref{alg1:reachcutmarkings}
we know that $\itn{Reached} \cup \itn{Final}$ is the set of cut-markings 
of all the reachable markings of $N$ and if some of them
marks the place \itn{out} then this must be a final marking, otherwise
the algorithm would return false at line~\ref{notproper}.
For condition a) we realise that the set \itn{Final} contains
all final markings reachable from $M_\itn{in}$ and 
in the second-phase of the algorithm we run a backward search
and remove from the reachable state-space all markings that have
a computation leading to one of the final markings. We
return true only if \itn{Reached} is empty, meaning that all reachable markings
have a computation to some final marking. This corresponds to
condition a).
Correctness of the minimum execution time follows from Lemma~\ref{alg1:reachcutmarkings}.
\qed
\end{proof}

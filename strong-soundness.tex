\section{Strong Soundness} \label{sec:strong}
Soundness 
ensures the possibility of correct
termination in a workflow net, however, it does not
give any guarantee on a timely termination of the workflow.
In untimed workflows, infinite behaviour
can be used to model for instance repeated queries for further information
until a decision can be taken. In a time setting,
we usually have a deadline such that if the information
is not acquired within the deadline, alternative behaviour in the net
is executed (compensation).

Consider the workflow nets presented in Figure~\ref{fig:soudnessnets}.
They represent a simple customer-complaint workflow where,
before a decision is made, the customer can be repeatedly requested
to provide additional information.
The net in Figure~\ref{fig:notstrongsoun}
is sound but there is no time guarantee by when
the decision is reached. On the other hand,
the net in Figure~\ref{fig:strongsoundness}
introduces additional timing, 
requiring that the process starts within 5 days and the
request/provide loop takes no more than 14 days, after which a decision
is made. The use of
transport arcs enables us to measure the accumulated time since the
place $\mathit{pending}$ was entered the first time. It is clear that
the workflow only permits behaviours up to 19 days in total.
In fact, the net %in Figure~\ref{fig:strongsoundness}
enables infinite executions never reaching any final marking,
however, this only happens within a bounded time interval 
(producing a so-called \emph{zeno run})
and we postulate that such a scenario is unrealistic in
a real-world workflow execution. After disregarding the zeno runs,
we are guaranteed that the workflow finishes within 19 days
and we can call it \emph{strongly sound}. 
%A formal definition of strong soundness follows.

\begin{definition}[Strong soundness of TAWFN] \label{def:strongsound}
An (extended or monotonic) timed-arc workflow net $N$ %=\tapntuple$ 
is \emph{strongly sound} if 
\begin{description}
\item{a)} $N$ is sound, 
\item{b)} every divergent marking reachable in $N$ is a final marking, and
\item{c)} there is no infinite computation starting from the initial marking \\
$\{(\itn{in},0)\}=M_0 \trans{d_0,t_0} M_1 
\trans{d_1,t_1} M_2 \trans{d_2,t_2} \cdots$ 
where $\sum_{i \in \nnul}d_i = \infty$.
\end{description}
\end{definition}

\begin{figure}[!t]
\begin{minipage}[c]{0.5\textwidth}
\centering
\begin{tikzpicture}[font=\scriptsize,xscale=1.9,yscale=1.1]
  \tikzstyle{arc}=[->,>=stealth,thick]
	\tikzstyle{transportArc}=[->,>=diamond,thick]
  \tikzstyle{every place}=[minimum size=6mm,thick]
  \tikzstyle{every transition}=[fill=black,minimum width=2mm,minimum height=5mm]
  \tikzstyle{every token}=[fill=white,text=black]
  \tikzstyle{urgency}=[place,fill=white,minimum size=2.0mm,thin]
        \node[] at (0,-0.8) (space) { };
	\tikzstyle{inhibArc}=[->,>=o,thick]
  \node at (0,0) {};
	\node[place,label=above:$\mathit{in}$] at (0,0) (in) {};
	\node[place,label=below:\itn{pending}] at (1.3,0) (pending) {};
	\node[place,label=\itn{out}] at (2.65,0) (out) {};
	\node[place,label=right:\itn{sending}] at (1.3,2) (sending) {};
  \node[transition,label=below:\itn{start}] at (0.65,0) (start) {};
	\draw[arc] (in) -- (start) {}; 
	\draw[arc] (start) -- (pending) {}; 
	\node[transition,label=above:\itn{req\_info},rotate=90] at (1.05,1) (request) {};
	\draw[arc] (pending)--(request){};
	\draw[arc] (request)--(sending){};
	\node[transition,label=below:\itn{provide\_info},rotate=90] at (1.55,1) (provide) {};
	\draw[arc] (provide)--(pending){};
	\draw[arc] (sending)--(provide){};
	\node[transition,label=below:\itn{decision}] at (1.95,0) (decision) {};
	\draw[arc] (pending) -- (decision) {}; 
	\draw[arc] (decision) -- (out) {}; 
		
\end{tikzpicture}
\captionsetup{type=subfigure}
\caption{Sound but strongly unsound workflow.}
\label{fig:notstrongsoun}
\end{minipage}
\begin{minipage}[c]{0.5\textwidth}
\centering
\vspace{-0.5cm}
\begin{tikzpicture}[font=\scriptsize,xscale=1.9,yscale=1.1]
  \tikzstyle{arc}=[->,>=stealth,thick]
	\tikzstyle{transportArc}=[->,>=diamond,thick]
  \tikzstyle{every place}=[minimum size=6mm,thick]
  \tikzstyle{every transition}=[fill=black,minimum width=2mm,minimum height=5mm]
  \tikzstyle{every token}=[fill=white,text=black]
  \tikzstyle{urgency}=[place,fill=white,minimum size=2.0mm,thin]
	\tikzstyle{inhibArc}=[->,>=o,thick]
  \node at (0,0) {};
	\node[place,label=below:{inv: $\leq 5$}] at (0,0) (in) {};
	\node[above] at (0,0.25) {\itn{in}};
	\node[place,label=below:{inv: $\leq 14$}] at (1.3,0) (pending) {};
	\node[above] at (1.3,-1.1) {\itn{pending}};
	\node[place,label=\itn{out}] at (2.65,0) (out) {};
	\node[place,label=left:{inv: $\leq 14$},label=right:\itn{sending}] at (1.3,2) (sending) {};
%	\node[above] at (1.3,2.5) {\itn{sending}};
  \node[transition,label=below:\itn{start}] at (0.65,0) (start) {};
	\draw[arc] (in) -- (start) {}; 
	\draw[arc] (start) -- (pending) {}; 
	\node[transition,label=above:\itn{req\_info},rotate=90] at (1.05,1) (request) {};
	\draw[transportArc] (pending) -- (request)  node[midway,left,yshift=-2]{$:1$} {};
	\draw[transportArc] (request) -- (sending)  node[midway,left]{$:1$} {};
	\node[transition,label=below:\itn{provide\_info},rotate=90] at (1.55,1) (provide) {};
	\draw[transportArc] (provide) -- (pending)  node[midway,right]{$:1$} {};
	\draw[transportArc] (sending) -- (provide)  node[midway,right,yshift=2]{$:1$} {};
	\node[transition,label=below:\itn{decision}] at (1.95,0) (decision) {};
	\draw[arc] (pending) -- (decision) {}; 
	\draw[arc] (decision) -- (out) {}; 
\end{tikzpicture}
\vspace{-0.2cm}
\captionsetup{type=subfigure}
\caption{Strongly sound workflow.}
\label{fig:strongsoundness}
\end{minipage}
\caption{Fragment of customer complaint workflow ($[0,\infty]$ guards
are omitted)}
\label{fig:soudnessnets}
\end{figure}

Next lemma shows that strong soundness %of a workflow net $N$
corresponds % to the finiteness of the set of execution times of $N$
to the property that any execution of the workflow net is time bounded.

%and hence to the possibility to define the maximum execution time of
%the workflow.

\begin{lemma} \label{lem:finite}
A sound and bounded ETAWFN is strongly sound if and only if
the set of its execution times $\mathcal{T}(N)$ is finite.
\end{lemma}
\begin{proof}
``$\Leftarrow$'': By contradiction we assume that $\mathcal{T}(N)$
is finite while $N$ is not strongly sound. This means that either
(i) there is a reachable divergent marking of $N$ that is not a final marking
or (ii) the net contains an infinite time-divergent computation.
In case (i) we can reach the divergent marking and perform an
arbitrarily long delay after which (thanks to soundness of $N$)
we can still reach some final marking. Hence $\mathcal{T}(N)$ is
clearly infinite, contradicting our assumption. In case (ii)
we can again follow the infinite execution for a sufficiently long time so that
an arbitrary accumulated delay is achieved and again (thanks to soundness
of $N$) we can reach some final marking, implying that
$\mathcal{T}(N)$ is again infinite, contradicting our assumption.

``$\Rightarrow$'': Let $N$ be a strongly sound workflow net.
From condition $b)$ of Definition~\ref{def:strongsound}
we know that any reachable nonfinal marking
in $N$ cannot diverge. Moreover, there is a global bound $B$ such
that any reachable marking can delay at most $B$ time units but not more.
This is due to the fact that nondivergent behaviour is guaranteed either
by age invariants (that have a fixed upper-bound limiting the maximum
delay) or by urgent transitions with input arcs having $[0,\infty]$ guards
only (prohibiting time delay as soon as a marking enables some urgent
transition).
 Also, it is impossible to have a reachable marking with
no tokens as the net cannot be sound in this case (Definition~\ref{defetawfn}
requires that every transition has at least one input place).

Let $S$ denote the number of reachable cut-markings
in the net $N$.
Hence any execution from the initial marking to some final one has either
length of no more than $S$, meaning that its accumulated time duration
is at most $S\cdot B$, or it contains the same cut marking twice, forming a loop
on the execution.
We know that there must be only zero delays on
any such a loop as otherwise we would be able to repeat the cycle
infinitely often, breaking condition $c)$ of Definition~\ref{def:strongsound}
(of course, this loop is only on the cut markings but due to
Theorem~\ref{thm:bisim} it can be found also in the real execution of the net
with exactly the same delays).
This implies that the loop can be omitted while preserving
the accumulated execution time of the path.
So we are guaranteed that the set $\mathcal{T}(N)$ is bounded by $S\cdot B$
and hence it is finite.
\end{proof}



Lemma~\ref{lem:finite} implies that 
for any strongly sound net $N$, the maximum execution
time is well defined.
Notice that for monotonic nets (even extended
with inhibitor arcs), the answer to the strong soundness is always
negative as all reachable markings are divergent. As expected,
strong soundness is in general undecidable.
%For general extended timed-arc workflow nets, 
%strong soundness in undecidable.

\begin{theorem} \label{thm:strong-undecid}
Strong soundness of ETAWFN %extended timed-arc workflow nets
is undecidable.
\end{theorem}

\begin{proof}
Similarly as in Theorem~\ref{thm:undecid},
we reduce the reachability problem for
two-counter Minsky machines into checking strong soundness. We can use
the same construction as in Figure~\ref{fig:cont} each place
$p_i$ contains the age invariant $\leq 1$ and hence there are no divergent
markings. At the same time, before executing any instruction we have to
wait exactly one time unit, hence there is no infinite computation of the
workflow net that happens during a fixed time bound. As a result,
the workflow net constructed in Figure~\ref{fig:cont} is sound if and only
if it is strongly sound and the undecidability result in 
Theorem~\ref{thm:undecid} is valid also for strong soundness.
\end{proof}

We focus now on bounded ETAWFN where
strong soundness is decidable and the maximum execution 
time computable, relying on Lemma~\ref{lem:finite}. 
We prove this by reducing strong soundness of a given bounded ETAWFN $N$
into a reachability problem on a bounded ETAPN $N(c)$, 
where $c$ is a nonnegative integer; the translation
is given in Figure~\ref{fig:nprime}. 
The token from the place \itn{timer} has to move to the place $\itn{ready}$
exactly at the time $c$ from the start of the workflow.
If the workflow can finish (by marking the place \itn{out})
after at least $c$ time units passed, then we can fire the transition
\itn{late} and mark the place \itn{after}. If %on the other hand
a token is moved to \itn{out} earlier, then the urgent transition
\itn{early} will have to fire immediately.

\begin{figure}[t]
\centering
\begin{tikzpicture}[font=\scriptsize,xscale=1.9,yscale=2.0]
  \tikzstyle{arc}=[->,>=stealth,thick]
	\tikzstyle{transportArc}=[->,>=diamond,thick]
  \tikzstyle{every place}=[minimum size=6mm,thick]
  \tikzstyle{every transition}=[fill=black,minimum width=2mm,minimum height=5mm]
  \tikzstyle{every token}=[fill=white,text=black]
  \tikzstyle{urgency}=[place,fill=white,minimum size=2.0mm,thin]
	\tikzstyle{inhibArc}=[->,>=o,thick]
  \node at (0,0) {};
  \node[place,label=above:\itn{in}] at (0,0) (in) {0};
  \node[place,label=above:\itn{out}] at (3.5,0) (out) {};
	\node[place,label=below:\itn{timer}] at (3.5,-1) (timW) {0};
	\node [above] at (3.5,-0.78) {inv: $\leq c$};
	\node[place,label=below:\itn{ready}] at (5,-1) (timD) {};
	\node[place,label=above:\itn{after},label=below:inv $\leq 0$] at (6,-1) (ok) {};
	\node[place,label=above:\itn{before},label=below:inv $\leq 0$] at (6,0) (nok) {};
	\draw[arc] (2.7,-0.3)-- (out){};
	\draw[arc] (2.7,0)-- (out){};
	\draw[arc] (2.7,0.3)-- (out){};
	\draw [dotted] (0.5,-0.5) rectangle (3,0.5){};
	\node at (1.8,0.0) (name) {workflow net $N$};
	\node[transition, label=above:$\mathit{tick}$] at (4.35,-1) (wtod) {}
	edge[pre,thick,>=stealth]node[midway,above] {[c,c]} (timW)
	edge[post,thick,>=stealth](timD);
	\node[transition, label=above:$\mathit{early}$] at (5,0) (tnok) {}
	edge[pre,thick,>=stealth](out)
	edge[pre,thick,>=o](timD)
	edge[post,thick,>=stealth](nok);
	\node[urgency] at (5,0) {};
	\node[transition, label=above:$\mathit{late}$] at (5.5,-1) (tok) {}
	edge[pre,thick,>=stealth, bend right=15](out)
	edge[pre,thick,>=stealth](timD)
	edge[post,thick,>=stealth](ok);
	\node[urgency] at (5.5,-1) {};
	\draw[arc] (in)-- (0.8,0.3){};
	\draw[arc] (in)-- (0.8,0){};
	\draw[arc] (in)-- (0.8,-0.3){};
\end{tikzpicture}
\caption{Transformation of an ETAWFN $N$ into an ETAPN $\mathit{N(c)}$}
\label{fig:nprime}
\end{figure}  

\begin{lemma} \label{lem:Nc}
Let $N$ be a sound ETAWFN. Let $M_\mathit{after} = \{(\itn{after},0)\}$
be a marking in $N(c)$ with one token in the place \itn{after}.
If $c \in \mathcal{T}(N)$ then
$N(c)$ can reach the marking $M_\mathit{after}$.
If $N(c)$ can reach the marking $M_\mathit{after}$ then
$c' \in \mathcal{T}(N)$ for some $c'\geq c$.
\end{lemma}
\begin{proof}
If $c \in \mathcal{T}(N)$ then we perform the execution lasting exactly
$c$ time units in the net $N$ and at the moment $c$ we fire the transition
$\itn{tick}$, enabling the transition \itn{late} and marking the
place \itn{after}. If on the other hand the place \itn{after} can be
marked then necessarily the token in the place \itn{out} arrived
at time $c'$ such that $c' \geq c$, otherwise the urgent transition \itn{early}
had to be fired instead.
\end{proof}



Let $N = \tapntuple$ be a given bounded ETAWFN.
We can run Algorithm~\ref{algthm:1} to check for soundness of $N$.
If it is not sound then $N$ cannot be strongly sound either.
Otherwise, let $S$ be the number of non-final cut markings reachable in $N$
(corresponding to the maximum cardinality of the set $\mathit{Reached}$
in Algorithm~\ref{algthm:1}). 
Let $B=\max\{ b \mid p \in P,\ \inv(p)=[0,b],\ b \not=\infty\}$
be the maximum integer number used in any of the age invariants in $N$.

\begin{lemma} \label{lem:mB1}
A sound and bounded ETAWFN  $N$ is strongly sound if and only if
$N(S\cdot B+1)$ cannot reach the marking $\{(\itn{after},0)\}$. 
\end{lemma}
\begin{proof}
If the net $N$ is strongly sound then there is no
reachable divergent marking with the possible exception of final markings.
Hence any reachable marking either contains
some enabled urgent transition (and so no delay is possible) or
the divergent behaviour is avoided by some age invariant, giving us
the guarantee that no reachable marking can delay more than $B$
units of time. As there are $S$ reachable non-final cut markings,
we know that any execution of $N$ using more than $S\cdot B$ units of time
must contain a loop with a non-zero time delay somewhere on the loop.
Hence if $N(S\cdot B+1)$ can mark the place \itn{after}, then either there
is a reachable divergent marking (and the net is not strongly sound) or
there exists an execution with a non-zero delay loop and by repeating
the loop infinitely often, we get an execution breaking the condition $c)$
of Definition~\ref{def:strongsound} and the net is not strongly sound either.

On the other hand, if
the place \itn{after} is not reachable in $N(S\cdot B+1)$ then it is surely
not reachable also for any other $c \geq S\cdot B+1$, meaning that
the set $\mathcal{T}(N)$ is finite by Lemma~\ref{lem:Nc}. Now
Lemma~\ref{lem:finite} and the fact that $N$ is sound
implies that $N$ is strongly sound.
\end{proof}


\begin{theorem} \label{thm:strong-max}
Strong soundness of bounded extended timed-arc workflow nets
is decidable and the maximum execution time is computable.
\end{theorem}
\begin{proof}
Let $N$ be a given bounded ETAWFN.
We first run Algorithm \ref{algthm:1} to check for soundness of $N$.
If it is not sound, we terminate and announce that $N$ is not strongly sound.
Otherwise,
we check whether $N(S\cdot B+1)$ can reach a marking containing just one token
in the place \itn{after} (this check is decidable for
bounded ETAPN~\cite{ALSST:MEMICS:12}). If this is the case, we
return that $N$ is not strongly sound due to Lemma~\ref{lem:mB1}.
Otherwise the net is sound and we return the maximum accumulated delay
in any marking discovered during the check as the maximum execution
time (correctness follows from Lemma~\ref{lem:Nc},
soundness of $N$ and the fact that once a token appears
in the place \itn{out} in $N(S\cdot B+1)$, no further delay is possible).
\end{proof}

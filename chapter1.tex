\chapter{Introduction}\label{chapter:c1}
\markboth{Chapter~\ref{chapter:c1}. Introduction}{}

\section{Motivation}\label{scope}
\markright{~\ref{scope}Motivation}

The development of software systems is becoming more complex with the 
appearance
of new computational paradigms such as Service-Oriented  Computing  (SOC), 
Grid Computing and Cloud Computing.
In these systems, the service provider needs to ensure some levels of 
quality and privacy to the final
user in a way that had never been raised. It is therefore necessary 
to develop new techniques to benefit from the advantages 
of recent approaches, as Web service compositions. 
Formal models of concurrency have been widely used for the description and analysis of concurrent and distributed systems. 
Grid/Cloud environments are characterized by a dynamic environment 
due to the heterogeneity and volatility of resources. 
There are two complementary views in composite web services: Choreography and Orchestration. 
The choreography view describes the observable interactions among services and can
be  defined  by  using  specific  languages such as Web  Services
Choreography  Description  Language  (WS-CDL), %\cite{WSCDL}
or  by
using some more general languages like UML Messages Sequence
Charts (MSC). On the other hand, orchestration concerns
the internal behaviour of a Web service in terms of invocations
to  other  services.  Web Services Business 
Process Execution Language (WS-BPEL) \cite{Andrews2003WSBPEL} is normally used 
to describe Web service orchestrations, so this
is considered the de-facto standard language for describing Web services
workflows in terms of web service compositions. 

To facilitate additional interoperability among services, more standardization is required to deal with distributed resources. In January of 2004, several members of the \emph{Globus Alliance} organization and the computer multinational \emph{IBM} with the help of experts from companies such as \emph{HP, SAP, Akamai, etc.} defined the basis architecture and the initial specification documents of a new standard for that purpose, Web Services Resource Framework (WSRF)~\cite{Ban06,foster2004}. Although the Web service 
definition does not consider the notion of state, interfaces frequently provide the user with the ability to access and manipulate
states, that is, data values that persist across, and evolve as a result of Web service interactions. The messages that the services 
send and receive imply (or encourage programmers to infer) the existence of an associated stateful resource. It is then desirable 
to define Web service conventions to enable the discovery of, introspection on, and interaction with stateful resources in standard 
and interoperable ways~\cite{foster2004}. 

The main motivation of the first part of the Thesis is to provide a formal 
semantics for WS-BPEL+WSRF to manage stateful Web services workflows 
by using the existing machinery in distributed systems,
and specifically a well-known formalism, such as prioritised-timed
coloured Petri nets, which are not only a graphical model, but they also
provide us with an easier way to simulate and analyse the modelled
system. Thus, our aim is not to provide just another WS-BPEL semantics. 
In order to deal with the integration of BPEL plus WSRF in a proper way, 
we have realized that it is more convenient 
to introduce a specific semantic model, which  covers 
properly all the relevant aspects of WSRF such as notifications and 
resource time-outs. 
The integration of WS-BPEL and WSRF is not new; in the literature, 
there are a bundle of works defining this integration, but 
none of these works define a formal semantics in terms of Petri
nets.

The aim of the second part of the Thesis is to extend a mature formalism like
Workflow nets with time features. To this end, we use timed-arc Petri nets, where
each token has attached a timestamp, allowing to measure, for instance, execution times. 
Workflow nets are useful when designing how the participants of a business process interact
and in which order. Thus, it is easy to check bottlenecks or undesirable behaviour in our processes.
Moreover, we prove the decidability (or undecidability) of a fundamental property for workflow nets, soundness.
This property is checked using a continuous and a discrete time semantics. To add value to our workflow theory, we have included
it into a well-known tool, Tapaal.
 
\section{Objectives}\label{objectives}

Next, we describe the general and specific objectives of the Thesis.

\subsection*{General Objectives}

Two main objectives for this Thesis have been considered. The first one is the formal definition (syntax and semantics)
of a language that encapsulates the main aspects of composite web services (provided with distributed resources) 
from the orchestration viewpoint.  
The second objective is to propose a new extension of workflow nets in terms 
of timed-arc Petri nets, thus providing two formal models in which the time semantics can be discrete or continuous. 
Obviously, these objectives are too general and, therefore, we list below a set of subobjectives that are
required to achieve these overall objectives.


\subsection*{Specific Objectives}

To meet this overall objectives the following specific objectives are also achieved:

\begin{itemize}

\item \textbf {Objective 1: State-of-the-art}

\begin{enumerate}

\item Study of different formalisms for the modelling and analysis of Grid/Cloud Computing applications using web services.

\item Summarise the current definitions of soundness and the different extensions of workflow nets presented to date.

\end{enumerate}

\item \textbf{Objective 2: Technological framework definition}

\begin{enumerate}

\item Study of the current techniques for modelling and implementing web service compositions and Grid/Cloud Computing applications.

\item Analysis of the different tools for modelling workflow nets as well as possible target applications of the theory presented in this Thesis.
\end{enumerate}

\item \textbf{Objective 3: Development of the proposal}

\begin{enumerate}

\item Define the specific models, first the syntax, and, then, the operational semantics and the Petri nets semantics for the language BPELRF.

\item Extend the current definition of workflow nets with a time semantics.

\item Adapt the definition of soundness to this timed scenario.

\item Develop tools supporting the theory presented here.

\item Analyse and evaluate both proposals.

\end{enumerate}

\item \textbf{Objective 4: Examples and Case studies}

\begin{enumerate}

\item Propose a set of simple examples where the main features are illustrated.
\item Study a set of theoretical examples where the power of both proposals and its main aspects are characterized .
\item Demonstrate the applicability of this work by applying it to real (industry-based) case studies.
\end{enumerate}

\end{itemize}

\section{Dissertation Structure}\label{structure}
\markright{~\ref{structure} Dissertation Structure}

This Thesis is organised in five different chapters, as follows.

\textbf{Chapter~\ref{chapter:c1}} makes a brief introduction to the Thesis, showing the motivation, the main objectives and the scope of it.

\textbf{Chapter~\ref{chapter:c2}} shows the state of art for the contents included here. This chapter includes a brief description of Service-Oriented Computing (SOC) and distributed computing, e.g. Grid and Cloud computing and the use of formal methods for the analysis of web service compositions and the benefits of using formal techniques in the development of software and hardware. Moreover, a comprehensive introduction of the standards used in this work is presented. Next, we get into workflow nets and the possible time extension of them, as well as its main properties. Finally, we describe the basic notions of the formal models of concurrency used in the Thesis.

In \textbf{Chapter~\ref{chapter:c3}}, we define the extended models of concurrency used in this work. We start defining a 
general extension of basic Petri nets and, then, we present two extensions of them. Some properties for the study of concurrent systems are also described. 
First, we introduce prioritised-timed Petri nets as they are used in 
the Chapter \ref{chapter:c4} as the visual formal model for the language BPELRF and, second, we define timed-arc Petri nets as they will be used as formalism
to define our timed extension of workflow nets. 

\textbf{Chapter~\ref{chapter:c4}} presents a formal specification language called BPELRF, which takes two well-known standards (WS-BPEL and WSRF) as basis, to model synchronous and asynchronous stateful interactions. This language is enriched with a publish-subscribe architecture, service discovery, event and fault handling and time-outs. As usual, an operational semantics for this language is defined. Moreover, we define a visual model of it in terms of coloured Petri nets and a tool to that allows us to get an automatic and easy translation to CPNTools input model, thus permitting us to make simulation and verification of some properties.

In \textbf{Chapter~\ref{chapter:c5}} we suggest a timed workflow model based on timed-arc Petri nets, and study
the foundational problems of soundness and strong (time-bounded) soundness.
We explore the decidability of these problems
and compare the discrete and continuous semantics of timed-arc
workflow nets. 

Finally, the main conclusions, contributions and future works of this Thesis are described in \textbf{Chapter~\ref{chapter:c5}}.
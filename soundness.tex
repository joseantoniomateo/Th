%%%%%%%%%%%%%%%%%%%%%% Section%%%%%%%%%%%%%%%%%%%%%%%%%%%%%
Timed-arc workflow nets are defined similarly as untimed 
workflow nets~\cite{Aalst97}. Every workflow net has a unique
input place and a unique output place. After initializing such a 
net by placing a token into the input place, it should be
guaranteed that any possible workflow execution can be always 
extended such that the workflow terminates with just one token
in the output place (also known as the soundness property).

% with the notion of time associated to the tokens in the net.

\begin{definition}[Extended timed-arc workflow net] \label{defetawfn}
An \itn{ETAPN} $N=\tapntuple$ is called an 
\emph{Extended Timed-Arc WorkFlow Net} (ETAWFN) 
if
\begin{itemize}
\item there exists a unique place $in \in P$ such that $^\bullet in= \emptyset$ and $in^\bullet \neq \emptyset$,
\item there exists a unique place $out \in P$ such that $out^\bullet=\emptyset$ and $^\bullet out \neq \emptyset$,
%\item $\forall t \in T$, there is a path from {\it in} to {\it t} and from {\it t} to {\it out}. % Jiri selected the item 3 instead of this
\item for all $p \in P \setminus \{in,out\}$ we have $^\bullet p \neq \emptyset$ and $p^\bullet \neq \emptyset$, and
\item for all $t \in T$ we have $^\bullet t \neq \emptyset.$
\end{itemize} 
\end{definition}

%Let us also present a short remark regarding Definition~\ref{defetawfn}.
\begin{remark}
Notice that the conditions $^\bullet in= \emptyset$ and
$^\bullet out \neq \emptyset$ necessarily imply that $in \not= out$.
Moreover, we allow the postset of a transition to be empty 
($t^\bullet=\emptyset$). This is just a technical detail and
an equivalent workflow net where all transitions satisfy
$t^\bullet \not= \emptyset$ can be constructed
by introducing a new place $p_\mathit{new}$ so that any outgoing transition
from the start place $in$ puts a token into $p_\mathit{new}$ and every
incoming transition to the final place $out$  consumes the token
from $p_\mathit{new}$. Now for any transition $t$ with 
$t^\bullet=\emptyset$ we add the pair of arcs $(p_\mathit{new},t)$
and $(t,p_\mathit{new})$ without influencing the behaviour of the net.
\end{remark}

Decidability of soundness crucially depends on the modelling
features allowed in the net. Hence we define a subclass
of so-called monotonic workflow nets.

\begin{definition}[Monotonic timed-arc workflow net] \label{deftawfn}
A monotonic timed-arc workflow net (MTAWFN) is an ETAWFN 
with no urgent transitions, no age invariants and no inhibitor arcs.
\end{definition}

The marking $M_\itn{in}=\{(\mathit{in},0)\}$
 of a timed-arc workflow net is called
\emph{initial}. 
%if $|M(in)|=1$ and for all
%$p \in P\smallsetminus\{in\}$ we have $|M(p)|=0$, i.e. it contains just
%one token (of some age) in the place $in$.
A marking $M$ is \emph{final} if 
$|M(\itn{out})|= 1$ and for all $p \in P\smallsetminus\{\itn{out}\}$ 
we have $|M(p)|=0$,
i.e. it contains just one token in the place \itn{out}. There may
be several final markings with different ages of the token in the 
place \itn{out}.

We now provide the formal definition of soundness that formulates 
the standard requirement on proper termination of workflow
nets~\cite{Aalst98,AalstHHSVVW11}.
%nets~\cite{Aalst98,JKJ:ATPN:10,AalstHHSVVW11}.

\begin{definition}[Soundness of timed-arc workflow nets] \label{def:soundness}
An (extended or monotonic) timed-arc workflow net $N=\tapntuple$ is 
sound if for %any initial marking $M_\mathit{in}$ and 
any marking $M \in [M_\mathit{in}\rangle$ reachable from the initial
marking $M_\mathit{in}$:
\begin{description}
\item{a)}  there exists some final marking $M_\mathit{out}$ such that 
           $M_\mathit{out} \in [M\rangle$, and 
% (option to complete), and
\item{b)} if $|M(\itn{out})|\geq 1$ then M is a final marking.
% (proper completion).
\end{description}
\end{definition}

A workflow is sound if once it is initiated by placing a token of age $0$ in
the place \itn{in}, it has always the possibility to terminate by moving 
a token to the place \itn{out} (option to complete)
%condition a) of soundness definition)
and moreover it is guaranteed that the rest of the workflow net is free 
of any remaining tokens as soon as the place \itn{out} is marked 
(proper completion).
%(condition b) of the definition).
We now define a subclass of bounded workflow nets.
% where the number
%of tokens in any reachable marking is a priori bounded by some given constant.

\begin{definition}[Boundedness]
\label{boundedness}
A %n (extended or monotonic) 
timed-arc workflow net $N$ is \emph{$k$-bounded}
for some $k \in \nnul$ if %for any initial marking $M_{in}$ and any
any marking $M$ reachable from the initial marking $M_{in}$
%($M \in [M_{in}\rangle$) 
satisfies $|M| \leq k$.
A net is \emph{bounded} if it is $k$-bounded for some $k$.
\end{definition}

A classical result states that any untimed sound net is bounded~\cite{Aalst97}.
This is not in general the case for extended timed-arc
workflow nets 
%as inhibitor arcs, age invariants and urgent transitions
%can model sound but unbounded workflows 
as demonstrated in Figure~\ref{fig:unsound}. Nevertheless, we recover the boundedness
result for the subclass of monotonic timed-arc workflow nets.

\begin{theorem}\label{thm:soundness}
Let $N$ an MTAWFN. If $N$ is sound then $N$ is bounded. 
\end{theorem}
\begin{proof}
By contradiction assume that $N$ is a sound and unbounded MTAWFN.
Let $M_{in}$ be the initial workflow marking.
Now we can argue that there must exist two
reachable markings $M, M' \in [M_{in}\rangle$
such that
\begin{itemize}
\item[i)] $M \sqsubseteq_{\cut} M'$, and
\item[ii)] $|M| < |M'|$.
\end{itemize}

This follows from the fact that $M \sqsubseteq_{\cut} M'$ iff
$\cut(M) \sqsubseteq \cut(M')$ and from Definition~\ref{def:cut}
where the cut function is given such that each token is placed
into one of the finitely many places, say $p$, and its age is bounded
by $\cmax(p)+1$. Thanks to Dickson's Lemma~\cite{dickson}, saying
that every set of $n$-tuples of natural numbers has only finitely
many minimal elements, we are
guaranteed that conditions i) and ii) are satisfied for some
reachable markings $M$ and $M'$.

Since $N$ is a sound workflow net, we now use condition $a)$
of Definition~\ref{def:soundness}, implying that
from $M$ we reach some final marking $M_{out}$. Assume that this is achieved
w.l.o.g.  by the following sequence of transitions:
 $$M \trans{d_1} M_1 \trans{t_1}
  M_2 \trans{d_2} M_3 \trans{t_2}
  M_4 \ldots \trans{t_n} M_{out} \ . $$
We know that $M \sqsubseteq_{\cut} M'$ and hence by repeatedly applying
Lemma~\ref{lem:mono}
also
 $$M' \trans{d_1} M'_1 \trans{t_1}
  M'_2 \trans{d_2} M'_3 \trans{t_2}
  M'_4 \ldots \trans{t_n} M'_{out}  $$
such that at the end $M_{out} \sqsubseteq_{\cut} M'_{out}$. The facts that
$M_{out} \sqsubseteq_{\cut} M'_{out}$ and $M_{out}$ is a final marking
imply that $|M'_{out}(out)| \geq 1$.
By a repeated application of Lemma~\ref{lem:mono}
we also get $|M'|-|M|=|M'_{out}|-|M_{out}|$.
By condition ii) of this lemma we know that $|M| < |M'|$,
this implies that also $|M_{out}| < |M'_{out}|$.
However, now the place
$out$ in $M'_{out}$ is marked and there is at least one more token
somewhere else in the marking $M'_{out}$. This contradicts
condition $b)$ of Definition~\ref{def:soundness}.
\end{proof}

\begin{figure}[h]
\centering
\subfigure[Inhibitor arcs.\label{fig:inhun}]{
\centering
\begin{tikzpicture}[font=\scriptsize,xscale=.6,yscale=0.7]
\tikzstyle{arc}=[->,>=stealth,thick]
\tikzstyle{darc}=[<->,>=stealth,thick]
\tikzstyle{transportArc}=[->,>=diamond,thick]
\tikzstyle{inhibArc}=[->,>=o,thick]
\tikzstyle{every place}=[minimum size=6mm,thick]
\tikzstyle{every transition} = [fill=black,minimum width=2mm,minimum height=5mm]
\tikzstyle{every token}=[fill=white,text=black]
\tikzstyle{sharedplace}=[place,minimum size=7.5mm,dashed,thin]
\tikzstyle{urgency}=[place,fill=white,minimum size=2.0mm,thin]
\tikzstyle{sharedtransition}=[transition, fill opacity=0, minimum width=3.5mm, minimum height=6.5mm,dashed]

%%%%%%%%%%%%%%%% Places %%%%%%%%%%%%%%%%%%%%%
\node[] at (0,0){};
\node[place,label=above:$in$] (in) at (0,1.5) {}; % Place in
\node[place]  (p1) at (2,1) {}; % Place p1
\node[place]  (p2) at (2,-2) {}; % Place p2
\node[place,label=above:$out$] (out) at (4,1.5) {}; % Place out

%%%%%%%%%%%%%%%% Transitions %%%%%%%%%%%%%%%%%%%%%%%%%

\node[transition, rotate=90] (a) at (0,0) {} % 
edge[pre] node[above] {} (in)
edge[post] (p1);

\node[transition,rotate=90] (b) at (2,-.7) {} % 
edge[pre, bend left=15] node[midway, left] {} (p1)
edge[post] (p2)
edge[post, bend right=15] (p1);

\node[transition, rotate=90] (c) at (4,0) {} % 
edge[pre] node[above] {} (p1)
edge[post] (out);

\node[transition] (d) at (0,-2) {} % 
edge[pre] node[auto] {$$} (p2);

\draw [-o] (p2) -- (4,-2) -- (c) ;
\end{tikzpicture}
}
 \hfill
\subfigure[Age invariants.\label{fig:invun}]{
\centering
\begin{tikzpicture}[font=\scriptsize,xscale=0.7,yscale=0.7]
\tikzstyle{arc}=[->,>=stealth,thick]
\tikzstyle{darc}=[<->,>=stealth,thick]
\tikzstyle{transportArc}=[->,>=diamond,thick]
\tikzstyle{inhibArc}=[->,>=o,thick]
\tikzstyle{every place}=[minimum size=6mm,thick]
\tikzstyle{every transition} = [fill=black,minimum width=2mm,minimum height=5mm]
\tikzstyle{every token}=[fill=white,text=black]
\tikzstyle{sharedplace}=[place,minimum size=7.5mm,dashed,thin]
\tikzstyle{sharedtransition}=[transition, fill opacity=0, minimum width=3.5mm, minimum height=6.5mm,dashed]

%%%%%%%%%%%%%%%% Places %%%%%%%%%%%%%%%%%%%%%

\node[place,label=above:$in$] (in) at (0.2,1.5) {}; % Place in
\node[place]  (p1) at (2,1) {}; % Place p1
\node[place, label=right:inv: $\leq 0$]  (p2) at (2,-2) {}; % Place p2
%\node[place]  (p3) at (-2,-3) {}; % Place p3
\node[place,label=above:$out$] (out) at (4,1.5) {}; % Place out

%%%%%%%%%%%%%%%% Transitions %%%%%%%%%%%%%%%%%%%%%%%%%

\node[transition, rotate=90] (a) at (0.2,0) {} % 
edge[pre] node[above] {} (in)
edge[post] (p1);

\node[transition,rotate=90] (b) at (2,-.7) {} % 
edge[pre, bend left=15] node[midway, left,xshift=1.5] {$[0,0]$} (p1)
edge[post] (p2)
edge[post, bend right=15] (p1);

\node[transition, rotate=90] (c) at (4,0) {} % 
edge[pre] node[midway, above] {$[1,\infty]$} (p1)
edge[post] (out);

\node[transition] (d) at (0.2,-2) {} % 
edge[pre] node[auto] {$$} (p2);
\end{tikzpicture}
}
 \hfill
\subfigure[Urgent transitions.\label{fig:urgun}]{
\begin{tikzpicture}[font=\scriptsize,xscale=0.95, yscale=0.7]
\tikzstyle{arc}=[->,>=stealth,thick]
\tikzstyle{darc}=[<->,>=stealth,thick]
\tikzstyle{transportArc}=[->,>=diamond,thick]
\tikzstyle{inhibArc}=[->,>=o,thick]
\tikzstyle{every place}=[minimum size=6mm,thick]
\tikzstyle{every transition} = [fill=black,minimum width=2mm,minimum height=5mm]
\tikzstyle{every token}=[fill=white,text=black]
\tikzstyle{sharedplace}=[place,minimum size=7.5mm,dashed,thin]
\tikzstyle{sharedtransition}=[transition, fill opacity=0, minimum width=3.5mm, minimum height=6.5mm,dashed]
\tikzstyle{urgency}=[place,fill=white,minimum size=2.0mm,thin]

%%%%%%%%%%%%%%%% Places %%%%%%%%%%%%%%%%%%%%%

\node[place,label=above:$in$] (in) at (0.2,1.5) {}; % Place in
\node[place]  (p1) at (1.5,1) {}; % Place p1
\node[place]  (p2) at (1.5,-2) {}; % Place p2
\node[place,label=above:$out$] (out) at (3,1.5) {}; % Place out

%%%%%%%%%%%%%%%% Transitions %%%%%%%%%%%%%%%%%%%%%%%%%

\node[transition, rotate=90] (a) at (0.2,0) {} % 
edge[pre] node[above] {} (in)
edge[post] (p1);

\node[transition,rotate=90] (b) at (1.5,-.7) {} % 
edge[pre, bend left=15] node[midway, left,xshift=1.5] {$[0,0]$} (p1)
edge[post] (p2)
edge[post, bend right=15] (p1);

\node[transition, rotate=90] (c) at (3,0) {} % 
edge[pre] node[midway, above] {$[1,\infty]$} (p1)
edge[post] (out);

\node[transition] (d) at (0.2,-2) {} % 
edge[pre] node[auto] {$$} (p2);
\node[urgency] at (0.2,-2){};
\end{tikzpicture}
}
\caption{Sound and unbounded extended timed-arc workflow nets.\label{fig:unsound}}
\end{figure}




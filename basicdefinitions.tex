\subsection*{Preliminaries}
In this section, we will define the preliminary concepts
used in this Thesis. The aim of this section is to provide the reader
with a review of the elemental notions used in future sections
as well as to fix the notations used throughout the Thesis.
Thus, we will start presenting basic concepts such as the standard definition of Petri
nets and we will continue with more technical details such as the addition of time features
to this formal model.


\subsubsection{Notation}
The notation used here is the following:

\begin{enumerate}
\item {\bf Numbers}\\
We will denote by $\nnul = \mathbb{N} \cup \{0\}$ the set of nonnegative integers including 0,
and by $\rnul$ the nonnegative real numbers including 0. Obviously, $\mathbb{N}$
and $\mathbb{R}$ mean that zero is excluded from the set. Moreover,
$\nnul^{\infty} = \nnul \cup \left\{\infty \right\}$ is the set of natural numbers including $\infty$. 
%Asimismo, denotaremos por $Q$ los n\'{u}meros racionales, y por
%$Q^{+}$ los n\'{u}meros racionales positivos.

\item {\bf Sets and Multisets}\\
We will use the standard delimiters for sets (\{\}) and multisets(\multiset{}). 
As usual, let $A$ be a (multi)set and $R\,:\,A \longrightarrow \nnul$, we say 
that $x \in A$ iff $R(x) > 0$.
Note that we abuse the notation using $x \in A$ to represent that $x$ is an element of
the multiset or the set $A$, and using indistinctly the general set operators (union, inclusion, etc.) for sets and multisets. 
The cardinality of a set (multiset) $A$
is denoted by $|A|$. Given a set $A$,
${\mathcal B}(A)$ is the set of all finite multisets over $A$.

For any $C_1, C_2 \in {\cal B}(A)$, we define:
\begin{itemize}
%\item $\nnul$ will denote the set of natural numbers,
%$\nat=\{0,1,2,\ldots\}$ and $\Sigma = \nat \times \nat$.
%%
%\item Multisets are defined as functions $C\,:\,X \rightarrow \nnul$,
%providing us with the number of instances of each element $x\in X$.
%As usual, we will enumerate the elements of a multiset $C$ as follows:
%$C = \{ r_1.x_1,\ldots,r_n.x_n \}$, meaning that $C(x_i)=r_i$,
%for all $i=1,\ldots,n$, and $C(x)=0$, for all $x \neq x_i, \, i=1,\ldots,n$.
%
%The set of multisets over a set $X$ will be denoted by ${\cal B}(X)$.
%For any $x \in X$ and $C\in {\cal B}(X)$ we say that $x \in C$
%if and only if $C(x)>0$.
%
%
\item $C_1 \uplus C_2 \in {\cal B}(A)$, where $\forall x \in A\,:\,
        (C_1 \uplus C_2)(x)= C_1(x) + C_2(x)$.
  %
\item $C_1 \subseteq C_2$ if and only if $\forall x \in A\,:\,
        C_1(x) \leq C_2(x)$.
  %
\item If $C_2 \subseteq C_1$ we can define the subtraction 
        $C_1 \smallsetminus C_2 \in {\cal B}(A)$, where $\forall x \in A\,:\,
        (C_1 \smallsetminus C_2)(x)= C_1(x) - C_2(x)$.

\item For any $C \in {\cal B}(A)$, 
we define the first projection
$\Pi_1(C) \in {\cal B}(\nnul)$, 
as follows: $\forall n \in \nnul,\,
\Pi_1(C)(n) = \sum_{m\in \nnul} C(n,m)$.
%
\item For any $C \in {\cal B}(A)$
and $n\in \nnul$ we define the second projection 
$\Pi_2(C,n)$ as the ordered list that consists of the
elements $(m_1,m_2,\ldots,m_{\Pi_1(C)(n)})$,
such that $(n,m_i) \in C$, $\forall i=1,\ldots, \Pi_1(C)(n)$
and $m_i \leq m_{i+1}$,
$\forall i=1,\ldots, \Pi_1(C)(n)-1$.
%
\item For any $C_1,C_2 \in {\cal B}(A)$, we say that
$C_1 \preceq C_2$ if and only if the following conditions hold:
  %
  \begin{itemize}
  \item $\Pi_1(C_1) \subseteq \Pi_1(C_2)$.
  %
  \item \mbox{$\forall n \in \nnul$}, taking \mbox{$\Pi_2(C_1,n) = (m^1_1,\ldots,
        m^1_{\Pi_1(C_1)(n)})$} and
        \mbox{$\Pi_2(C_2,n) = $} \linebreak
        $(m^2_1,\ldots,m^2_{\Pi_1(C_2)(n)})$,
        we must have  $m^1_i \geq m^2_i$, $\forall i=1,\ldots,
        \Pi_1(C_1)(n)$.
  \end{itemize}
%
%  
To sum up, these conditions state that for every $n$ the total number of
elements $(n,m)$ (moving $m$) must be lesser in $C_1$ than in $C_2$, and 
for every element $(n,m)$ in $C_1$ there must be a 
corresponding (distinct element) $(n,m')$ in $C_2$, with $m\geq m'$.
%
\item For any $C_1,C_2 \in {\cal B}(A)$, with $C_1 \preceq C_2$,
we define $C_2 \ominus C_1$ in the following (recursive) way: 
  %
  \begin{itemize}
  \item For $C_1 = \emptyset$ we take $C_2 \ominus C_1 = C_2$.
  %
  \item For $C_1 \neq \emptyset$, let us consider that 

        \noindent $C_2 = \{ r^1_1.(n_1,m^1_1), \ldots,
         r^1_{i_{n_{1}}}.(n_1,m^1_{i_{n_{1}}}),
        \ldots, r^k_1.(n_k,m^k_1),\ldots, r^k_{i_{n_{k}}}.(n_k,m^k_{i_{n_{k}}}) \}$,
        where $n_l \neq n_j$, $\forall l \neq j$, and
        $m^l_j < m^l_{j+1}$, $\forall l=1,\ldots,k$ and $\forall
        j=1,\ldots,i_{n_{l}}$.

        Since $C_1 \preceq C_2$, we can take one element $(n_l,m) \in C_1$, 
        for some $l\in \{1,\ldots,k\}$, as well as the largest index $j$
        for which $m^l_{j} \leq m$. We then define recursively:
       
        \noindent $C_2 \ominus C_1 = %\begin{array}[t]{l}
        (
        \{ r^1_1.(n_1,m^1_1), \ldots, r^1_{i_{n_{1}}}.(n_1,m^1_{i_{n_{1}}}),
        \ldots,
        r^l_1.(n_l,m^l_1), \ldots,(r^l_{j} - 1).(n_l,m^l_j),\\
        \ldots,
        r^l_{i_{n_{l}}}.(n_l,m^l_{i_{n_{l}}}), 
        \ldots, r^k_1.(n_k,m^k_1),\ldots, r^k_{i_{n_{k}}}.(n_k,m^k_{i_{n_{k}}}) \}
        ) 
        \ominus (C_1 - \{ 1.(n_l,m) \})
        %\end{array}
        $.

        %It is immediate that this recursive definition is correct.
        Thus, $C_2 \ominus C_1$ is obtained by removing from $C_2$ 
        elements $(n,m)$ that correspond to elements $(n,m')$ of $C_1$,
        such that $m$ is the largest value with $m\leq m'$.

        % 
        \vspace*{0.12cm}

        For instance, taking $C_1 = \{
             1.(2,3), 1.(2,5), 1.(1,4), 1.(7,6)
        \}$, and $C_2 = $ \linebreak
        $\{
             1.(2,0), 1.(2,1), 1.(2,2), 1.(1,3), 2.(7,6), 3.(3,3)
        \}$ it follows that $C_1 \preceq C_2$. Then,
        $C_2 \ominus C_1= \{
            1.(2,0), 1.(7,6), 3.(3,3)
        \}$.
  \end{itemize} 
\end{itemize}

The last multiset operations are related with the definition of binding and firing for coloured Petri nets.

\item {\bf Relations}\\
Let $X$ be a set, a relation over $X$ is a set $R \subseteq X \times X$. 
The domain (or the set of departure) of $R$, denoted by $dom(R)$, is:
\[dom(R) = \{ x \in X \,|\, \exists
y \in X\,:\, (x,y) \in R \}\]
and the codomain (or the set of destination) of $R$, denoted by $cod(R)$, is:
\[cod(R) = \{ x \in X \,|\, \exists\ y \in X\,:\,(y,x) \in R\}\]
Given a relation $R$, the {\it reflexive and transitive closure} of $R$, $R^*$, is defined as follows:
\[ R^* = \{ (x,y)\,|\,x=y \,\vee\,
\exists\ x_1,\ldots,x_n\, \text{s.t.} \,(x,x_1)\in R,\ldots,(x_n,y) \in R\}\]
Moreover, the {\it transitive closure} of $R$, $R^+$, is given by:
\[R^+ = \{ (x,y)\,|\,
\exists \ x_1,\ldots,x_n \text{ s.t. } (x,x_1)\in R,\ldots,(x_n,y) \in R\}\]

%\item {\bf Vectores}\\
%La notaci\'{o}n empleada para representar los vectores
%ser\'{a} la usual, mediante tuplas. En el caso de vectores con componentes
%en $\nnul$, diremos que $v \geq w$ sii todas las componentes de $v$
%son mayores o iguales que las correspondientes de $w$. Adem\'{a}s,
%diremos que $v > w$ si $v \geq w$ y $v \neq w$.
\end{enumerate}


\section*{Petri nets}

\begin{definition} [(Basic Petri nets)]
A \emph{basic Petri net} (PN) is a triple $N=(P,T,F)$, where $P$ and $T$
are sets and $F$ is a relation defined over $P \,\cup\,T$. Moreover, it has to satisfy
the following constraints:
\begin{enumerate}
\item $P \,\cap \,T = \emptyset$ ($P$ and $T$ are disjoint)
\item $F \subseteq (P \times T) \,\cup\, (T \times P)$ (arcs from places to transitions and vice versa)
\item $dom(F) \, \cup \, cod(F) = P \, \cup \, T$ (no isolated places or transitions)
\end{enumerate}

In a Petri net, $P$ is known as the set of \emph{places} of $N$, $T$ 
is the set of {\it transitions} and $F$ is a flow relation between the places in $P$
and the transitions in $T$. This relation is graphically represented by arcs.
In this Thesis, we suppose that $P$ and $T$ are finite. Petri nets
can be graphically represented by means of bipartite graphs (or bigraphs), which
are graphs whose vertices can be divided into two disjoint sets ($P$ and $T$ in this case) such that 
every edge connects an element from $P$ to $T$, and vice versa. In the graphical representation, places are drawn
as circles and transitions as rectangles or boxes.  The places 
from which an arc runs to a transition are called the \emph{input places} of the transition, whereas
the places from which an arc runs from the transition are called the \emph{output places}.

Let $X = P\,\cup\,T$ be a set and $x \in X$
an element of this set. The preset of $x$ is
$\precond{x} = \{ y \in X \,|\, (y,x) \in F\}$, whereas the postset of $x$ 
is defined as $x^{\bullet} = \{ y \in X \,|\, (x,y) \in F\}~$.

A net $N$ is $T$-restricted iff $\precond{t} = t^{\bullet} =
\emptyset\,\,\,\ \text{for all } t \in T$.
\qed
\end{definition}

\begin{example} Let $N=(P,T,F)$ be a Petri net such that:
\[\begin{array}{l}
P = \{ p_1,\,p_2,\,p_3\}\\
T = \{ t_1,\,t_2\}\\
F = \{ (p_1,t_1),\,(p_2,t_1),\,(t_1,p_3),\,(p_3,t_2)\}
\end{array}\]

This net is depicted in Figure \ref{fig201}.
\end{example}

\begin{figure}
\setlength{\unitlength}{0.0125in}
\begin{picture}(130,170)(46,638)
\thicklines
\put(267,722){\circle{16}}
\put(300,802){\circle{16}}
\put(241,803){\circle{16}}
\put(267,712){\vector( 0,-1){ 37}}
\put(267,755){\vector( 0,-1){ 25}}
\put(290,800){\line(-2,-3){ 10}}
\put(280,785){\vector(-1,-4){  5}}
\put(250,800){\line( 2,-3){ 10}}
\put(260,785){\vector( 0,-1){ 20}}
\put(255,666){\framebox(25,7){}}
\put(255,756){\framebox(25,7){}}
\put(279,730){\makebox(0,0)[lb]{\raisebox{0pt}[0pt][0pt]{ $p_3$}}}
\put(283,676){\makebox(0,0)[lb]{\raisebox{0pt}[0pt][0pt]{ $t_2$}}}
\put(284,764){\makebox(0,0)[lb]{\raisebox{0pt}[0pt][0pt]{ $t_1$}}}
\put(309,810){\makebox(0,0)[lb]{\raisebox{0pt}[0pt][0pt]{ $p_2$}}}
\put(220,812){\makebox(0,0)[lb]{\raisebox{0pt}[0pt][0pt]{ $p_1$}}}
\end{picture}




\caption{\label{fig201} Example of a basic Petri net.}
\end{figure}

Graphically, places in a Petri net may contain a discrete number of marks called \emph{tokens}. 

\begin{definition} [(Marking on basic Petri nets)]
Let $N=(P,T,F)$ be a basic Petri net.
The function $M: P \longrightarrow
\nnul$ is called {\it the marking of $N$}. Then,
$(P,T,F,M)$ is called a {\it marked Petri net}.
\end{definition}

Any distribution of tokens over the places will represent a configuration of the net called \emph{marking}.
The marking of a Petri net is graphically represented by drawing
in each place as many dots as tokens correspond,
or putting into each place the number of tokens associated with it. 


\begin{example} In the net of Figure \ref{fig201}, we can consider the following marking:
\[ M(p_1) = 1,\,\,\,M(p_2)=1,\,\,\,M(p_3)=0 \]
The graphical representation of this marking is shown in Figure \ref{fig202}.
\end{example}

\begin{figure}
\setlength{\unitlength}{0.0125in}
\begin{picture}(130,170)(46,638)
\thicklines
\put(267,722){\circle{16}}
\put(300,802){\circle{16}}
\put(300,802){\circle*{2}}
\put(241,803){\circle{16}}
\put(241,803){\circle*{2}}
\put(267,712){\vector( 0,-1){ 37}}
\put(267,755){\vector( 0,-1){ 25}}
\put(290,800){\line(-2,-3){ 10}}
\put(280,785){\vector(-1,-4){  5}}
\put(250,800){\line( 2,-3){ 10}}
\put(260,785){\vector( 0,-1){ 20}}
\put(255,666){\framebox(25,7){}}
\put(255,756){\framebox(25,7){}}
\put(279,730){\makebox(0,0)[lb]{\raisebox{0pt}[0pt][0pt]{$p_3$}}}
\put(283,676){\makebox(0,0)[lb]{\raisebox{0pt}[0pt][0pt]{$t_2$}}}
\put(284,764){\makebox(0,0)[lb]{\raisebox{0pt}[0pt][0pt]{$t_1$}}}
\put(309,810){\makebox(0,0)[lb]{\raisebox{0pt}[0pt][0pt]{$p_2$}}}
\put(220,812){\makebox(0,0)[lb]{\raisebox{0pt}[0pt][0pt]{$p_1$}}}
\end{picture}




\caption{\label{fig202} Example of a marked Petri net.}
\end{figure}

The semantics of a Petri net is defined by the following \emph{firing rule}, which
represents the marking reached after firing a transition.

\begin{definition}[(Enabling rule)]
Let $N=(P,T,F,M)$ be a marked Petri net. A transition $t \in T$
is {\it enabled} by the marking M, denoted by $M[t\rangle$, if for all place
$p \in P$ such that $(p,t) \in F$, $M(p) > 0$.
\end{definition}

\begin{definition}[(Firing rule)]
The \emph{firing} of a transition $t$ enabled by the marking $M$
produces a new marking on the net, $M'$, defined as:
\[M'(p) = M(p) - W_f(p,t) + W_f(t,p)~~~\forall p \in P\]
where for all $x \in (T \times P) \, \cup \, (P \times T)$,  
$W_f(x) = 1$ if $x \in F$ and $W_f(x) = 0$, if $x \not\in F$.
This is denoted by $M[t\rangle M'$.
\end{definition}

\begin{example} In Figure \ref{fig202}, the firing of the transition $t_1$
creates the marking $M'$:
\[ M'(p_1) = 0,\,\,\,M'(p_2)=0,\,\,\,M'(p_3)=1 \]
\end{example}

\begin{definition} [(Concurrent enabling of transitions)]
Let $N= (P,T,F,M)$ be a marked Petri net and $R \subseteq T$ a subset of transitions of $N$.
The set of transitions $R$ is \emph{concurrently enabled}, denoted by $M [ R \rangle$
iff $M(p) \geq \sum_{t \in R} W_f(p,t),~
\forall p\in P$, where
$W_f(p,t)$ is defined as in the last definition.

We can also extend this definition to multisets, thus allowing
multiple instances of the same transition to be fired in just one step.
In this way, we say that the multiset of transitions $R$ 
is enabled in $M$ iff $M(p)
\geq \sum_{t \in T} W_f(p,t) \cdot R(t)$, $\forall p \in P$.

The firing of the multiset of transitions $R$ in $M$ 
produces the new marking $M'$ of $N$:
\[ M'(p) = M(p) - \sumas{t \in T} (W_f(p,t) - W_f(t,p)) \cdot R(t)\]
This net evolution in just one step is denoted by
$M[ R \rangle M'$.
\end{definition}

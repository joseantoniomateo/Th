\subsection*{Preliminaries}
In this section, we will present the preliminary concepts
used in this Thesis. The aim of this section is to provide the reader
with a review of the elemental notions used in future sections
as well as to fix the notion used throughout the Thesis.
Thus, we will start presenting basic concepts such as the standard definition of Petri
nets and we will continue with more technical details such as the addition of time features
to this formal model.


\subsubsection{Notation}
The notation used here is the following:

\begin{enumerate}
\item {\bf Numbers}\\
We will denote by $\nnul = \mathbb{N} \cup \{0\}$ the set of nonnegative integers including 0,
and by $\rnul$ the nonnegative real numbers including 0. Obviously, $\mathbb{N}$
and $\mathbb{R}$ mean that zero is excluded from the set. Moreover,
$\nnul^{\infty} = \nnul \cup \left\{\infty \right\}$ is the set of natural numbers including $\infty$. 
%Asimismo, denotaremos por $Q$ los n\'{u}meros racionales, y por
%$Q^{+}$ los n\'{u}meros racionales positivos.

\item {\bf Sets and Multisets}\\
We will use the standard delimiters for sets (\{\}) and multisets(\multiset{}). 
As usual, let $A$ be a multiset and $R\,:\,A \longrightarrow \nnul$, we say 
that $x \in A$ iff $R(x) > 0$.
Moreover, we abuse the notion using $x \in A$ to represent that $x$ is an element of
the multiset $A$ and using indistinctly the typical set operators (union, inclusion, etc.). The cardinality of the set (multiset) $A$
is denoted by $|A|$. Given a set $A$,
${\mathcal B}(A)$ is the set of all finite multisets over $A$.

For any $C_1, C_2 \in {\cal B}(A)$, we define:
\begin{itemize}
%\item $\nnul$ will denote the set of natural numbers,
%$\nat=\{0,1,2,\ldots\}$ and $\Sigma = \nat \times \nat$.
%%
%\item Multisets are defined as functions $C\,:\,X \rightarrow \nnul$,
%providing us with the number of instances of each element $x\in X$.
%As usual, we will enumerate the elements of a multiset $C$ as follows:
%$C = \{ r_1.x_1,\ldots,r_n.x_n \}$, meaning that $C(x_i)=r_i$,
%for all $i=1,\ldots,n$, and $C(x)=0$, for all $x \neq x_i, \, i=1,\ldots,n$.
%
%The set of multisets over a set $X$ will be denoted by ${\cal B}(X)$.
%For any $x \in X$ and $C\in {\cal B}(X)$ we say that $x \in C$
%if and only if $C(x)>0$.
%
%
\item $C_1 + C_2 \in {\cal B}(A)$, where $\forall x \in A\,:\,
        (C_1+C_2)(x)= C_1(x) + C_2(x)$.
  %
\item $C_1 \subseteq C_2$ if and only if $\forall x \in A\,:\,
        C_1(x) \leq C_2(x)$.
  %
\item If $C_2 \subseteq C_1$ we can define the subtraction 
        $C_1 - C_2 \in {\cal B}(A)$, where $\forall x \in A\,:\,
        (C_1-C_2)(x)= C_1(x) - C_2(x)$.

\item For any $C \in {\cal B}(\Sigma)$, 
we define the first projection
$\Pi_1(C) \in {\cal B}(\nnul)$, 
as follows: $\forall n \in \nnul,\,
\Pi_1(C)(n) = \sum_{m\in \nnul} C(n,m)$.
%
\item For any $C \in {\cal B}(\Sigma)$
and $n\in \nnul$ we define the second projection 
$\Pi_2(C,n)$ as the ordered list that consists of the
elements $(m_1,m_2,\ldots,m_{\Pi_1(C)(n)})$,
such that $(n,m_i) \in C$, $\forall i=1,\ldots, \Pi_1(C)(n)$
and $m_i \leq m_{i+1}$,
$\forall i=1,\ldots, \Pi_1(C)(n)-1$.
%
\item For any $C_1,C_2 \in {\cal B}(\Sigma)$, we say that
$C_1 \preceq C_2$ if and only if the following conditions hold:
  %
  \begin{itemize}
  \item $\Pi_1(C_1) \subseteq \Pi_1(C_2)$.
  %
  \item \mbox{$\forall n \in \nnul$}, taking \mbox{$\Pi_2(C_1,n) = (m^1_1,\ldots,
        m^1_{\Pi_1(C_1)(n)})$} and
        \mbox{$\Pi_2(C_2,n) = $} \linebreak
        $(m^2_1,\ldots,m^2_{\Pi_1(C_2)(n)})$,
        we must have  $m^1_i \geq m^2_i$, $\forall i=1,\ldots,
        \Pi_1(C_1)(n)$.
  \end{itemize}
%
%  
These conditions state that for every $n$ the total number of
elements $(n,m)$ (moving $m$) must be lesser in $C_1$ than in $C_2$, and 
for every element $(n,m)$ in $C_1$ there must be a 
corresponding (distinct element) $(n,m')$ in $C_2$, with $m\geq m'$.
%
\item For any $C_1,C_2 \in {\cal B}(\Sigma)$, with $C_1 \preceq C_2$,
we define $C_2 \ominus C_1$ in the following (recursive) way: 
  %
  \begin{itemize}
  \item For $C_1 = \emptyset$ we take $C_2 \ominus C_1 = C_2$.
  %
  \item For $C_1 \neq \emptyset$, let us consider that 

        \noindent $C_2 = \{ r^1_1.(n_1,m^1_1), \ldots,
         r^1_{i_{n_{1}}}.(n_1,m^1_{i_{n_{1}}}),
        \ldots, r^k_1.(n_k,m^k_1),\ldots, r^k_{i_{n_{k}}}.(n_k,m^k_{i_{n_{k}}}) \}$,
        where $n_l \neq n_j$, $\forall l \neq j$, and
        $m^l_j < m^l_{j+1}$, $\forall l=1,\ldots,k$ and $\forall
        j=1,\ldots,i_{n_{l}}$.

        Since $C_1 \preceq C_2$, we can take one element $(n_l,m) \in C_1$, 
        for some $l\in \{1,\ldots,k\}$, as well as the largest index $j$
        for which $m^l_{j} \leq m$. We then define recursively:
       
        \noindent $C_2 \ominus C_1 = %\begin{array}[t]{l}
        (
        \{ r^1_1.(n_1,m^1_1), \ldots, r^1_{i_{n_{1}}}.(n_1,m^1_{i_{n_{1}}}),
        \ldots,
        r^l_1.(n_l,m^l_1), \ldots,(r^l_{j} - 1).(n_l,m^l_j),\\
        \ldots,
        r^l_{i_{n_{l}}}.(n_l,m^l_{i_{n_{l}}}), 
        \ldots, r^k_1.(n_k,m^k_1),\ldots, r^k_{i_{n_{k}}}.(n_k,m^k_{i_{n_{k}}}) \}
        ) 
        \ominus (C_1 - \{ 1.(n_l,m) \})
        %\end{array}
        $.

        %It is immediate that this recursive definition is correct.
        Thus, $C_2 \ominus C_1$ is obtained by removing from $C_2$ 
        elements $(n,m)$ that correspond to elements $(n,m')$ of $C_1$,
        such that $m$ is the largest value with $m\leq m'$.

        % 
        \vspace*{0.12cm}

        For instance, taking $C_1 = \{
             1.(2,3), 1.(2,5), 1.(1,4), 1.(7,6)
        \}$, and $C_2 = $ \linebreak
        $\{
             1.(2,0), 1.(2,1), 1.(2,2), 1.(1,3), 2.(7,6), 3.(3,3)
        \}$ it follows that $C_1 \preceq C_2$. Then,
        $C_2 \ominus C_1= \{
            1.(2,0), 1.(7,6), 3.(3,3)
        \}$.
  \end{itemize} 
\end{itemize}

\item {\bf Relations}\\
Let $X$ be a set, a relation over $X$ is a set $R \subseteq X \times X$. 
The domain (or the set of departure) of $R$, denoted by $dom(R)$, is:
\[dom(R) = \{ x \in X \,|\, \exists
y \in X\,:\, (x,y) \in R \}\]
and the codomain (or the set of destination) of $R$, denoted by $cod(R)$, is:
\[cod(R) = \{ x \in X \,|\, \exists\ y \in X\,:\,(y,x) \in R\}\]
Given a relation $R$, the {\it reflexive and transitive closure} of $R$, $R^*$, is defined as follows:
\[ R^* = \{ (x,y)\,|\,x=y \,\vee\,
\exists\ x_1,\ldots,x_n\, \text{s.t.} \,(x,x_1)\in R\ldots(x_n,y) \in R\}\]
Moreover, the {\it transitive closure} of $R$, $R^+$, is given by:
\[R^+ = \{ (x,y)\,|\,
\exists \ x_1,\ldots,x_n \text{ s.t. } (x,x_1)\in R,\ldots,(x_n,y) \in R\}\]

%\item {\bf Vectores}\\
%La notaci\'{o}n empleada para representar los vectores
%ser\'{a} la usual, mediante tuplas. En el caso de vectores con componentes
%en $\nnul$, diremos que $v \geq w$ sii todas las componentes de $v$
%son mayores o iguales que las correspondientes de $w$. Adem\'{a}s,
%diremos que $v > w$ si $v \geq w$ y $v \neq w$.
\end{enumerate}


\section{Petri nets}

\begin{definition} [(Basic Petri nets)]
A \emph{basic Petri net} (PN) is a triple $N=(P,T,F)$, where $P$ and $T$
are sets and $F$ is a relation defined over $P \,\cup\,T$. Moreover, it has to satisfy
the following constraints:
\begin{enumerate}
\item $P \,\cap \,T = \emptyset$
\item $F \subseteq (P \times T) \,\cup\, (T \times P)$
\item $dom(F) \, \cup \, cod(F) = P \, \cup \, T$
\end{enumerate}

In a Petri net, $P$ is known as the set of \emph{places} of $N$, $T$ 
is the set of {\it transitions} and $F$ is a flow relation between the places in $P$
and the transitions in $T$. This relation is graphically represented by arcs.
In this Thesis, we suppose that the sets $P$ and $T$ are finite. Petri nets
can be graphically represented by means of bipartite graphs (or bigraphs), which
is a graph whose vertices can be divided into two disjoint sets ($P$ and $T$ in this case) such that 
every edge connects an element from $P$ to $T$, and vice versa. In graphical representation, places are drawn
as circles and transitions as rectangles or boxes.  The places 
from which an arc runs to a transition are called the \emph{input places} of the transition, whereas
the places from which an arc runs from the transition are called the \emph{output places}.

Let $X = P\,\cup\,T$ be a set and $x \in X$
an element of this set. The preset of $x$ is
$\precond{x} = \{ y \in X \,|\, (y,x) \in F\}$, whereas the postset of $x$ 
is defined as $x^{\bullet} = \{ y \in X \,|\, (x,y) \in F\}~$.

A net $N$ is $T$-restricted iff $\precond{t} = t^{\bullet} =
\emptyset\,\,\,\ \text{for all } t \in T$.
\qed
\end{definition}

\begin{example} Let $N=(P,T,F)$ be a Petri net such that:
\[\begin{array}{l}
P = \{ p_1,\,p_2,\,p_3\}\\
T = \{ t_1,\,t_2\}\\
F = \{ (p_1,t_1),\,(p_2,t_1),\,(t_1,p_3),\,(p_3,t_2)\}
\end{array}\]

This net is depicted in Figure \ref{fig201}.
\end{example}

\begin{figure}
\setlength{\unitlength}{0.0125in}
\begin{picture}(130,170)(46,638)
\thicklines
\put(267,722){\circle{16}}
\put(300,802){\circle{16}}
\put(241,803){\circle{16}}
\put(267,712){\vector( 0,-1){ 37}}
\put(267,755){\vector( 0,-1){ 25}}
\put(290,800){\line(-2,-3){ 10}}
\put(280,785){\vector(-1,-4){  5}}
\put(250,800){\line( 2,-3){ 10}}
\put(260,785){\vector( 0,-1){ 20}}
\put(255,666){\framebox(25,7){}}
\put(255,756){\framebox(25,7){}}
\put(279,730){\makebox(0,0)[lb]{\raisebox{0pt}[0pt][0pt]{ $p_3$}}}
\put(283,676){\makebox(0,0)[lb]{\raisebox{0pt}[0pt][0pt]{ $t_2$}}}
\put(284,764){\makebox(0,0)[lb]{\raisebox{0pt}[0pt][0pt]{ $t_1$}}}
\put(309,810){\makebox(0,0)[lb]{\raisebox{0pt}[0pt][0pt]{ $p_2$}}}
\put(220,812){\makebox(0,0)[lb]{\raisebox{0pt}[0pt][0pt]{ $p_1$}}}
\end{picture}




\caption{\label{fig201} Example of a basic Petri net.}
\end{figure}

Graphically, places in a Petri net may contain a discrete number of marks called \emph{tokens}. 

\begin{definition} [(Marking on basic Petri nets)]
Let $N=(P,T,F)$ be a basic Petri net.
The function $M: P \longrightarrow
\nnul$ is called {\it the marking of $N$}. Then,
$(P,T,F,M)$ is called a {\it marked Petri net}.
\end{definition}

Any distribution of tokens over the places will represent a configuration of the net called \emph{marking}.
The marking of a Petri net is graphically represented by drawing
in each place as many dots as tokens correspond,
or putting into each place the number of tokens associated with it. 


\begin{example} In the net of Figure \ref{fig201}, we can consider the following marking:
\[ M(p_1) = 1,\,\,\,M(p_2)=1,\,\,\,M(p_3)=0 \]
The graphical representation of this marking is shown in Figure \ref{fig202}.
\end{example}

\begin{figure}
\setlength{\unitlength}{0.0125in}
\begin{picture}(130,170)(46,638)
\thicklines
\put(267,722){\circle{16}}
\put(300,802){\circle{16}}
\put(300,802){\circle*{2}}
\put(241,803){\circle{16}}
\put(241,803){\circle*{2}}
\put(267,712){\vector( 0,-1){ 37}}
\put(267,755){\vector( 0,-1){ 25}}
\put(290,800){\line(-2,-3){ 10}}
\put(280,785){\vector(-1,-4){  5}}
\put(250,800){\line( 2,-3){ 10}}
\put(260,785){\vector( 0,-1){ 20}}
\put(255,666){\framebox(25,7){}}
\put(255,756){\framebox(25,7){}}
\put(279,730){\makebox(0,0)[lb]{\raisebox{0pt}[0pt][0pt]{$p_3$}}}
\put(283,676){\makebox(0,0)[lb]{\raisebox{0pt}[0pt][0pt]{$t_2$}}}
\put(284,764){\makebox(0,0)[lb]{\raisebox{0pt}[0pt][0pt]{$t_1$}}}
\put(309,810){\makebox(0,0)[lb]{\raisebox{0pt}[0pt][0pt]{$p_2$}}}
\put(220,812){\makebox(0,0)[lb]{\raisebox{0pt}[0pt][0pt]{$p_1$}}}
\end{picture}




\caption{\label{fig202} Example of a marked Petri net.}
\end{figure}

The semantics of a Petri net is defined by the following \emph{firing rule}, which
represents the marking reached after firing a transition.

\begin{definition}[(Enabling rule)]
Let $N=(P,T,F,M)$ a marked Petri net. A transition $t \in T$
is {\it enabled} by the marking M, denoted by $M[t\rangle$, if for all place
$p \in P$ such that $(p,t) \in F$, $M(p) > 0$.
\end{definition}

\begin{definition}[(Firing rule)]
The \emph{firing} of a transition $t$ enabled by the marking $M$
produces a new marking on the net, $M'$, defined as:
\[M'(p) = M(p) - W_f(p,t) + W_f(t,p)~~~\forall p \in P\]
where for all $x \in (T \times P) \, \cup \, (P \times T)$,  
$W_f(x) = 1$ if $x \in F$ and $W_f(x) = 0$, if $x \not\in F$,.
This is denoted by $M[t\rangle M'$.
\end{definition}

\begin{example} In Figure \ref{fig202}, the firing of the transition $t_1$
creates the marking $M'$:
\[ M'(p_1) = 0,\,\,\,M'(p_2)=0,\,\,\,M'(p_3)=1 \]
\end{example}

\begin{definition} [(Concurrent enabling of transitions)]
Let $N= (P,T,F,M)$ be a marked Petri net and $R \subseteq T$ a subset of transitions of $N$.
The set of transitions $R$ is \emph{concurrently enabled}, denoted by $M [ R \rangle$
iff $M(p) \geq \sum_{t \in R} W_f(p,t),~
\forall p\in P$, where
$W_f(p,t)$ is defined as in the last definition.

We can also extend this definition to multisets, thus allowing
multiple instances of the same transition to be fired in just one step.
In this way, we say that the multiset of transitions $R$ 
is enabled in $M$ iff $M(p)
\geq \sum_{t \in T} W_f(p,t) \cdot R(t)$, $\forall p \in P$.

The firing of the multiset of transitions $R$ in $M$ 
produces the new marking $M'$ of $N$:
\[ M'(p) = M(p) - \sumas{t \in T} (W_f(p,t) - W_f(t,p)) \cdot R(t)\]
This net evolution in just one step is denoted by
$M[ R \rangle M'$.
\end{definition}

\chapter{Extended Petri nets}\label{chapter:extended}
After introducing web services (and their composition) and the basic formal models
that can be used to model and analyse them, we will focus on this chapter in defining the specific models used in this Thesis. Thus, we
will present the extensions of the basic model of Petri nets stated previously and some properties
that can be analysed. We will recall some notions (marking, firing, enabledness and so on) defined in the preceding chapter
and will adapt them to a particular case. This chapter is mainly divided in two parts. On the one hand,
we will focus on the definition of Coloured Petri nets since they are used in the definition of the language BPELRF, whereas
we will present timed-arc Petri nets, in its two variants (discrete and continuous), as they are the basis of the workflow model presented in the
second part of this Thesis.

\begin{definition} [(General Petri nets)]
A general Petri net is a 5-tuple $N=(P,T,F,K,W)$, where:
\begin{enumerate}
\item $(P,T,F)$ is a basic Petri net.
\item $K\,:\,P \longrightarrow \nnul \,\cup\, \{\infty\}$ is a function that
indicates the maximum number of tokens in each place.
\item $W\,:\,F \longrightarrow \nnul$ is a function that indicates
the multiplicity of the arcs ({\it weight of the arcs}).
\end{enumerate}
When the context is clear, we will call them Petri nets. The function $K$
can be omitted if it is infinite for all the places in the net.
%Adem\'{a}s, es
%usual extender la definici\'{o}n de $W$ a todo el universo de
%posibles arcos, haci\'{e}ndola nula para pares
%$(p,t)$ o $(t,p)$ que no est\'{e}n en $F$.
\end{definition}

\begin{definition} [(Firing rule for general Petri nets)]
Let $N=(P,T,F,K,W)$ be a general Petri net.
\begin{enumerate}
\item A function $M\,:\, P \longrightarrow \nnul$ is a marking
$N$ iff $M(p) \leq K(p)$, for all
$p \in P$.
\item A transition $t \in T$ is enabled in $M$, denoted by $M[ t \rangle$,
iff $W(p,t) \leq M(p) \leq K(p) - W(t,p)$, for all $p \in P$.
The firing of $t$ produces the marking $M'$:
$M'(p) = M(p) - W(p,t) + W(t,p)$, for all $p \in P$.
Again, this evolution is denoted by $M[ t \rangle M'$.
\item A multiset of transitions $R$ is enabled in $M$, written $M [ R \rangle$, if and only if
$M(p) \geq \sum_{t \in T} W(p,t) \cdot R(t)$. The firing of $R$
produces $M'$:
\[ M'(p) = M(p) - \sum_{t \in T} (W(p,t) - W(t,p)) \cdot R(t), \,\,
\forall p \in P\], denoted by $M [ R \rangle M'$.
\end{enumerate}
\end{definition}

\begin{definition} [(Occurrence Sequence)]
Let $N=(P,T,F,K,W,M_0)$ be a marked Petri net.
\begin{enumerate}
\item $\sigma = M_0 t_1 M_1 \ldots t_n M_n$ is a finite occurrence sequence of $N$
if and only if $\forall i \in \{1,\ldots,n\},\,M_{i-1} [ t_i \rangle M_i$.
Occasionally, we will write $t_1 \ldots t_n$, omitting the corresponding markings, since starting from
$M_0$ it is easy to obtain the rest of the markings knowing the transitions fired.
We extend the conventional notation to occurrence sequence, 
obtaining $M_0 [ \sigma \rangle M_n$.
The set of occurrence sequences starting from $M_0$ are denoted by $L(N,M_0)$.

\item An occurrence sequence $\sigma = M_0 R_1 M_1 \ldots R_n M_n$ is finite
iff $\forall i \in \{1,\ldots,n\},\,M_{i-1} [ R_i \rangle M_i$.
%Again, we extend the notation to occurrence sequence, $M_0 [ \sigma \rangle M_n$.
The set of occurrence sequence of $N$ starting from $M_0$ is denoted by $P(N,M_0)$.
\end{enumerate}
\end{definition}

%En lo sucesivo trabajaremos usualmente sobre la sem\'{a}ntica de
%secuencias de ocurrencia, salvo que expl\'{\i}citamente se
%indique lo contrario.

%\bdfn (Matrices de Incidencia)\\
%Sea $N=(P,T,F,W,M_0)$ una Red de Petri Marcada.
%\begin{enumerate}
%\item Se dice que $N$ es {\it pura} sii $\forall t \in T$,
%$\forall p \in P$, $W(t,p) \cdot W(p,t) = 0$.
%\item Si $N$ es una red pura, podemos definir su
%{\it matriz de incidencia previa}, $C^{-} = (c_{i,j}^{-})$,
%$i=1,\ldots,|P|\,;\,j=1,\ldots,|T|$, siendo
%$c_{i,j}^{-} = W(p_i,t_j)$, y su
%{\it matriz de incidencia posterior}
%$C^{+} = (c_{i,j}^{+})$,
%$i=1,\ldots,|P|\,;\,j=1,\ldots,|T|$, siendo
%$c_{i,j}^{+} = W(t_i,p_j)$.
%\item Si $N$ es una red pura se define su {\it matriz de
%incidencia} $C$ por medio de $C = C^+ - C^-$.
%\end{enumerate}
%\edfn

%La matriz de incidencia puede ser definida tambi\'{e}n sobre redes
%que no sean puras, pero en tal caso no caracteriza a las mismas, pues
%una misma matriz de incidencia corresponde a varias redes diferentes.
%
%\bex Es sencillo obtener dos Redes de Petri diferentes con la misma
%matriz de incidencia. Para ello basta tomar
%una Red de Petri Ordinaria y elegir un lugar y una transici\'{o}n
%no conectados inicialmente, y conectarlos formando un loop. Por
%ejemplo, las dos redes de la figura \ref{fig203} tienen la misma
%matriz de incidencia. De hecho, si se trabaja con Redes Generalizadas
%el a\~{n}adido se puede hacer sobre cualquier par.
%\eex
%
%\begin{figure}
%\input{fig28}
%\caption{\label{fig203} Dos Redes de Petri con la misma Matriz de Incidencia}
%\end{figure}

%\bprop (Ecuaci\'{o}n de Estado)\\
%Sea $N=(P,T,F,W,M_0)$ una Red de Petri Marcada Pura,
%$\sigma \in L(N,M_0)$, y $M_0 [ \sigma \rangle M$.
%Entonces se tiene: $M = M_0 + C \cdot \bar{\sigma}$, siendo
%$\bar{\sigma}$ el vector de Parikh asociado a la
%secuencia $\sigma$, que est\'{a} definido como
%$\bar{\sigma}(i) = $n\'{u}mero de ocurrencias de
%la transici\'{o}n $t_i$ en la secuencia $\sigma$.
%
%\proof Sea la secuencia de marcajes producida a lo
%largo de la ejecuci\'{o}n de $\sigma$: $M_0 t_{i_{1}} M_1 \ldots t_{i_{n}} M_n$.
%Entonces, de la regla de disparo se concluye que $M_1 = M_0 +
%C \cdot U_{i_{1}}$, siendo $U_{i_{1}}$ el vector cuyas componentes son todas
%nulas, salvo la $i_{1}$-\'{e}sima, que vale 1. En general, se obtiene
%que $M_k = M_{k-1} + C \cdot U_{i_{k}}$.
%
%Por tanto:
%\[ M_k = M_{k-2} + C \cdot (U_{i_{k-1}} + U_{i_{k}}) = \ldots =
%M_0 + C \cdot \sum_{j=1}^{k} U_{i_{j}}\]
%Ahora bien, $\suma{j=1}{k} U_{i_{j}} = \bar{\sigma}$, lo que
%termina la demostraci\'{o}n.
%\eprop

\section{Petri nets analysis}
When designing a new system, the construction of a graphical model (e.g. a Petri net) of it is always
helpful since it is interesting to broadly understand how this system works. This also helps designers to have a
deeper knowledge about it. Nevertheless, the presence of a graphical model is not enough in many cases
as the designers want the system to meet some properties of interest. For instance, the system is useless
if it reaches a deadlock in each execution.  To this end, it is important to have tools that allow to
evaluate properties in the model. In finite sequential systems, it is not particularly challenging to check
the fulfilment of certain statement, whereas the presence of concurrency complicates this task.
The analysis of systems behaviour is intended to determine the compliance of 
certain properties such as that the number of processes in a queue does not exceed
certain threshold or that the mutual exclusion is guaranteed when accessing to a shared resource.

In Petri nets, one can use a set of powerful tools to formally analyse
the compliance of such properties. Among these tools, designers can check
the absence of deadlocks, the reachability of a certain
state, the possibility of reaching a concrete situation after performing some computations and so on.


\medskip
Normally, these properties are divided in two categories \cite{citar}:

\subsection{Safety properties}
A safety property asserts that \emph{``nothing bad happens''}.
Thus, they guarantee that a set of undesirable states are not reached or that
the system does not execute an unwanted occurrence sequence.

The \emph{safety properties} are the following:

\begin{enumerate}
\item {\bf Reachability}. A marking $M$ of a marked Petri net
$N= (P,T,F,W,M_0)$ is {\it reachable} in $N$
iff there exists an occurrence sequence $\sigma \in L(N,M_0)$
such that $M_0 [ \sigma \rangle M$. We will denote by $[M_0\rangle$ the
set of reachable markings of $N$ starting from $M_0$, and
by $[ M \rangle$ the set of reachable markings starting from the marking $M$.

\item {\bf Boundedness}. A marked Petri net $N=(P,T,F,W,M_0)$ 
is {\it k-bounded}, for some $k \in \nnul$, if all reachable marking
$M$ from $M_0$ hold that $M(p) \leq k$, for all $p \in P$. $N$ is safe if
it is 1-bounded. A place $p \in P$ is
{\it n-safe} if $M(p) \leq n$, for all markings $M$ reachable from $M_0$.

%\item {\bf Boundedness}. Sea
%$N=(P,T,F,W,M_0)$ una Red de Petri Marcada.
%Se dice que un lugar $p \in P$ es limitado si
%existe un n\'{u}mero natural $n \in \nnul$ tal que dicho lugar es
%$n$-seguro; y se dice que $N$ es {\it limitada} si todos sus lugares
%son limitados.
\item {\bf Deadlock-free}. Let $N=(P,T,F,W,M_0)$ a marked Petri net and $M$ be a reachable marking.
$M$ is a {\it dead marking} if there is no $t \in T$ enabled in $M$. 
The net $N$ is deadlock-free iff there are no dead markings.

%\item {\bf Conservaci\'{o}n Respecto a un Vector de Pesos}.
%Sea una Red de Petri Marcada $N=(P,T,F,W,M_0)$, con
%$P = \{p_1,\ldots,p_n\}$. Se dice que $N$ es
%{\it conservativa respecto a un vector de pesos w}, con
%$w \in \nnul^n$, si para todo marcaje alcanzable $M$
%a partir de $M_0$ se cumple:
%\[ \suma{i=1}{n} w_i \cdot M(p_i) =
%\suma{i=1}{n} w_i \cdot M_0(p_i)\]
\item {\bf Coverability}.
Let $N=(P,T,F,W,M_0)$ be a marked Petri net and $M$ be a marking of $N$.
$M$ is said to be {\it coverable} if there exists $M' \in
[M_0 \rangle$ such that $M' \geq M$.
\end{enumerate}

%\noindent {\sc NOTA:} Para ser m\'{a}s precisos, la alcanzabilidad no es
%exactamente una propiedad de seguridad, sino la negaci\'{o}n de la propiedad,
%la no alcanzabilidad. An\'{a}logamente ocurre con el cubrimiento de marcajes.

\subsection{Liveness properties}
A liveness property asserts that \emph{``something good eventually happens''}.
For instance, they guarantee that, independently of the current state of the system,
a specific state can eventually be reached or that a certain occurrence sequence can eventually
be executed in the system.

The \emph{liveness properties} are:
\begin{enumerate}
\item {\bf Liveness}.
Let $N=(P,T,F,W,M_0)$ be a marked Petri net. A transition $t \in T$
is said to be {\it live} if for all reachable marking $M \in
[ M_0 \rangle$ there is an occurrence sequence $\sigma$ starting from $M$ such that
$\sigma = t_1 \ldots t_m$, with $t_m = t$. The $N$ is {\it live} iff all the transitions are live.
\item {\bf Home State}.
Let $N=(P,T,F,W,M_0)$ be a marked Petri net. A marking $M$ of $N$ is a {\it home state} if for all
$M' \in [ M_0 \rangle$, $M \in [ M' \rangle$.
\item {\bf Home Space}. Let $N=(P,T,F,W,M_0)$ be a marked Petri net.
The set of markings ${\cal M}$ is a {\it home-space}
of $N$ if for all marking $M' \in [ M_0 \rangle$ there is marking
$M'' \in {\cal M}$ such that $M'' \in [ M' \rangle$.
\item {\bf Cyclic}.
Let $N=(P,T,F,W,M_0)$ be a marked Petri net. It is said that
$N$ is cyclic if for all marking $M \in [ M_0 \rangle$ there exists an occurrence sequence $\sigma$
starting in $M$ such that $M [ \sigma \rangle M_0$.
\end{enumerate}


% %JIRI % % % %



\section{Timed extensions of Petri nets}

In the
literature on timed extensions of Petri nets we can identify a
first group of models, which assign time delays to transitions,
by using either a fixed deterministic value
\cite{Ram73,Sif77,VFC93} or choosing it from a probability
distribution \cite{AjCh85}. Other models use time intervals to
establish the enabling times of transitions \cite{Mer74}. 
There are also models that introduce time on tokens
\cite{van93,van95,BLT90}. In \cite{Bow96,Wan98} 
a description is given of the different approaches 
to introduce time in Petri nets.

Extender un poco mas esto y luego introducir las nuestras




\section{Proofs from Section~\ref{sec:tawfn}} \label{app:undecid}
%s~\ref{sec:def} and~\ref{sec:tawfn}}

\noindent {\bf Theorem~\ref{thm:undecid}.}
{\em
Soundness is undecidable for extended timed-arc workflow nets.
This is the case also for MTAWFNs that contain additionally only
inhibitor arcs, age invariants or urgent transitions but not necessarily
all of them together.
}
\begin{proof}
The proofs are by reduction from the Minsky machine.
A \emph{Minsky machine} with two nonnegative counters $c_1$ and $c_2$ is
a sequence of labelled instructions
$$1:\mathsf{inst}_1;\ 2:\mathsf{inst}_2;\ \ldots, n:\mathsf{inst}_n$$
where $\mathsf{inst}_n = \mathsf{HALT}$ 
and each $\mathsf{inst}_i$, $1\le i < n$, is of one of the following forms
\begin{itemize}
\item
{(Inc)} \hspace{5mm} {\tt $i$: $c_j$++; goto $k$}
\item
{(Dec)} \hspace{4mm} {\tt $i$: if $c_j$=$0$ then goto $k$ else 
 ($c_j$--; goto $\ell$)}
\end{itemize}
for $j\in \{1,2\}$ and $1\le k,\ell \le n$.

Instructions of type (Inc) are called \emph{increment} instructions and
those of type (Dec) are called \emph{test and decrement} instructions.
A configuration is a triple $(i, v_1, v_2)$ where $i$ is the current
instruction and $v_1$ and $v_2$ are the values of the counters $c_1$
and $c_2$, respectively. A computation step between configurations is
defined in the natural way. If starting from the initial
configuration $(1,0,0)$ the machine reaches the instruction $\mathsf{HALT}$
then we say it \emph{halts}, otherwise it \emph{loops}.
%
The problem whether a given Minsky machine halts 
is undecidable~\cite{Minsky:book}. W.l.o.g. we assume
that the machine halts only when both counters are empty
(we can add a few instructions that will always empty the counters
before reaching the halting instruction).

We shall now reduce reachability in Minsky machines into the soundness
problem on ETAWFN. Counters $c_1$ and $c_2$ will be simulated
by two places $p_{c_1}$ and $p_{c_2}$ such that the number of
tokens in those places represents the value of the counters.
For every instruction label $i$, $1 \leq i \leq n$, we
add a new control place $p_i$. At any moment exactly one of the $p_i$
places will be marked by a token, representing the instruction
to be executed in the next step.

If we allow urgent transitions, we can create for any given Minsky machine
a workflow net constructed according to the patterns given in Figure~\ref{fig:urgent-reduction} (we only show
the encoding of the instructions that manipulate the first counter;
the encoding for the second counter is completely analogous).
We also postulate that the input place is $in=p_1$ and the output place is
$out=p_n$. Now, given the initial marking with one token in $p_1$,
the net will faithfully simulate the (deterministic) computation of
the Minsky machine. This is clear for the increment instruction
as the control token moves from $p_i$ to $p_k$ and the number of tokens
in $p_{c_1}$ is increased by one. For the test and decrement instruction,
if $p_{c_1}$ contains at least one token then the transition
$t_i^{\mathit{dec}}$ will be fired with no delay (the transition is urgent),
decreasing the counter by one and moving the control token to $p_\ell$
as required. Only if the counter $c_1$ is empty (there are no tokens
in $p_{c_1}$), we are allowed to delay one time unit and fire the transition
$t_i^{\mathit{zero}}$ such that the control token is moved to $p_k$.
Hence the test and decrement instruction is also faithfully simulated
and there is no possibility of any deadlock situation, meaning that
either $t_i^{\mathit{dec}}$ or $t_i^{\mathit{zero}}$ can always fire.
It is now an easy observation that the workflow net is sound if and only
if the Minsky machine halts.


\begin{figure}[!t]
\centering
\subfloat[][{\tt $i$: $c_1$++; goto $k$}]{
\begin{tikzpicture}[font=\scriptsize,xscale=1.5,yscale=1.5]
   \tikzstyle{arc}=[->,>=stealth,thick]
   \tikzstyle{transportArc}=[->,>=diamond,thick]
   \tikzstyle{every place}=[minimum size=6mm,thick]
   \tikzstyle{every transition}=[fill=black,minimum width=2mm,minimum height=5mm]
   \tikzstyle{every token}=[fill=white,text=black]
  \node[place,label=above:$p_{i}$] at (0,0) (pi) {};
  \node[place,label=above:$p_{k}$] at (2,0) (pk) {};
  \node[place,label=above:$p_{c_1}$] at (1,1) (pc1) {};
  \node[transition,label=60:$t_i$] at (1,0) (t) {};
  \draw[arc] (pi) -- (t) node[midway,above]{$[0,\infty]$} {};
  \draw[arc] (t) -- (pk) {};
  \draw[arc] (t) -- (pc1) {};
\end{tikzpicture}
    \label{fig:inc}} 
\hspace{4mm}
\subfloat[][{\tt $i$: if $c_1$=$0$ then goto $k$ else 
   ($c_1$--; goto $\ell$)}]{
\begin{tikzpicture}[font=\scriptsize,xscale=1.9,yscale=1.5]
  \tikzstyle{arc}=[->,>=stealth,thick]
  \tikzstyle{transportArc}=[->,>=diamond,thick]
  \tikzstyle{every place}=[minimum size=6mm,thick]
  \tikzstyle{every transition}=[fill=black,minimum width=2mm,minimum height=5mm]
  \tikzstyle{every token}=[fill=white,text=black]
  \tikzstyle{urgency}=[place,fill=white,minimum size=2.0mm,thin]
  \node at (-0.85,0) {};
  \node at (2.8,0) {};
  \node[place,label=above:$p_{i}$] at (0,0) (pi) {};
  \node[place,label=above:$p_{k}$] at (2,0) (pk) {};
  \node[place,label=above:$p_{\ell}$] at (2,1) (pl) {};
  \node[place,label=above:$p_{c_1}$] at (0,1) (pc1) {};
  \node[transition,label=60:$t_i^{\mathit{dec}}$] at (1,1) (tdec) {};
  \node[urgency] at (1,1) {};
  \node[transition,label=60:$t_i^{\mathit{zero}}$] at (1,0) (tzero1) {};
 \draw[arc] (pi) -- (tdec) node[midway,left]{$[0,\infty]$} {};
 \draw[arc] (pi) -- (tzero1) node[midway,above]{$[1,\infty]$} {};
 \draw[arc] (pc1) -- (tdec) node[midway,above,pos=0.5]{$[0,\infty]$} {}; 
 \draw[arc] (tdec) -- (pl) {};
 \draw[arc] (tzero1) -- (pk) {};
\end{tikzpicture}
   \label{fig:dec}}
  \caption{Simulation of (Inc) and (Dec) instructions with urgent transitions}%
  \label{fig:urgent-reduction}%
\end{figure}

\begin{figure}[t!]
  \centering
  \subfloat[][Simulation of the counters $c_1$ and $c_2$]{
    %\input{figures/simu_register}
\begin{tikzpicture}[font=\scriptsize, xscale=.9,yscale=1.6]
   \tikzstyle{arc}=[->,>=stealth,thick]
   \tikzstyle{transportArc}=[->,>=diamond,thick]
   \tikzstyle{every place}=[minimum size=6mm,thick]
   \tikzstyle{every transition}=[fill=black,minimum width=2mm,minimum height=5mm]
   \tikzstyle{every token}=[fill=white,text=black]
  \node[place,label=above:$p_{c_1}$,label=270:{inv: $\leq 2$}] at (0,0) (p1) {};
  \node[place,label=above:$p^{\mathit{reset}}_{c_1}$](p2) at (4,0) {};
  \node[transition] at (2,0)[label=above:$t^{\mathit{reset}}_{c_1}$] {} 
    edge [pre,bend right=10]  node[above,xshift=0mm,pos=0.5]{$[2,2]$} (p1) 
    edge [post,bend right=10] (p2)
    edge [pre,bend left=10] node[above,xshift=0mm,pos=0.5]{$[0,\infty]$}(p2) 
    edge [post,bend left=10] (p1);

 \node[place,label=above:$p_{c_2}$,label=270:{inv: $\leq 2$}] at (6.5,0) (p1c) {};
  \node[place,label=above:$p^{\mathit{reset}}_{c_2}$](p2c) at (10.5,0) {};
  \node[transition] at (8.5,0)[label=above:$t^{\mathit{reset}}_{c_1}$] {}
    edge [pre,bend right=10]  node[above,xshift=0mm,pos=0.5]{$[2,2]$} (p1c)
    edge [post,bend right=10] (p2c)
    edge [pre,bend left=10] node[above,xshift=0mm,pos=0.5]{$[0,\infty]$}(p2c)
    edge [post,bend left=10] (p1c);
\end{tikzpicture}
    \label{fig:register_simulation}
  }%
%  \qquad\qquad
%\\
%  \subfloat[][Simulation start]{
%\begin{tikzpicture}[font=\scriptsize,scale=1.6]
%   \tikzstyle{arc}=[->,>=stealth,thick]
%   \tikzstyle{transportArc}=[->,>=diamond,thick]
%   \tikzstyle{every place}=[minimum size=6mm,thick]
%   \tikzstyle{every transition}=[fill=black,minimum width=2mm,minimum height=5mm]
%   \tikzstyle{every token}=[fill=white,text=black]
  %\node[place, label=above:$in$] at (0,1) (pin) {};
  %\node[place, label=above:$p_1$] at (2,1) (p1) {};
  %\node[transition, label=80:$t_{\mathit{start}}$] at (1,1) {}
  %  edge [pre] node[above] {$[0,\infty]$} (pin) {}
  %  edge [post] (p1) {};
%\end{tikzpicture}
%    \label{fig:start}
%  }
  \\
  \subfloat[][{\tt $i$: $c_1$++; goto $k$}]{
%    \input{figures/simu_increment}
\begin{tikzpicture}[font=\scriptsize,scale=.9]
   \tikzstyle{arc}=[->,>=stealth,thick]
   \tikzstyle{transportArc}=[->,>=diamond,thick]
   \tikzstyle{every place}=[minimum size=6mm,thick]
   \tikzstyle{every transition}=[fill=black,minimum width=2mm,minimum height=5mm]
   \tikzstyle{every token}=[fill=white,text=black]
  \node[place,label=above:$p_{i}$,label=below:{inv: $\leq 1$}] at (0,0) (pj) {};
  \node[place,label=-90:$p^{\mathit{reset}}_{c_1}$](presetri) at (5,-1.4) {};
  \node[place,label=below:{inv: $\leq 1$},label={[label distance=-0mm]90:$q_{i}$}](q1) at (5,0) {};
  \node[place,label=90:$p^{\mathit{reset}}_{c_2}$](presetr3i) at (5,1.4) {};
  \node[place,label=above:$p_{c_1}$](pri) at (0,1.4) {};
  \node[place,label=above:$p_k$,label=below:{inv: $\leq 1$}](pk) at (10,0) {};

  \node[transition] at (2.5,0)[label=above:$t_i$] {}
    edge [pre] node[above,pos=0.6]{$[1,1]$} (pj) 
    edge [post] (q1) 
    edge [post, bend right=20] (pri) 
    edge [post, bend right] (presetri) 
    edge [post, bend left] (presetr3i);
  \node[transition] at (7.5,0)[label={[yshift=1mm,xshift=1mm]above:$t^{\mathit{goto}}_i$}] {}
    edge [pre, bend left] node[above,sloped,pos=0.6] {$[0,\infty]$} (presetri) 
    edge [pre] node[above,sloped,pos=0.55] {$[1,1]$} (q1) 
    edge [pre, bend right] node[above,sloped,pos=0.7] {$[0,\infty]$} (presetr3i) 
    edge [post] (pk) {};
\end{tikzpicture}
    \label{fig:increment_simulation}
  } 
  \\
  \subfloat[][{\tt $i$: if $c_1$=$0$ then goto $k$ else 
        ($c_1$--; goto $\ell$)}.]{
    %\input{figures/simu_test_decrement}
\begin{tikzpicture}[font=\scriptsize,xscale=0.9,yscale=1.5]
   \tikzstyle{arc}=[->,>=stealth,thick]
   \tikzstyle{transportArc}=[->,>=diamond,thick]
   \tikzstyle{every place}=[minimum size=6mm,thick]
   \tikzstyle{every transition}=[fill=black,minimum width=2mm,minimum height=5mm]
   \tikzstyle{every token}=[fill=white,text=black]

  \node[place,label=above:$p_{i}$,label=below:{inv: $\leq 1$}] at (0,1) (pj) {};
  \node[place,label=above:$p_{c_1}$,label=220:$$](pri) at (0,-1.5) {};
  \node[place,label=below:{inv: $\leq 1$},label={[label distance=-1mm]60:$q_{i}$}](q1) at (5,0.3) {};
  \node[place,label=below:{inv: $\leq 1$},label={[label distance=-1mm]60:$q'_{i}$}](q1p) at (5,-0.7) {};
  \node[place,label=above:$p^{\mathit{reset}}_{c_{2}}$](presetr3i) at (5,1) {};  
  \node[place,label=below:$p^{\mathit{reset}}_{c_{1}}$](prs) at (5,-1.5) {};  
  \node[place,label=right:$p_k$,label=below:{inv: $\leq 1$}](pl) at (10,1) {};
  \node[place,label=right:$p_\ell$,label=below:{inv: $\leq 1$}](pk) at (10,-1.5) {};

  \node[transition, label=above:$t_i$] at (2.5,1) {}
    edge [pre] node[auto,swap,yshift=0mm]{$[1,1]$} (pj)
    edge [post,bend right=15] (q1)
    edge [post] (presetr3i);
  \node[transition, label=above:$t^{\mathit{zero}}_i$] at (7.5,1) {}
    edge [pre, bend left=15] node[right,pos=0.4,xshift=0.9mm]{$[1,1]$} (q1)
    edge [pre] node[auto,swap,xshift=-0mm,yshift=-.0mm]{$[0,\infty]$} (presetr3i)
    edge [post] (pl);
  \node[transition, label=below:$t^{\mathit{dec}}_i$] at (2.5,-1.5) {}
    edge [pre] node[above,yshift=.0mm]{$[2,2]$} (pri)
    edge [pre,bend left=22] node[auto,yshift=-1.4mm]{$[0,0]$} (q1)
    edge [post,bend left=15] (q1p)
    edge [post] (prs);
\node[transition, label=below:$t^{\mathit{end}}_i$] at (7.5,-1.5) {}
    edge [pre,bend left=0] node[below,yshift=.0mm]{$[0,\infty]$} (prs)
    edge [pre,bend right=15] node[above,yshift=0.5mm]{$[1,1]$} (q1p)
    edge [pre,bend right=30] node[right,yshift=.0mm,pos=0.3]{$[0,\infty]$} (presetr3i)
    edge [post] (pk);
\end{tikzpicture}
    \label{fig:test_decrement_simulation}
  }

  \caption{Simulation of a Minsky machine by a workflow net with invariants}%
  \label{fig:cont}%
\end{figure}

%A very similar reduction can be done for workflow nets that can, instead
%of urgent transitions, employ inhibitor arcs. In order to guarantee
%that the simulation is fair, we simply add an inhibitor arc from the
%counter place $p_{c_1}$ to the transition $t_i^{\mathit{zero}}$.
%Otherwise the net is completely untimed and again, the workflow net is
%sound if and only if the Minsky machine halts.

The reduction for workflow nets that contain only age invariants
is more complicated. The reduction idea is based on~\cite{memics_paper}, 
however, it had to be nontrivially modified in order to avoid the
large number of possible deadlocks introduced in the reduction.

The counters are now modelled by two places that contain age invariants
ensuring that no tokens can get older than $2$, see 
Figure~\ref{fig:register_simulation}. The intuition is that
before and after the simulation of any instruction, all tokens in
the places $p_{c_1}$ and $p_{c_2}$ are exclusively of age $1$. 
As before the input place is $\itn{in}=p_1$ and the output place
is $\itn{out}=p_n$.

Let us now observe
that this invariant is preserved after simulating the 
increment instruction (Figure~\ref{fig:increment_simulation}).
Assume that all tokens in the counter places are of age $1$ and that
the place $p_i$ contains one token of age $0$.
%(initially created by the initialization transition in Figure~\ref{fig:start}
%that just resets the token age to $0$). 
Before $t_i$ can be fired, one time unit must pass and this guarantees
that all tokens in the counter places will become of age $2$. After firing
of $t_i$, we also add one token of age $0$ to $p_{c_1}$ 
and moreover, a token of age $0$ in the place $q_i$ is created.
Before we can proceed and delay one time unit 
and then fire $t_i^{\mathit goto}$, we must fire the
transitions $t_{c_1}^{\mathit reset}$ and $t_{c_2}^{\mathit reset}$ once
for every token of age $2$ in the counter places in order to reset 
them all to the age $0$, otherwise the age invariant $\leq 2$ in the token 
places disables the delay of one time unit. Clearly, after the transition 
$t_i^{\mathit goto}$ is fired, all counter tokens are again of age $1$
(including the one added to $p_{c_1}$) and we argued for a faithful simulation
of the increment instruction.

Let us consider now the test and decrement instruction simulated by
the net in Figure~\ref{fig:test_decrement_simulation}. Again, let us assume
that all counter tokens are of age $1$ and that there is one token
of age $0$ in $p_i$. First we wait one time unit and then fire $t_i$,
meaning again that all counter tokens are of age $2$. Now all
tokens in the place $p_{c_2}$ can be reset to $0$ and if $p_{c_1}$
does not contain any tokens, we can wait one time unit and fire
$t_i^{\mathit zero}$, implying that we place a token to $p_k$
as expected and all counter tokens are again of age $1$.
If on the other hand $p_{c_1}$ contains some tokens (all of age $2$
as we already mentioned), we must without any delay fire
$t_i^{\mathit dec}$ consuming one token from $p_{c_1}$ while marking
the places $q'_i$ and $p^{\mathit reset}_{c_1}$, allowing
now all counter tokens to be reset to $0$. After this we can wait
one time unit and fire the transition $t_i^{\mathit end}$, while
continuing with the execution of the instruction with label $\ell$.
All tokens in the counter places are now again of age $1$.
Notice that these two scenarios are deterministically determined
by the presence or absence of tokens in $p_{c_1}$ and that there are
no deadlock situations possible during the simulation.

As a result
%, and assuming that the output place of the workflow net is $out = p_n$, 
we can see that the net is sound if and only if
the Minsky machine halts. This completes the undecidability proof
also for the situation where we use only age invariants.
\qed
\end{proof}

